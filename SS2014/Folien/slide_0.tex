\documentclass{beamer} %%% FÜR VORTRAG MIT PAUSEN
%\documentclass[handout]{beamer}  %%% FÜR HANDOUT ALS PDF

\setbeamertemplate{navigation symbols}{}
\usetheme{Madrid}
\usecolortheme{seagull}
\beamersetuncovermixins{\opaqueness<1>{25}}{\opaqueness<2->{15}}
\usepackage[english]{babel}
\usepackage{amssymb,amsmath}


\begin{document}
\author[Beccarini/Mutschler]{Dr. Andrea Beccarini $\qquad$ Willi Mutschler, M.Sc.}
\date{Summer 2014}
\institute[]{Institute of Econometrics and Economic
Statistics\\University of M\"unster}
\title[DSGE]{Dynamic Stochastic General Equlibibrium Models}
\subtitle{Overview}

\begin{frame}
\titlepage
\end{frame}

\begin{frame}\frametitle{Topics of the course}
\begin{enumerate}
\item \textbf{Introduction}
	\begin{enumerate}
	\item New Keynesian Theory and its main counterparts
	\item Dynamic Programming
	\end{enumerate}
\item \textbf{Main theoretical features of the Smets-Wouters model}
	\begin{enumerate}
	\item Households, Firms, Central Bank, Government
	\item Log-Linearization
	\item Structural and reduced-form
	\end{enumerate}
\item \textbf{The Econometrics of DSGE models}
	\begin{enumerate}
	\item General model representation(s)
	\item Solution methods: linear vs. nonlinear approximation
	\item Identification of DSGE Models
	\item Calibration and Impulse-responses
	\item Classical estimation methods: GMM and Maximum Likelihood
	\item Bayesian estimation methods and model evaluation
	\end{enumerate}
\item \textbf{Further topics (if we have time):} 
	\begin{enumerate}
		\item Nonlinear estimation methods: Extended/Quadratic Kalman Filter, Particle Filter
		\item Regime switching models
	\end{enumerate}
\end{enumerate}
\end{frame}

\begin{frame}{About the Course}
\begin{itemize}
\item Timetable: 
	\begin{itemize}
	\item Begin of the course: 07.04.2014
	\item Mondays 16.00-18:00, CAWM 3
	\item Fridays 10.00-12.00, CAWM 3
	\end{itemize}
\item Note: There is no distinct separation between lectures and classes
\item Parts 1 and 2 will be taught by Andrea Beccarini
\item Parts 3 and 4 will be taught by Willi Mutschler, please bring your own laptop, since we will heavily use Matlab and Dynare (www.dynare.org)
\item Students who wish to get credit for the course are asked to write a thesis (15-20 pages) covering both a summary as well as an extension or application of one of the topics covered in the course.
\end{itemize}
\end{frame}

\begin{frame}\frametitle{Bibliography}
Recommended Readings
\begin{itemize}
\item An and Schorfheide (2007). "Bayesian Analysis of DSGE Models".
\item Canova (2007). Methods for Applied Macroeconomic Research.
\item DeJong and Dave (2011). Structural Macroeconometrics.
\item Smets and Wouters (2002, 2004). "An Estimated Stochastic Dynamic General Equilibrium Model of the Euro Area"
\end{itemize}
Further references will be given during the course.

\end{frame}
\end{document}
