\documentclass[handout]{beamer}  %%% FÜR HANDOUT ALS PDF

\setbeamertemplate{navigation symbols}{}
\usetheme{Madrid}
\usecolortheme{seagull}
\beamersetuncovermixins{\opaqueness<1>{10}}{\opaqueness<2->{15}}
\usepackage[english]{babel}
\usepackage{csquotes}
\usepackage{graphicx}
\usepackage{verbatim}
\newcounter{saveenumi}
\newcommand{\seti}{\setcounter{saveenumi}{\value{enumi}}}
\newcommand{\conti}{\setcounter{enumi}{\value{saveenumi}}}
\usepackage[hyperref=true,style=authoryear, dashed=false, maxnames=3,backend=bibtex8,doi=false,isbn=false,backref=true]{biblatex}
\usepackage{amssymb,amsmath}
%\setlength{\bibitemsep}{\baselineskip}
\usepackage{xcolor}
\usepackage{graphicx,pstricks,beamerprosper}
\usepackage{subfigure}
\usepackage{verbatim}
\usepackage{cancel}
\renewcommand{\CancelColor}{\red}
\begin{document}
\author[Willi Mutschler]{Willi Mutschler, M.Sc.}
\date{Summer 2013}
\institute[Institute of Econometrics]{Institute of Econometrics and Economic
Statistics\\University of M\"unster\\willi.mutschler@uni-muenster.de}
\title[DSGE methods]{Advanced Macroeconomics (PhD) - DSGE methods}
\subtitle{Solution methods}

\begin{frame}
\titlepage
\end{frame}

\section{DSGE framework}

\begin{frame}
\frametitle{\secname}\footnotesize
DSGE model consists of
\begin{itemize}[<+->]
     \item set of Euler equations, i.e. first-order optimality conditions,
     \item transition equations for state $(x_t)$ and control variables $(y_t)$,
     \item transition equations for structural shocks and innovations $(\varepsilon_t)$,
     \item measurement equation with $D$ selecting observable variables $(d_t)$  out of controls and measurement errors $\mu_t$,
     \item vector of deep parameters $\theta$,
\end{itemize}\pause
which can be cast into a nonlinear first-order system of expectational difference equations $$E_t f \left( x_{t+1},y_{t+1},x_t,y_t|\theta \right)= 0.$$
\begin{block}{Perturbation approach}
\begin{itemize}
  \item introduce perturbation parameter $\sigma$ that captures stochastic nature of the model
  \item solution is characterized by \emph{policy-functions}, $g$ and $h$, that solve (at least approximately) the system of equations $f$.
\end{itemize}
\end{block}
\end{frame}


\begin{frame}
\frametitle{\secname}
\begin{block}{General DSGE model, perturbation approach}
\begin{align*}
   0 &=   E_t f \left( x_{t+1},y_{t+1},x_t,y_t|\theta \right),\\
 x_{t+1}  &= h(x_{t},\sigma|\theta) + \sigma \varepsilon_{t+1},\\
   y_t &= g(x_t,\sigma|\theta) ,\\
   d_t &= D y_t+  \mu_{t}.
\end{align*}
with $E(\varepsilon_t)=0$, $E(\varepsilon_t\varepsilon_t')=\Sigma_\varepsilon$ and $E(\mu_t)=0$, $E(\mu_t\mu_t')=\Sigma_\mu$.
\end{block}

\begin{block}{Non-stochastic steady-state}
\begin{eqnarray*}
&\overline{z}:=(\bar{x}',\bar{y}')',~ \bar{x}=h(\bar{x},0|\theta),~ \bar{y}=g(\bar{x},0|\theta),~\bar{d} = D \bar{y}
\end{eqnarray*}
\end{block}
\end{frame}

\begin{frame}
\frametitle{\secname}
Some remarks
\begin{itemize}
   \item  DSGE-models can be interpreted as \emph{state-space-models}.
\item Driving force of the model are exogenous shocks and innovations $\varepsilon_t$.
\item Innovations can be mutually correlated as well as the conditional variance can change over time, this features are captured by $\Sigma_\varepsilon$.
\item State vector $x_t$ consists of $n_{x_1}$ endogenous predetermined states and $n_{x_2}$ exogenous predetermined states
 \begin{align*}
   x_{2,t+1} = h^2(x_{2,t},\sigma|\theta) + \sigma \varepsilon_{t+1}, \qquad x_{1,t+1} = h^1(x_{2,t},\sigma|\theta)
 \end{align*}
 \item Stability: Eigenvalues of Jacobian of $h^2(\cdot)$ w.r.t. $x_{2,t}$ lie inside the unit circle
\item Flexible framework: you can add auxiliary variables and equations.
 \item Important property: $x_t,y_t,d_t$ are weakly stationary (or $\varepsilon_t$ is $i.i.d$).
\end{itemize}
\end{frame}

\section{DSGE solution}
\begin{frame}
\frametitle{\secname}
One distinguishes between linear and non-linear methods:
    \begin{itemize}
   \item Linear methods: Anderson/Moore (1983), Binder and Pesaran, Blanchard/Khan (1980),
       (1997), Christiano (2002), King and Watson (1998), $\boxed{\text{Klein (2000)}}$, Sims (2001) and Uhlig (1999) (See Anderson (2008) for a comparison).
   \item Nonlinear methods: Projection methods, iteration procedures or perturbation approach (see DeJong/Dave (2011) for a comparison)
\end{itemize}
Our Focus will be on perturbation approach in the fashion of Schmitt-Groh\'{e}/Uribe (2004):
\begin{itemize}
\item Perturbation approach finds a \underline{local} approximation of the policy functions in a neighborhood of a particular point
\item Mostly steady-state, since we can solve it analytically or numerically.
\end{itemize}
\end{frame}

\subsection{First-order approximation}
\begin{frame}
\frametitle{\secname}\framesubtitle{\subsecname}
\begin{itemize}
\item We are looking for approximations to $g$ and $h$ around the point $(x,\sigma)=(\bar{x},0)$ of the form
\begin{align*}
  g(x,\sigma) &= g(\overline{x},0) + g_x(\overline{x},0)(x-\overline{x}) + g_\sigma(\overline{x},0)(\sigma-0)\\
  h(x,\sigma) &= h(\overline{x},0) + h_x(\overline{x},0)(x-\overline{x}) + h_\sigma(\overline{x},0)(\sigma-0)
\end{align*}
\item We would like to know $g_x,h_x,g_\sigma,h_\sigma$.
\item Idea:
\begin{itemize}
\item Substitute the solution into the model,
\begin{align*}
F(x,\sigma):=E_t f \left( \underbrace{h(x_{t},\sigma)+\sigma \varepsilon_{t+1}}_{x_{t+1}},\underbrace{g[h(x_{t},\sigma)+\sigma \varepsilon_{t+1},\sigma]}_{y_{t+1}},\underbrace{x_t}_{x_t},\underbrace{g(x_t,\sigma)}_{y_t} \right)
\end{align*}
\item Insight: $f$ as well as all derivatives of $f$ are $0$ when evaluated at the non-stochastic steady state $(\overline{x},0)$.
\end{itemize}
\end{itemize}
\end{frame}

\begin{frame}
\frametitle{\secname}\framesubtitle{\subsecname, computing $g_\sigma,h_\sigma$}
\begin{itemize}
  \item Take derivative of $F$ w.r.t. $\sigma$ and evaluate at the non-stochastic steady-state
  \footnotesize\begin{eqnarray*}
&F_\sigma(\overline{x},0) = E_t f_1[h_\sigma + \varepsilon_{t+1}] + E_t f_2 [g_x(h_\sigma+ \varepsilon_{t+1})+g_\sigma] + f_3\cdot 0 + f_4 g_\sigma = 0\\
 & \Leftrightarrow \begin{bmatrix} \underset{(n \times n_x)}{f_1 + f_2 g_x} & \underset{(n\times n_y)}{f_2 +f_4}\end{bmatrix} \begin{bmatrix} \underset{(n_x \times 1)}{h_\sigma} \\ \underset{(n_y \times 1)}{g_\sigma} \end{bmatrix} = \underset{n \times 1}{0}
  \end{eqnarray*}
with $f_1=\partial f(\overline{z})/\partial x_{t+1}, f_2=\partial f(\overline{z})/\partial y_{t+1}, f_3=\partial f(\overline{z})/\partial x_{t}, f_4=\partial f(\overline{z})/\partial y_{t}$\normalsize
  \item We have n equations in n unknowns
  \item Notice that this is a linear and homogenous system, that is, if there is a unique solution, it must be $$h_\sigma=g_\sigma=0$$
\end{itemize}
\end{frame}

\begin{frame}
\frametitle{\secname}\framesubtitle{\subsecname, Certainty-equivalence}
Important theoretical result:
\begin{itemize}
  \item Even though agents take the effect of future shocks into account when optimizing, in a linearization to the first-order they don't matter for the decision rule.
\begin{block}{\emph{Certainty-equivalence-property}}
       \begin{itemize}
         \item In a first-order approximation the constant term needs not to be corrected for uncertainty (i.e. variance of shocks)
         \item Expectation of $x_t$ and $y_t$ is equal to its non-stochastic steady-state
       \end{itemize}
\end{block}
\item This is problematic when uncertainty does matter: risk premia, welfare evaluation, \dots
\end{itemize}
\end{frame}


\begin{frame}
\frametitle{\secname}\framesubtitle{\subsecname, computing $g_x,h_x$}\footnotesize
\begin{itemize}
  \item Take derivative of $F$ w.r.t. $x_t$ and evaluate at the non-stochastic steady-state
  \begin{eqnarray*}
&    F_x (\overline{x},0) = f_1 h_x + f_2 g_x h_x + f_3 + f_4 g_x = 0\\
& \underbrace{- \begin{bmatrix} \underset{(n\times n_x)}{f_1} & \underset{(n\times n_y)}{f_2} \end{bmatrix}}_{:=A} \begin{bmatrix} \underset{(n_x\times n_x)}{h_x} \\ \underset{(n_y\times n_x)}{g_x} \underset{(n_x\times n_x)}{h_x} \end{bmatrix}  = \underbrace{\begin{bmatrix} \underset{(n\times n_x)}{f_3} & \underset{(n\times n_y)}{f_4}\end{bmatrix}}_{:=B} \begin{bmatrix} \underset{(n_x\times n_x)}{I} \\\underset{(n_y\times n_x)}{g_x} \end{bmatrix}
  \end{eqnarray*}
  \item $n\times n_x$ equations for $n\times n_x$ unknown elements of $h_x$ and $g_x$
  \item Postmultiply by $\widehat{x}_t := (x_t-\overline{x})$
  \begin{align*}
    A \begin{bmatrix} h_x \widehat{x}_t \\ g_x h_x \widehat{x}_t \end{bmatrix} = B \begin{bmatrix} \widehat{x}_t \\ g_x \widehat{x}_t \end{bmatrix}
  \end{align*}
  \item Notice that this is equivalent to the first order approximation of the model
  \begin{eqnarray*}
    A\begin{bmatrix} \widehat{x}_{t+1} \\ E_t \widehat{y}_{t+1} \end{bmatrix} = B\begin{bmatrix} \widehat{x}_t\\ \widehat{y}_t \end{bmatrix} + \begin{bmatrix} \sigma \varepsilon_{t+1} \\ 0\end{bmatrix}
  \end{eqnarray*}
\end{itemize}
\end{frame}


\begin{frame}
\frametitle{\secname}\framesubtitle{\subsecname, computing $g_x,h_x$}
\begin{itemize}
\item $A$ and $B$ are known matrices, inverting A yields solution for $h_x$ and $g_x$
\item \textbf{BUT:} In general A is singular and non-invertible.
  \item We follow Klein (2000)'s approach to approximate the first order solution using the generalized Schur decomposition
  \item Idea:
  \begin{itemize}
      \item Uncouple system into a (block) triangular system of equations using a particular matrix decomposition method, and solve system recursively.
  \end{itemize}
\item We'll need some matrix theory about decompositions and eigenvalues for that...
\end{itemize}
\end{frame}

\section{Aside: Matrix theory}


\begin{frame}
\frametitle{\secname}\framesubtitle{}\footnotesize
\begin{block}{Matrix pencil}
  Let $A$ and $B$ be two $n\times n$ matrices. The set of all matrices of the form $A-\lambda B$ with $\lambda \in \mathbb{C}$ is said to be a \emph{pencil}. The eigenvalues of the pencil are elements of the set $\lambda(A,B)$ defined by
  \begin{align*}
    \lambda(A,B) =  \left\{z \in \mathbb{C}:det(A-zB)=0\right\}
  \end{align*}
\end{block}
\begin{block}{Generalized Eigenvalue problem}
  Let $A$ and $B$ be two $n\times n$ matrices. Then $\lambda \in \lambda(A,B)$ is called a generalized Eigenvalue if there exist a nonzero vector $q\in \mathbb{C}^n$ such that $$A q = \lambda B q$$
\end{block}
\begin{itemize}
\item If $B=I$, then this simplifies to the ordinary Eigenvalue problem: $A q = \lambda q$
\item A always has $n$ eigenvalues, which can be ordered (in more than one way) to form an $n\times n$ diagonal matrix $\Lambda$ and a corresponding matrix of nonzero columns $Q$ that satisfies the eigenvalue equation: $$AQ = Q\Lambda$$
\end{itemize}
\end{frame}

\begin{frame}
\frametitle{\secname}\framesubtitle{}
\begin{block}{Eigendecomposition (or spectral decomposition) of a matrix}
  Let $A$ be a squared $(n\times n)$ matrix with $n$ linearly independent columns, then $A$ can be factorized as
  \begin{align*}
  AQ = Q \Lambda \Leftrightarrow A = Q \Lambda Q^{-1}
  \end{align*}
   $\Lambda$ is a diagonal matrix such that $\lambda_i =\Lambda_{ii}$ is the Eigenvalue of $A$ associated to the eigenvector $q_i$ stored in column $i$ of the square $(n \times n)$ matrix $Q$.
\end{block}
\begin{itemize}
  \item Not every matrix has a Eigendecomposition, since $Q$ might not be invertible (sufficient: Eigenvalues must be distinct).
  \item Useful for solving differential or difference equations, computing powers of covariance matrices, etc., example: $$x_{t} = A x_{t-1} = A^t x_{0}=Q\Lambda^t Q^{-1} x_{0}$$
  \item $\Lambda^t$ is very easy to calculate, whereas $A^t$ is computationally very demanding.
\end{itemize}
\end{frame}

\begin{frame}[shrink]
\frametitle{\secname}\framesubtitle{}\scriptsize
\begin{block}{Jordan decomposition}
  Let $A$ be an $n \times n$ matrix. Denote by $J_k(\lambda)$ a $k\times k$ matrix of the form (a so-called Jordan-block)
  \begin{align*}
    J_k(\lambda) = \begin{bmatrix}
      \lambda & 1 & 0 & \dots & 0\\
      0 & \lambda & 1 & \dots & 0\\
      \vdots & \vdots & \vdots & \dots & \vdots\\
      0 & 0& 0 & \dots & 1\\
      0 & 0 &0 & \dots & \lambda
    \end{bmatrix}
  \end{align*}
  where $J_1(\lambda)=\lambda$. Then there exists a non-singular $n\times n$ matrix $T$ such that
  \begin{align*}
    T^{-1} A T = \begin{bmatrix}
      J_{k_1}(\lambda_1) & 0 & \dots & 0 \\
      0 & J_{k_2}(\lambda_2) & \dots & 0 \\
      \vdots & \vdots & \dots & \vdots\\
      0 & 0 & \dots & J_{k_r}(\lambda_r)
    \end{bmatrix}
  \end{align*}
  with $k_1+k_2+\dots+k_r$ = n. The $\lambda_i$ are the eigenvalues of $A$, not necessarily distinct.
\end{block}
\begin{itemize}
  \item $T^{-1} A T$ is structured, i.e. upper triangular and diagonal elements are eigenvalues of $A$
  \item Special case: If $A$ has distinct eigenvalues, then $T^{-1}AT = \Lambda$, with $\Lambda$ diagonal.
  \item Useful for proofs due to its block structure, but numerically difficult due to instabilities.
\end{itemize}
\end{frame}

\begin{frame}[shrink]
\frametitle{\secname}\framesubtitle{}\footnotesize
\begin{block}{Schur decomposition (Complex version)}
  Let $A$ be an $n \times n$ matrix. Then there exist a \textbf{unitary} $n\times n$ matrix $S$ (that is, $S^*S=SS^*=S^{-1}S=I_n$) and an upper triangular matrix $M$ whose diagonal elements are the eigenvalues of $B$, such that
  \begin{align*}
    S^* A S = M \Leftrightarrow A = S M S^*
  \end{align*}
\end{block}
\begin{block}{Schur decomposition (Real version)}
  Let $A$ be a real symmetric $n \times n$ matrix. Then there exist an \textbf{orthogonal} real $n\times n$ matrix $S$ (that is, $S'S=SS'=S^{-1}S=I_n$), whose columns are eigenvectors of $A$ and a diagonal matrix $\Lambda$ whose diagonal elements are the eigenvalues of $A$, such that
  \begin{align*}
    S' A S = \Lambda \Leftrightarrow A = S \Lambda S'
  \end{align*}
\end{block}
\begin{itemize}
  \item $~^*$ denotes conjugate or Hermitian transpose, $~'$ denotes the ordinary transpose.
  \item A complex matrix always has a complex Schur decomposition.
  \item A real matrix has a real Schur decomposition if and only if all eigenvalues are real.
  \item $S$ is structured, i.e. unitary or orthogonal.
  \item Useful for proofs (e.g. Eigenvalues of Kronecker products, differentials,\dots) and numerically attractive.
\end{itemize}
\end{frame}

\begin{frame}[shrink]\scriptsize
\frametitle{\secname}\framesubtitle{}
\begin{block}{Generalized (complex) Schur decomposition or QZ decomposition}
Let $A$ and $B$ be $n\times n$ matrices. Then there exist matrices $Q,Z,T$ and $S$ such that
\begin{align*}
  Q^* A Z &= S \Leftrightarrow A = Q S Z^*\\
  Q^* B Z &= T \Leftrightarrow B = Q T Z^*
\end{align*}
\begin{enumerate}
  \item $Q$ and $Z$ are unitary, i.e. $Q^*Q=QQ^*=I_n$ and $Z^*Z=ZZ^*=I_n$.
  \item $S$ and $T$ are upper triangular.
 \item pairs $(s_{ii},t_{ii})$ can be arranged in any desired order.
  \item If for some $i$, $t_{ii}$ and $s_{ii}$ are both zero, then $\lambda(A,B)=\mathbb{C}$. Otherwise:
$\lambda(A,B) = \left\{\frac{t_{ii}}{s_{ii}}:s_{ii} \neq 0 \right\}$
\end{enumerate}
\end{block}
\begin{itemize}\scriptsize
  \item There is also a real version.
  \item We will limit ourselves to the case $\lambda(A,B)\neq \mathbb{C}$ and rule-out unit roots, that is $t_{ii}$ and $s_{ii}$ are not both zero, and $|t_{ii}|\neq |s_{ii}|$.
  \item Eigenvalues
  \begin{itemize}\scriptsize
  \item If A is singular, then there are some generalized eigenvalues missing, i.e. $s_{ii}=0$ for some $i$ $\Rightarrow$ call these \emph{infinite},
  \item If $|\lambda_i|>1 \Leftrightarrow |s_{ii}|<|t_{ii}|$ $\Rightarrow$ call these finite and unstable,
  \item If $|\lambda_i|<1 \Leftrightarrow |s_{ii}|>|t_{ii}|$ $\Rightarrow$ call these finite and stable.
  \end{itemize}
\end{itemize}
\end{frame}

\begin{frame}
\frametitle{\secname}\framesubtitle{}
\begin{block}{Singular value decomposition}
  Let $A$ be a real $m \times n$ matrix with $rank(A)=r>0$. Then there exist an $m\times r$ matrix $S$ such that $S'S=I_r$, an $n \times r$ matrix $T$ such that $T'T=I_r$ and an $r \times r$ diagonal matrix $\Lambda$ with positive diagonal elements, such that
  \begin{align*}
    A = S\Lambda^{1/2}T'
  \end{align*}
\end{block}
\begin{itemize}
  \item There is also a complex version using the conjugate transpose.
  \item Diagonal elements of $\Lambda^{1/2}$ are called singular values of $A$ and $m$ columns of $S$ and the $n$ columns of $T$ are called left-singular vectors and right-singular vectors of $A$, respectively.
  \item Convention: List singular values in descending order.
  \item $\Lambda$ contains the $r$ non-zero eigenvalues of $AA'$ (and of $A'A$) and $S$ and $T$ contain corresponding eigenvectors.
  \item Applications: Pseudoinverses, least squares, matrix approximations, rank, range, null space,\dots
\end{itemize}
\end{frame}

\section{DSGE solution}
\subsection{First-order approximation, continued}
\begin{frame}
\frametitle{\secname}\framesubtitle{\subsecname, computing $g_x$ and $h_x$}
Going back to our problem of finding $g_x$ and $h_x$:
  \begin{eqnarray*}
    A E_t \begin{bmatrix} \widehat{x}_{t+1} \\ \widehat{y}_{t+1} \end{bmatrix} = B\begin{bmatrix} \widehat{x}_t\\ \widehat{y}_t \end{bmatrix}
  \end{eqnarray*}
The Schur decomposition of $A$ and $B$ are given by
\begin{align*}
  Q^* A = S Z^*,\qquad   Q^* B = T Z^*
\end{align*}
where we choose the following order: the stable generalized eigenvalues $(|s_{ii}|>|t_{ii}|)$ come first (ascending order of $\lambda_i$). Premultiplying by $Q^*$ and using
$\begin{bmatrix} \underset{n_x \times 1}{s_t} \\ \underset{n_y\times1}{u_t} \end{bmatrix} : = Z^* \begin{bmatrix} \underset{n_x \times 1}{x_t} \\ \underset{n_y \times 1}{y_t} \end{bmatrix}$, we get
\begin{align*}
  S E_t \begin{bmatrix} s_{t+1}\\ u_{t+1} \end{bmatrix} = T \begin{bmatrix} s_t \\ u_t \end{bmatrix}
\end{align*}
\end{frame}

\begin{frame}
\frametitle{\secname}\framesubtitle{\subsecname, computing $g_x$ and $h_x$}
$S$ and $T$ are upper triangular:
\begin{align*}
\begin{bmatrix} \underset{n_x \times n_x}{S_{11}} & \underset{n_x \times n_y}{S_{12}} \\ \underset{n_y \times n_x}{0} & \underset{n_y \times n_y}{S_{22}}\end{bmatrix} E_t \begin{bmatrix} s_{t+1}\\ u_{t+1} \end{bmatrix} = \begin{bmatrix} \underset{n_x \times n_x}{T_{11}} & \underset{n_x \times n_y}{T_{12}} \\ \underset{n_y \times n_x}{0} & \underset{n_y \times n_y}{T_{22}}\end{bmatrix} \begin{bmatrix} s_t \\ u_t \end{bmatrix}
\end{align*}
Let's solve the lower block first, that is
\begin{align*}
  S_{22} E_t[u_{t+1}] = T_{22} u_t
\end{align*}
and note that due to our ordering we have unstable matrix pairs $|s_{ii}|<|t_{ii}|$ (or $|s_{ii}|\leq|t_{ii}|$ ). It can be shown that any solution with bounded mean and variance (we want that!) must satisfy $u_t=0$ for all $t$ (unless $\Sigma=0$), otherwise we would have an exploding solution.\\
Note: We need as many state variables as there are stable generalized eigenvalues!
\end{frame}

\begin{frame}
\frametitle{\secname}\framesubtitle{\subsecname, computing $g_x$ and $h_x$}
Now given $u_t=0$, let's back out $x_t$ and $y_t$ from the definition of $s_t$ and $u_t$:
\begin{align*}
  \begin{bmatrix} x_t \\ y_t \end{bmatrix} = \begin{bmatrix} Z_{11} & Z_{12}\\ Z_{21} & Z_{22} \end{bmatrix} \begin{bmatrix} s_t \\ u_t \end{bmatrix}
  \Rightarrow
  \begin{cases}
    x_t = Z_{11} s_t\\
    y_t = Z_{21} s_t
  \end{cases}
\end{align*}
If $Z_{11}$ is invertible, then
\begin{align*}
  y_t = \underbrace{Z_{21} Z_{11}^{-1}}_{=g_x} x_t
\end{align*}
Thus, in order to compute $g_x$ we need a nonsingular $Z_{11}$!
\end{frame}

\begin{frame}
\frametitle{\secname}\framesubtitle{\subsecname, computing $g_x$ and $h_x$}
Now given $u_t=0$ and a nonsingular $Z_{11}$, let's solve the first block
\begin{align*}
   E_t[s_{t+1}] &= S_{11}^{-1} T_{11} s_t
\end{align*}
assuming $S_{11}$ is invertible. Plugging in $s_t = Z_{11}^{-1} x_t$ we get
\begin{align*}
   E_t[ x_{t+1}] = \underbrace{Z_{11} S_{11}^{-1} T_{11} Z_{11}^{-1}}_{=h_x} x_t
\end{align*}
Thus, in order to compute $h_x$ we need nonsingular $S_{11}$ and $Z_{11}$. However, $S_{11}$ has full rank by construction.
\end{frame}

\begin{frame}
\frametitle{\secname}\framesubtitle{\subsecname, computing $g_x$ and $h_x$}\footnotesize
\begin{block}{Summary Klein (2000)}
\begin{itemize}
  \item Compute generalized Schur decomposition of $A=-\begin{bmatrix} f_1 & f_2\end{bmatrix}$ and $B=\begin{bmatrix} f_3 & f_4\end{bmatrix}$.
  \item Reorder generalized eigenvalues with the block inside the unit circle in the upper left.
  \item Check if $Z_{11}$ is invertible.
  \item Check number of stable eigenvalues, i.e. Blanchard-Khan condition.
  \item Compute $h_x=Z_{11} S_{11}^{-1} T_{11} Z_{11}^{-1}$ and $g_x=Z_{21} Z_{11}^{-1}$.
\end{itemize}
\end{block}
\begin{block}{Blanchard/Khan-conditions}
The number of Eigenvalues, that are in absolute terms greater than 1, must be equal to the number of state variables, in order to get a stable solution (saddle-path).
\end{block}
\end{frame}

\begin{frame}
\frametitle{\secname}\framesubtitle{First-order approximation}
\textbf{Pros}:
\begin{itemize}
   \item Simple linear state-space representation of the model, which in many cases is sufficiently exact.
   \item One can use the Kalman-filter to empirically evaluate the system.
\end{itemize}
\textbf{Cons}:
\begin{itemize}
   \item One looses important information during the linearization.
   \item Higher moments play an important role for analyzing markets, risk, welfare, etc.
   \item An approximation to, say, the second order can yield different results, because the variance of future shocks matters (risk premium).
\end{itemize}

\end{frame}

\subsection{Second-order approximation, Schmitt-Groh\'{e}/Uribe (2004)}
\begin{frame}
\frametitle{\secname}\framesubtitle{\subsecname}\footnotesize
The second-order approximations to $g$ and $h$ around the point $(\overline{x},0)$ are of the form
\begin{align*}
  [g(x,\sigma)]^i &= [g(\overline{x},0)]^i + [g_x(\overline{x},0)]^i_a [(x-\overline{x})]_a + [g_\sigma(\overline{x},0)]^i [\sigma-0]\\
  & + \frac{1}{2} [g_{xx}(\overline{x},0)]^i_{ab} [(x-\overline{x})]_a [(x-\overline{x})]_b\\
  & + \frac{1}{2} [g_{x\sigma}(\overline{x},0)]^i_{a} [(x-\overline{x})]_a [\sigma-0]\\
  & + \frac{1}{2} [g_{\sigma x}(\overline{x},0)]^i_{a} [(x-\overline{x})]_a [\sigma-0]\\
  & + \frac{1}{2} [g_{\sigma\sigma}(\overline{x},0)]^i [\sigma-0][\sigma-0]\\
  [h(x,\sigma)]^j &= [h(\overline{x},0)]^j + [h_x(\overline{x},0)]^j_a [(x-\overline{x})]_a + [h_\sigma(\overline{x},0)]^j [\sigma-0]\\
  & + \frac{1}{2} [h_{xx}(\overline{x},0)]^j_{ab} [(x-\overline{x})]_a [(x-\overline{x})]_b\\
  & + \frac{1}{2} [h_{x\sigma}(\overline{x},0)]^j_{a} [(x-\overline{x})]_a [\sigma-0]\\
  & + \frac{1}{2} [h_{\sigma x}(\overline{x},0)]^j_{a} [(x-\overline{x})]_a [\sigma-0]\\
  & + \frac{1}{2} [h_{\sigma\sigma}(\overline{x},0)]^j [\sigma-0][\sigma-0]
\end{align*}
where $i=1,\dots,n_y$; $a,b, = 1,\dots,n_x$; and $j=1,\dots n_x$.
\end{frame}

\begin{frame}
\frametitle{\secname}\framesubtitle{\subsecname}
This is so-called tensor notation:
\begin{itemize}
  \item $[f_2]_\alpha^i$ is the $(i,\alpha)$ element of the derivative of $f$ with respect to $y_{t+1}$. It is of dimension $n\times n_y$.
  \item $[f_2]_\alpha^i [g_x]_\beta^\alpha [h_x]_j^\beta=\sum_{\alpha=1}^{n_y}\sum_{\beta=1}^{n_x} \frac{\partial f^i}{\partial y_{t+1}^\alpha} \frac{\partial g^\alpha}{\partial x_{t}^\beta} \frac{\partial h^\beta}{\partial x_{t}^j}$
  \item $f_{11}$ is a three-dimensional array with $n$-rows, $n_y$ columns and $n_y$ pages. $[f_{11}]^i_{\alpha\gamma}$ denotes the element of $f_{11}$ located at the intersection of row $i$, column $\alpha$ and page $\gamma$.
  \item Aside note: There is also a different (prettier) way to do this using the Magnus/Neudecker definition of the Hessian, see Gomme/Klein (2011).
\end{itemize}
\end{frame}

\begin{frame}
  \frametitle{\secname}\framesubtitle{\subsecname, computing $g_{xx}$ and $h_{xx}$}\scriptsize
We can use the derivative of $F_x = f_1 h_x + f_2 g_x h_x + f_3 + f_4 g_x$ w.r.t. $x$ to identify $g_{xx}$ and $h_{xx}$:
\begin{align*}
  [F_{xx}(\overline{x},0)]^i_{jk} = &\left([f_{11}]^i_{\beta \delta} [h_x]_k^\delta + [f_{12}]^i_{\beta \gamma} [g_x]_\delta^\gamma [h_x]_k^\delta + [f_{13}]^i_{\beta k} + [f_{14}]^i_{\beta \gamma} [g_x]_k^\gamma\right)[h_x]_j^\beta\\
  &+[f_1]_\beta^i[h_{xx}]_{jk}^\beta\\
  &+\left([f_{21}]_{\alpha\delta}^i[h_x]_k^\delta + [f_{22}]_{\alpha\gamma}^i[g_x]_\delta^\gamma[h_x]_k^\delta + [f_{23}]_{\alpha k}^i + [f_{24}]_{\alpha \gamma}^i[g_x]_k^\gamma \right)[g_x]_\beta^\alpha[h_x]_j^\beta\\
  &+[f_2]_\alpha^i[g_{xx}]_{\beta\delta}^\alpha [h_x]_k^\delta [h_x]_j^\beta\\
  &+[f_2]_\alpha^i[g_x]_\beta^\alpha[h_{xx}]_{jk}^\beta\\
  &+[f_{31}]_{j\delta}^i[h_x]_k^\delta + [f_{32}]_{j\gamma}^i[g_x]_\delta^\gamma[h_x]_k^\delta + [f_{33}]_{jk}^i + [f_{34}]_{j\gamma}^i[g_x]_k^\gamma\\
  &+\left([f_{41}]_{\alpha\delta}^i[h_x]_k^\delta +[f_{42}]_{\alpha\gamma}^i[g_x]_\delta^\gamma[h_x]_k^\delta + [f_{43}]_{\alpha k}^i + [f_{44}]_{\alpha \gamma}^i [g_x]_k^\gamma \right)[g_x]_j^\alpha\\
  &+[f_4]_\alpha^i[g_{xx}]_{jk}^\alpha\\
  &= 0, \qquad i=1,\dots n,\qquad j,k,\beta,\delta = 1,\dots n_x, \qquad \alpha,\gamma=1,\dots n_y
\end{align*}
This is a linear system of $n\times n_x\times n_x$ equations in $n\times n_x\times n_x$ unknowns given by the elements of $g_{xx}$ and $h_{xx}$ $\Rightarrow$ easy to solve using matrix algebra (and a computer).
\end{frame}

\begin{frame}
  \frametitle{\secname}\framesubtitle{\subsecname, computing $g_{xx}$ and $h_{xx}$}
Similarly, we can use the derivative of $$F_\sigma = E_t f_1[h_\sigma + \varepsilon_{t+1}] + E_t f_2 [g_x(h_\sigma+\sigma \varepsilon_{t+1})+g_\sigma] + f_4 g_\sigma = 0$$ w.r.t. $\sigma$ to identify $g_{\sigma\sigma}$ and $h_{\sigma\sigma}$, we omit the expression here.\\ $\Rightarrow$ This yields a system of $n$ linear equations in the $n$ unknowns given by the elements of $g_{\sigma\sigma}$ and $h_{\sigma\sigma}$.
\end{frame}

\begin{frame}
\frametitle{\secname}\framesubtitle{\subsecname, computing $g_{\sigma x}$ and $h_{\sigma x}$}
Let's have a look at the cross-derivatives, i.e. $F_{\sigma,x}(\overline{x},0)=0$ using the fact that $g_\sigma=h_\sigma=0$:
\begin{align*}
  [F_{\sigma_x}]^i_j = [f_1]^i_\beta[h_{\sigma x}]_j^\beta + [f_{2}]_\alpha^i[g_x]_\beta^\alpha[h_{\sigma x}]_j^\beta + [f_2]_\alpha^i [g_{\sigma x}]_\gamma^\alpha [h_x]_j^\gamma + [f_4]_\alpha^i[g_{\sigma x}]_j^\alpha = 0
\end{align*}
$i=1,\dots, n; \alpha=1,\dots, n_y; \beta,\gamma,j=1,\dots,n_x$\\~\\
This is a system of $n\times n_x$ equations in $n\times n_x$ unknowns given by the elements of $g_{\sigma x}$ and $h_{\sigma x}$. Obviously, this is a linear \textbf{homogenous} system of equations, thus a unique solution implies $g_{\sigma x}=h_{\sigma x}=0$
\end{frame}
\subsection{Unconditional moments}
\begin{frame}
\frametitle{\secname}\framesubtitle{\subsecname}
\begin{itemize}
  \item Given stationary and the approximated solution, we can calculate the unconditional moments.
  \item Note: we focus on second order accurate moments, that is to keep only first and second order terms in the formula to compute moments.
  \item Compute variance from first order approximation alone, since all other cross-product terms result in something of order higher than two.
\end{itemize}
\begin{block}{Unconditional autocovariogram}\footnotesize
\begin{align*}
  \Sigma_x :=  E (x_{t}-\bar{x})(x_{t}-\bar{x})' &= h_x \Sigma_x h_x' + \sigma^2 \Sigma_\varepsilon\\
  \Sigma_y := E (y_{t}-\bar{y})(y_{t}-\bar{y})' & = g_x \Sigma_x g_x'\\
  \Sigma_d := E (d_{t}-\bar{d})(d_{t}-\bar{d})' & = D \Sigma_y D' + \Sigma_\mu\\
  \Sigma_x(t) := E[(x_t- \bar{x})(x_0- \bar{x})'] &= (h_x)^t \Sigma_x\\
  \Sigma_y(t) := E[(y_t- \bar{y})(y_0- \bar{y})'] &= g_x (h_x)^t \Sigma_xg_x'\\
  \Sigma_d(t) := E[(d_t- \bar{d})(d_0- \bar{d})'] &= D g_x (h_x)^t \Sigma_xg_x' D'
\end{align*}
\end{block}
\end{frame}

\begin{frame}
\frametitle{\secname}\framesubtitle{\subsecname}\footnotesize
Regarding the mean, we have
\begin{align*}
  x_{t+1}^i \approx \overline{x}^i + h_{x}^i (x_t - \overline{x}) + \frac{1}{2} \underbrace{(x_t - \overline{x})' h_{xx}^i (x_t - \overline{x})}_{(x_t-\bar{x})'\otimes(x_t-\bar{x})' vec(h_{xx}^i)}  + \frac{1}{2} h_{\sigma\sigma}^i + \sigma \varepsilon_{t+1}
\end{align*}

Notice that $E\left[(x_t-\bar{x})'\otimes(x_t-\bar{x})'\right]=vec(\Sigma_x)'$
\begin{block}{Unconditional mean}\footnotesize
\begin{align*}
  \mu_x &:= E (x_t) = \bar{x} + \left(I_{n_x}-h_x\right)^{-1} \frac{ \Lambda_x}{2},\quad
  \mu_y :=   E (y_t) = \bar{y} +g_x (\mu_x - \bar{x})+ \frac{ \Lambda_y}{2},\\
 with \\ \Lambda_x &=  \begin{bmatrix} vec(h_{xx}^1)'\\vec(h_{xx}^2)'\\ \dots \\ vec(h_{xx}^{n_x})'\end{bmatrix}  vec(\Sigma_x)  + \sigma^2 h_{\sigma\sigma},\quad
  \Lambda_y =  \begin{bmatrix} vec(g_{xx}^1)'\\vec(g_{xx}^2)'\\ \dots \\ vec(g_{xx}^{n_y})'\end{bmatrix} vec(\Sigma_x) + \sigma^2 g_{\sigma\sigma}.
\end{align*}
\end{block}
First order approximation: $\Lambda_x=\Lambda_y=0 \Rightarrow \mu_x = \bar{x}, \mu_y = \bar{y}$.
\end{frame}

\section{Summary}
\begin{frame}
\frametitle{\secname}
\begin{itemize}
  \item In a first-order approximation the unconditional mean is equal to the steady-state.
  \item Up to second order, the presence of uncertainty affects only the constant term of the decision rules
  \item[$\Rightarrow$] Unconditional mean can be significantly different from non-stochastic steady-state!
\end{itemize}
\end{frame}

\section{Further topic: Indeterminacy}
\begin{frame}
\frametitle{\secname}\framesubtitle{Lubik \& Schorfheide (2004)}
\begin{itemize}
  \item Consider a different log-linear (first-order) canonical form of the DSGE model (e.g. Sims (2002))
  \begin{align*}
    \Gamma_0(\theta) z_t = \Gamma_1(\theta) z_{t-1} + \Psi(\theta) \varepsilon_t + \Pi(\theta)\eta_t
  \end{align*}
  \item $z_t$ are all model variables, $\varepsilon_t$ are model shocks and $\eta_t=\widetilde{x}_t-E_{t-1}[\widetilde{x}_{t-1}]$ are expectational errors for the non-predetermined variables $\widetilde{x}$.
  \item $\Gamma_0,\Gamma_1, \Psi, \Pi$ are parameter matrices dependent on deep parameters $\theta$.
\end{itemize}
\end{frame}

\begin{frame}
\frametitle{\secname}\framesubtitle{Lubik \& Schorfheide (2004)}
\begin{itemize}
  \item Assume that there is a sunspot shock $\zeta_t$, i.e. the expectational errors are driven by
  \begin{align*}
    \eta_t = A_1 \varepsilon_t + A_2 \zeta_t
  \end{align*}
  \item Solution algorithms construct a mapping from the shocks to the expectation errors:
  \begin{enumerate}
    \item nonexistence of a stable solution
    \item existence of a unique stable solution (determinacy), i.e. $A_1=A_1(\theta)$ and $A_2=0$
    \item existence of multiple stable solutions (indeterminacy), i.e. $A_1$ is not uniquely determined by $\theta$ and $A_2\neq0$.
  \end{enumerate}
 \item Loosely speaking: there exists a unique stable solution, if the Taylor principle, that is the central bank raises the real interest rate in response to inflation more than one-to-one, and has multiple stable solutions otherwise (interesting in its own right)
\end{itemize}
\end{frame}
\end{document}
