%Erstellt mit WinEdt 6
\documentclass{beamer} %%% F\"{U}R VORTRAG MIT PAUSEN
%\documentclass[handout]{beamer}  %%% F\"{U}R HANDOUT ALS PDF

\setbeamertemplate{navigation symbols}{}
\usetheme{Madrid}
\usecolortheme{seagull}
\beamersetuncovermixins{\opaqueness<1>{25}}{\opaqueness<2->{15}}
\usepackage[T1]{fontenc}
\usepackage{ae,aecompl}
\usepackage[ansinew]{inputenc}
\usepackage[ngerman]{babel}
\usepackage{amsmath}
\usepackage{amsfonts}
\usepackage[babel,german=quotes]{csquotes} %im deutschen \"{u}bliche Anf\"{u}hrungszeichen
\usepackage{verbatim}
\usepackage{graphicx,pstricks,beamerprosper}
\newcounter{saveenumi}
\newcommand{\seti}{\setcounter{saveenumi}{\value{enumi}}}
\newcommand{\conti}{\setcounter{enumi}{\value{saveenumi}}}


\begin{document}
\author[Willi Mutschler]{Dr. Andrea Beccarini $\qquad$ Willi Mutschler, M.Sc.}
\date{Summer 2012}
\institute[Institute of Econometrics]{Institute of Econometrics and Economic
Statistics\\willi.mutschler@uni-muenster.de}
\title{DSGE-Models}
\subtitle{Calibration and Introduction to Dynare}

\begin{frame}
\titlepage
\end{frame}

\begin{frame}\frametitle{Calibration and Introduction to Dynare}
\tableofcontents
\end{frame}

\section{Overview Estimation-Methods}
\begin{frame}\frametitle{Overview Estimation-Methods}\framesubtitle{}
\begin{itemize}
    \item Econometrically, a DSGE-Model is a state-space model of which
        one has to determine the parameters.
      \item Three concepts:
      \begin{enumerate}
        \item \textbf{Calibration}: The parameters are set in such a
            way, that they closely correspond to some theoretical
            moment or stylized fact of data.
        \item \textbf{Methods of limited information} or weak
            econometric interpretation: Minimize the distance between
            theoretical and empirical moments, i.e.
            \emph{General-Method-of-Moments} or \emph{Indirect
            Inference}.
        \item \textbf{Methods of full information} or strong
            econometric interpretation: The goal is an exact
            characterization of observed data, i.e.
            \emph{Maximum-Likelihood} or \emph{bayesian methods}.
      \end{enumerate}
\end{itemize}
\end{frame}

\section{Calibration}
\begin{frame}\frametitle{Calibration}\framesubtitle{}
\begin{itemize}
   \item Goal: To answer a specific quantitative research question using
       a structural model.
   \item Construct and parameterize the model such, that it corresponds
       to certain properties of the true economy.
   \item Use steady-state-characteristics for choosing the parameters in
       accordance with observed data.
   \item Often: stable long-run averages (wages, working-hours, interest
       rates, inflation, consumption-shares, government-spending-ratios,
       etc.).
   \item You can use micro-studies as well, however, one has to be
       careful about the aggregation!
\end{itemize}
\end{frame}

\subsection{Hints for calibrating a model}
\begin{frame}\frametitle{Calibration}\framesubtitle{Hints for calibrating a model}

\begin{itemize}
   \item Use long-term averages of interest rates, inflation, average
       growth of productivity, etc. for \emph{steady-state} values.
   \item BUT: Weil (1989) shows, that in models with representative
       agents there is an overestimation of \emph{steady-state} interest
       rates (\emph{risk-free rate puzzle}). It is possible that you get
       absurd constellation of parameters, like a discount-factor of
       $\beta>1$.
   \item Usual mark-up on prices is around 1.15 (Corsetti et al (2012)).
   \item Intertemporal elasticity of substitution  $1/\sigma$ somewhere
       between $\sigma=1$ and $\sigma=3$ (King, Plosser and Rebelo
       (1988), Rotemberg and Woodford (1992), Lucas (2003)).
\end{itemize}
\end{frame}

\begin{frame}\frametitle{Calibration}\framesubtitle{Hints for calibrating a model}
\begin{itemize}
   \item Rigidity of prices: For an average price adjustment of 12-15
       months see Keen and Wang (2007).
   \item Coefficients of monetary policy: Often Taylor-Rule, you can use
       the relative coefficients to put more emphasize/weight on the
       stability of prices or on smoothing the business cycle.
   \item Parameters of stochastic processes: Often persistent, small
       standard-deviations, otherwise you get high oscillations. You can
       also estimate the production function via OLS (Solow-residual).
   \item How to choose shocks: Look at similar studies: Christiano,
       Eichenbaum and Evans (2005), Smets and Wouters (2003), etc..
   \item Ultimately: Try \& Error!
\end{itemize}
\end{frame}

\section{Exercise 2: Calibration of a RBC-model with monopolistic competition}
\begin{frame}\frametitle{Exercise 2:}\framesubtitle{Calibration of a RBC-model with monopolistic competition}
Households maximize expected utility over consumption $c_t$ and leisure
$f_t=1-l_t$, where $l_t$ denotes labor:
\begin{align*}
  E_t \sum_{t=0}^\infty \beta [log(c_t) + \psi log(1-l_t)],
\end{align*}
taking account of the following budget constraint:
\begin{align*}
  c_t + k_{t+1} = \underbrace{w_t l_t + r_t k_t}_{=y_t} + (1-\delta)k_t.
\end{align*}
$k_t$ ist the capital stock of the economy, $w_t$ the real wage, $r_t$ the
real interest rate and $\delta$ the rate of depreciation. Further, investment
is given by:
\begin{align*}i_t=y_t-c_t. \end{align*}
\end{frame}

\begin{frame}\frametitle{Exercise 2:}\framesubtitle{Calibration of a RBC-model with monopolistic competition}
In the market for intermediate goods there is monopolistic competition,
whereas perfect competition applies to the market for final goods. The
production-function of a firm $i\in[0;1]$, that sells intermediate goods, is
given by:
\begin{align*}
  y_{it}&=A_t k_{it}^\alpha l_{it}^{1-\alpha}, \qquad \qquad ~~0<\alpha<1,\\
  log(A_t) &= \rho log(A_{t-1}) + \epsilon_t, \qquad \epsilon_t \sim N(0,\sigma^2),
\end{align*}
with $A_t$ denoting the level of technology. Firms cannot influence the real
wage $w_t$ or the real interest rate $r_t$. However, they have market power
over their price $p_{it}$ for their good $y_{it}$. The intermediate goods are
combined into a final good by a Dixit/Stiglitz-type aggregator:
\begin{align*}
  y_t = \left(\int_0^1 (y_{it})^{\frac{\varepsilon}{\varepsilon-1}}\right)^{\frac{\varepsilon-1}{\varepsilon}},
\end{align*}
with $\varepsilon$ being the elasticity of substitution.
\end{frame}

\begin{frame}\frametitle{Exercise 2:}\framesubtitle{Calibration of a RBC-model with monopolistic competition}
\begin{enumerate}[(a)]
   \item Show that, the structural form of the DSGE-model is given by the
       following equations and interpret these.
       \begin{align}
         \frac{1}{c_t} &= \beta E_t \left[\frac{1}{c_{t+1}} (1+r_{t+1}-\delta)\right]\\
         w_t &= \psi \frac{c_t}{1-l_t}\\
         y_t &= c_t + i_t\\
         y_t &= A_t k_{t}^\alpha l_t^{1-\alpha}\\
         w_t & = (1-\alpha) \frac{y_t}{l_t} \frac{\varepsilon -1}{\varepsilon}\\
         r_t & = \alpha \frac{y_t}{k_t} \frac{\varepsilon -1}{\varepsilon}\\
         i_t &= k_{t+1} - (1-\delta) k_t\\
         log(A_t) &= \rho log(A_{t-1}) + \epsilon_t
       \end{align}\seti
\end{enumerate}

\end{frame}


\begin{frame}\frametitle{Exercise 2:}\framesubtitle{Calibration of a RBC-model with monopolistic competition}
\begin{enumerate}[(a)]\conti
   \item What are the parameters of the model?
   \item Write a mod-file for the model and calibrate the vector of
       parameters $\boldsymbol{\mu}$. Simulate the model for 1000 periods
       with Dynare. Save the middle 100 observations of $c_t, y_t, i_t,
       w_t$ and $r_t$ into an Excel-file as well as into a mat-file. Plot
       the path of consumption.
   \item Reformulate the structural equations such that variables are
       expressed as percentage deviations from \emph{steady-state}: $x_t=
       e^{log(x_t)-log(x)+log(x)}=x e^{\widehat{x}_t}$. Write a mod-file
       for this model. What has changed?
\end{enumerate}
\end{frame}

\section{Calibration - Pros \& Cons}
\begin{frame}\frametitle{Calibration}\framesubtitle{Pros}
\begin{itemize}
   \item Calibration is commonly used in the literature. It gives a first
       impression, a flavor of the strengths and weaknesses of a model.
   \item A good calibration can provide a valuable and precise image of
       data.
   \item Using different calibrations, one can asses interesting
       implications of different policies:
       \begin{itemize}
       \item How does the economy react, if the central bank focuses
           more on smoothing the business cycle than on price
           stability?
           \item What happens to consumption, if the households have
               a strong intertemporal elasticity of substitution?
               What if it is low?
       \end{itemize}
\end{itemize}
\end{frame}

\begin{frame}\frametitle{Calibration}\framesubtitle{Cons}
\begin{itemize}
   \item This Ad-hoc-approach is at the center of criticism of
       DSGE-models.
   \item There is no statistical foundation, it is based upon subjective
       views, assessments and opinions.
   \item Many parameter, such as those of the exogenous processes, leave
       room for different values and interpretations (intertemporal
       elasticity of substitution, monetary and fiscal parameters,
       coefficients of rigidity, standard deviations, etc.).
\end{itemize}
\begin{block}{Prescott (1986, S.~10) regarding RBC-models:} The models constructed within this theoretical framework are
necessarily \textbf{highly abstract}. Consequently, they are necessarily
false, and statistical hypothesis testing will reject them. This does not
imply, however, that nothing can be learned from such a \textbf{quantitative
theoretical exercise}.\end{block}
\end{frame}

\section{Exercise 3: A simple RBC model - Practicing Dynare}
\begin{frame}
  \frametitle{Exercise 3: A simple RBC model}
  Consider the following model of an economy.
\begin{itemize}

\item Representative agent preferences
\begin{equation*}
U=\sum_{t=1}^{\infty }\left( \frac{1}{1+\rho }\right) ^{t-1}E_{t}\left[ \log
\left( C_{t}\right) -\frac{L_{t}^{1+\gamma }}{1+\gamma }\right] .
\end{equation*}%
The household supplies labor and rents capital to the corporate sector.
\begin{itemize}
\item $L_{t}$ is labor services.
\item $\rho \in \left( 0,\infty \right) $
is the rate of time preference.
\item $\gamma \in \left( 0,\infty \right) $ is a
labor supply parameter.
\item $C_{t}$ is consumption.
\item $w_{t}$ is the real wage.
\item $r_{t}$ is the real rental rate.
\end{itemize}
\end{itemize}
\end{frame}



\begin{frame}
  \frametitle{Exercise 3: A simple RBC model}

\begin{itemize}
\item The household faces the sequence of budget constraints

\begin{equation*}
K_t=K_{t-1}\left( 1-\delta \right) +w_{t}L_{t}+r_{t}K_{t-1}-C_{t},
\end{equation*}%
where
\begin{itemize}
\item $K_{t}$ is capital at the end of period.
\item $\delta \in \left(
0,1\right) $ is the rate of depreciation.
\end{itemize}
\item The production function is given by the expression%
\begin{equation*}
Y_{t}=A_{t}K_{t-1}^{\alpha }\left( \left( 1+g\right) ^{t}L_{t}\right) ^{1-\alpha},
\end{equation*}%
where $g\in \left( 0,\infty \right) $ is the growth rate and $\alpha $ and $%
\beta $ are parameters.
\item $A_{t}$ is a technology shock that follows the
process%
\begin{equation*}
A_{t}=A_{t-1}^{\lambda }\exp \left( e_{t}\right) ,
\end{equation*}%
where $e_{t}$ is an i.i.d. zero mean normally distributed error with
standard deviation $\sigma$ and $\lambda \in \left( 0,1\right) $ is a
parameter.
\end{itemize}
\end{frame}

\begin{frame}  \frametitle{Exercise 3: A simple RBC model}
  \framesubtitle{The household problem}
  Lagrangian
\begin{multline*}
L = \max_{C_t,L_t,K_t} \sum_{t=1}^\infty \left( \frac{1}{1+\rho }\right) ^{t-1}E_{t}\Big[ \log
\left( C_{t}\right) -\frac{L_{t}^{1+\gamma }}{1+\gamma }\\
-\mu_t\left(K_t-K_{t-1}\left( 1-\delta \right) -w_{t}L_{t}-r_{t}K_{t-1}+C_{t}\right)\Big].
\end{multline*}
First order conditions
\begin{align*}
  \frac{\partial L}{\partial C_t} &= \left( \frac{1}{1+\rho }\right) ^{t-1}\left(\frac{1}{C_t}-\mu_t\right) = 0,\\
  \frac{\partial L}{\partial L_t} &= \left( \frac{1}{1+\rho }\right) ^{t-1}\left(L_t^\gamma-\mu_tw_t\right) = 0,\\
  \frac{\partial L}{\partial K_t} &= -\left( \frac{1}{1+\rho }\right) ^{t-1}\mu_t+\left( \frac{1}{1+\rho }\right) ^tE_t\left(\mu_{t+1}(1-\delta+r_{t+1})\right) = 0.
\end{align*}
\end{frame}

\begin{frame}  \frametitle{Exercise 3: A simple RBC model} \framesubtitle{First order conditions}
Eliminating the Lagrange multiplier, one obtains
\begin{align*}
L_t^\gamma&=\frac{w_t}{C_t},\\
\frac{1}{C_t}&=\frac{1}{1+\rho}E_t\left(\frac{1}{C_{t+1}}(r_{t+1}+1-\delta)\right).
\end{align*}

\end{frame}

\begin{frame}  \frametitle{Exercise 3: A simple RBC model}
  \framesubtitle{The firm problem}
  \[
  \max_{L_t,K_{t-1}} A_tK_{t-1}^\alpha\left( \left( 1+g\right) ^{t}L_t\right)^{1-\alpha}-r_tK_{t-1}-w_tL_t.
  \]
First order conditions:
\begin{align*}
  r_t &= \alpha A_tK_{t-1}^{\alpha-1}\left( \left( 1+g\right) ^{t}L_t\right)^{1-\alpha},\\
  w_t &= (1-\alpha) A_tK_{t-1}^\alpha\left( \left( 1+g\right) ^{t}\right)^{1-\alpha}L_t^{-\alpha}.
\end{align*}
\end{frame}

\begin{frame}\frametitle{Exercise 3: A simple RBC model}
  \framesubtitle{Goods market equilibrium}
  \[
  K_t+C_t = K_{t-1}(1-\delta)+\underbrace{A_tK_{t-1}^\alpha\left( \left( 1+g\right) ^{t}L_t\right)^{1-\alpha}}_{w_t L_t + r_t K_t}.
  \]
\end{frame}

\begin{frame}\frametitle{Exercise 3: A simple RBC model}
  \framesubtitle{Dynamic Equilibrium}
  \begin{align*}
    \frac{1}{C_t}&=\frac{1}{1+\rho}E_t\left(\frac{1}{C_{t+1}}(r_{t+1}+1-\delta)\right),\\
    L_t^\gamma&=\frac{w_t}{C_t},\\
    r_t &= \alpha A_tK_{t-1}^{\alpha-1}\left( \left( 1+g\right) ^{t}L_t\right)^{1-\alpha},\\
    w_t &= (1-\alpha) A_tK_{t-1}^\alpha\left( \left( 1+g\right) ^{t}\right)^{1-\alpha}L_t^{-\alpha},\\
    K_t+C_t &= K_{t-1}(1-\delta)+A_tK_{t-1}^\alpha\left( \left( 1+g\right) ^{t}L_t\right)^{1-\alpha},\\
        log(A_{t})&=\lambda log(A_{t-1}) + e_{t}.
  \end{align*}
\end{frame}

\begin{frame}\frametitle{Exercise 3: A simple RBC model}
  \framesubtitle{Existence of a balanced growth path}
Good markets equilibrium for each period t: $ K_t+C_t = K_{t-1}(1-\delta)+A_tK_{t-1}^\alpha\left( \left( 1+g\right) ^{t}L_t\right)^{1-\alpha}$.\\
So, there must exist growth rates $g_c$ and $g_k$ such that
\begin{multline*}
(1+g_k)^tK_1 + (1+g_c)^tC_1 =\\
 \frac{(1+g_k)^t}{1+g_k}K_1(1-\delta) + A_t\left(\frac{(1+g_k)^t}{1+g_k}K_1\right)^\alpha\left( \left( 1+g\right) ^{t}L_t\right)^{1-\alpha}
\end{multline*}
\begin{multline*}
\Leftrightarrow K_1 + \left(\frac{1+g_c}{1+g_k}\right)^tC_1 =\\
 \underbrace{\frac{K_1}{1+g_k}}_{K_0} (1-\delta) + A_t\left(\frac{K_1}{1+g_k}\right)^\alpha\left( \left( \frac{1+g}{1+g_k}\right) ^{t}L_t\right)^{1-\alpha}.
\end{multline*}

This is only valid, if $\qquad\qquad g_c=g_k=g$.

\end{frame}

\begin{frame}\frametitle{Exercise 3: A simple RBC model}
  \framesubtitle{Stationarized model}
  Let's define
  \begin{align*}
    \widehat{C}_t &= C_t/(1+g)^t,\\
    \widehat{K}_t &= K_t/(1+g)^t,\\
    \widehat{w}_t &= w_t/(1+g)^t.\\
  \end{align*}
\end{frame}

\begin{frame}\frametitle{Exercise 3: A simple RBC model}
  \framesubtitle{Stationarized model (continued)}
  \begin{align*}
    \frac{1}{\widehat{C}_t{\red(1+g)^t}}&=\frac{1}{1+\rho}E_t\left(\frac{1}{\widehat{C}_{t+1}(1+g){\red(1+g)^t}}(r_{t+1}+1-\delta)\right),\\
    L_t^\gamma&=\frac{\widehat{w}_t{\red(1+g)^t}}{\widehat{C}_t{\red(1+g)^t}},\\
    r_t &= \alpha A_t\left(\widehat{K}_{t-1}\frac{\red(1+g)^t}{1+g}\right)^{\alpha-1}\left( {\red\left( 1+g\right) ^{t}}L_t\right)^{1-\alpha},\\
    \widehat{w}_t{\red(1+g)^t} &= (1-\alpha) A_t\left(\widehat{K}_{t-1}\frac{\red(1+g)^t}{1+g}\right)^\alpha\left( {\red\left( 1+g\right) ^{t}}\right)^{1-\alpha}L_t^{-\alpha},\\
    \left(\widehat{K}_t+\widehat{C}_t\right){\red(1+g)^t} &= \widehat{K}_{t-1}\frac{\red(1+g)^t}{1+g}(1-\delta)\\
&+A_t\left(\widehat{K}_{t-1}\frac{\red(1+g)^t}{1+g}\right)^\alpha\left( {\red\left( 1+g\right) ^{t}}L_t\right)^{1-\alpha}.
  \end{align*}

\end{frame}

\begin{frame}\frametitle{Exercise 3: A simple RBC model}
  \framesubtitle{Stationarized model (continued)}
  \begin{align*}
    \frac{1}{\widehat{C}_t}&=\frac{1}{1+\rho}E_t\left(\frac{1}{\widehat{C}_{t+1}(1+g)}(r_{t+1}+1-\delta)\right),\\
    L_t^\gamma&=\frac{\widehat{w}_t}{\widehat{C}_t},\\
    r_t &= \alpha A_t\left(\frac{\widehat{K}_{t-1}}{1+g}\right)^{\alpha-1}L_t^{1-\alpha},\\
    \widehat{w}_t &= (1-\alpha) A_t\left(\frac{\widehat{K}_{t-1}}{1+g}\right)^\alpha L_t^{-\alpha},\\
    \widehat{K}_t+\widehat{C}_t &= \frac{\widehat{K}_{t-1}}{1+g}(1-\delta)+A_t\left(\frac{\widehat{K}_{t-1}}{1+g}\right)^\alpha L_t^{1-\alpha},\\
    log(A_{t})&=\lambda log(A_{t-1}) + e_{t}.
  \end{align*}
\end{frame}

\begin{frame}\frametitle{Exercise 3: A simple RBC model}\framesubtitle{Practicing Dynare}
\begin{enumerate}[(a)]
  \item Write a mod-File for this simple RBC-model and use for calibration:
$\alpha=0.33, \delta=0.1, \rho = 0.03,\lambda=0.97, \gamma=0, g = 0.015$. \\Use initval with these values:
$C=1,K=3,L=0.9,w=1,r=0.15,A=1$.
\item Show that the steady-state implies:
\begin{align*}
  A &=1, \qquad \qquad\qquad\qquad \qquad\qquad r =(1+g)(1+\delta) + \delta -1\\
  L &= \left(\frac{1-\alpha}{\frac{r}{\alpha}-\delta-g}\right) \left(\frac{r}{\alpha}\right),\qquad \qquad\qquad
  K = (1+g)\left(\frac{r}{\alpha}\right)^{\frac{1}{\alpha-1}} L\\
  C &= (1-\delta) \frac{K}{1+g} + \left(\frac{K}{1+g}\right)^\alpha L ^{1-\alpha} - K, \qquad \qquad\qquad
  w = C
\end{align*}
\item Use this analytical solution for the mod-file, i.e. use steady\_state\_model instead of initval. Dynare creates a steady-state m-file. Have a look at it.
\end{enumerate}

\end{frame}

\end{document}

