\documentclass{beamer} %%% F�R VORTRAG MIT PAUSEN
%\documentclass[handout]{beamer}  %%% F�R HANDOUT ALS PDF

\setbeamertemplate{navigation symbols}{}
\usetheme{Madrid}
\usecolortheme{seagull}
\beamersetuncovermixins{\opaqueness<1>{25}}{\opaqueness<2->{15}}
\usepackage[T1]{fontenc}
\usepackage{ae,aecompl}
\usepackage[ansinew]{inputenc}
\usepackage[ngerman]{babel}
\usepackage{amsmath}
\usepackage{amsfonts}
\usepackage[babel,german=quotes]{csquotes} %im deutschen �bliche Anf�hrungszeichen
\usepackage{verbatim}
\newcounter{saveenumi}
\newcommand{\seti}{\setcounter{saveenumi}{\value{enumi}}}
\newcommand{\conti}{\setcounter{enumi}{\value{saveenumi}}}


\begin{document}
\author[Willi Mutschler]{Dr. Andrea Beccarini $\qquad$ Willi Mutschler, M.Sc.}
\date{Summer 2012}
\institute[Institute of Econometrics]{Institute of Econometrics and Economic
Statistics\\University of M�nster\\willi.mutschler@uni-muenster.de}
\title{DSGE-Models}
\subtitle{Limited Information Estimation\\General Method of Moments and
Indirect Inference}

\begin{frame}
\titlepage
\end{frame}

\begin{frame}\frametitle{Limited Information Estimation}
\tableofcontents
\end{frame}

\section{Overview}
\begin{frame}\frametitle{Limited Information Estimation}\framesubtitle{}
\begin{itemize}
   \item Estimating the parameters of a DSGE-models one has to cope with
       several difficulties: Expectations about future variables,
       non-linearities, stochastic processes\dots
   \item For such cases there exist very general econometric methods like
       the \emph{General Method of Moments (GMM)} and the method of
       \emph{Indirect Inference}.
   \item \emph{Limited-information-estimators}, since there is no
       likelihood, but only specific moments of interest that are
       adjusted to data \emph{(Matching-Moments)}.
\end{itemize}
\end{frame}

\section{General Method of Moments}
\begin{frame}\frametitle{Limited Information Estimation}
\tableofcontents[currentsection]
\end{frame}
\begin{frame}\frametitle{General Method of Moments}\framesubtitle{}
\begin{itemize}
   \item Developed by Hansen (1982), first to use it for DSGE-models were
       Christiano and Eichenbaum (1992) and Burnside, Eichenbaum and
       Rebelo (1993).
   \item Idea: Represent DSGE-model as \emph{moment-} or
       \emph{orthogonality-conditions}:
\begin{align*}
E\left[\mathfrak{g}(\underset{k \times 1}{\boldsymbol{\mu}},\underset{a \times 1}{\boldsymbol{\Upsilon_t}}\right]
= E\begin{bmatrix} \mathfrak{f_1}(\underset{d\times1}{\mathbf{w_t}},\underset{k\times1}{\boldsymbol{\mu}})\underset{l\times1}{\mathbf{u_t}}\\
\vdots\\
\mathfrak{f_m}\left(\mathbf{w_t},\boldsymbol{\mu}\right)\mathbf{u_t}
 \end{bmatrix}=\mathbf{0},
\end{align*}
\item $\boldsymbol{\mu}$ is the true vector of parameters, $\mathbf{w_t}$
    a matrix of exogenous variables, $\mathbf{u_t}$ a matrix of
    instruments and
    $\boldsymbol{\Upsilon_t}=(\mathbf{w_t}',\mathbf{u_t}')'$.
\item Vector-valued functions: $\mathfrak{g}: r\times 1$ and
    $\mathfrak{f_i} :l\times 1$.
\item Number of orthogonality-conditions is equal to $r=ml$.
\end{itemize}
\end{frame}


\begin{frame}\frametitle{General Method of Moments}\framesubtitle{}
\begin{itemize}
   \item Orthogonality-conditions are derived from the
       first-order-conditions, the \emph{steady-state} relations and the
       properties of the stochastic processes.
   \item Find the estimator $\boldsymbol{\widehat{\mu}_G}$, that solves
       the empirical analogous of the orthogonality-conditions as
       \enquote{close} as possible, with the weight-matrix
       $\boldsymbol{\Omega}$ defining, what \enquote{close} means.
       \begin{block}{GMM-estimator}
\begin{align*}
  \boldsymbol{\widehat{\mu}_G} = \underset{\mu}{\text{min}} \left\{\left(\frac{1}{T} \sum_{t=0}^T \mathfrak{g}(\boldsymbol{\mu},\boldsymbol{\Upsilon})\right)'\times \boldsymbol{\Omega} \times \left(\frac{1}{T} \sum_{t=0}^T \mathfrak{g}(\boldsymbol{\mu},\boldsymbol{\Upsilon})\right) \right\}.
\end{align*}
\end{block}
\end{itemize}
\end{frame}

\begin{frame}\frametitle{General Method of Moments}\framesubtitle{}
\begin{itemize}
   \item If $r<k$, then the model is under-identified $\Rightarrow$ find
       additional instruments: lagged variables or use the fact that the
       shocks are not correlated.
   \item If $r=k$, then the model is exactly-identified: The
       weight-matrix does not play any role, since there is a unique
       solution to the quadratic form.
   \item If $r>k$, then the model is over-identified $\Rightarrow$
       several solutions that can be derived either analytically or
       numerically. The weight-matrix picks those moment-conditions that
       lead to a more precise estimation.
\begin{itemize}
   \item Hansen (1982) shows, that the optimal weight matrix is given
       by the inverse of the variance-covariance-matrix of the
       empirical analogous.
   \item Given some regularity conditions one can show that
       $\sqrt{T}(\widehat{\boldsymbol{\mu}}-\boldsymbol{\mu})$ is
       gaussian.
   \item In the over-identified case this enables one to formally
       test the hypothesis, that the model is able to describe the
       data generating process (\emph{J-Test}).
\end{itemize}
\end{itemize}
\end{frame}

\subsection{Example: Estimating the Euler-equation with GMM}
\begin{frame}\frametitle{General Method of Moments}\framesubtitle{Example: Estimating the Euler-equation with GMM}
Simple Euler-equation:
\begin{align*}
\beta E_t \left[c_{t+1}^{-\tau} (1+r_{t+1}-\delta)\right] &= c_t^{-\tau}\\
\Leftrightarrow E_t\left[ \beta \left(\frac{c_{t+1}}{c_{t}}\right)^{-\tau} (1+r_{t+1}-\delta)\right] &=1\\
\Leftrightarrow E_t\left[ \beta \left(\frac{c_{t+1}}{c_{t}}\right)^{-\tau} (1+r_{t+1}-\delta)-1\right] &= 0\\
 \Rightarrow E_t\left\{ \left[ \beta \left(\frac{c_{t+1}}{c_{t}}\right)^{-\tau} (1+r_{t+1}-\delta)-1\right]\begin{pmatrix}1 \\\frac{c_{t}}{c_{t-1}}\\r_t \end{pmatrix} \right\}&= \mathbf{0}\\
\end{align*}
with $\boldsymbol{\mu} = (\beta,\delta,\tau)'$ parameters to be estimated, exogenous variables (data) $\mathbf{w_t}=\left(\frac{c_{t+1}}{c_{t}}, r_{t+1}\right)'$ and instruments e.g. $\mathbf{u_t}=(1,\frac{c_{t}}{c_{t-1}},r_t)'$.
\end{frame}

\section{Indirect Inference}
\begin{frame}\frametitle{Limited Information Estimation}
\tableofcontents[currentsection]
\end{frame}
\begin{frame}\frametitle{Indirect Inference}\framesubtitle{}
\begin{itemize}
   \item Introduced to econometrics by Gourieroux, Monfort and Renault
       (1993) and Smith (1993) for nonlinear time-series models.
   \item Indirect: estimation is based on the simulation of data.
   \item Important requirement: One has to be able to simulate data with
       the model for different values of the parameters.
   \item These simulated datasets are estimated by an auxiliary model and
       compared to an estimation of the true data.
   \item Idea of \emph{Indirect Inference}: Estimate the true dataset
       with the auxiliary model. Simulate data using your DSGE-model for
       different values of the parameters. Estimate these simulated
       datasets with the auxiliary model and compare the estimators to
       the ones obtained with the true dataset until you find estimators
       that are \emph{almost} the same. Choose the parameters that
       generated that dataset.
\end{itemize}
\end{frame}

\begin{frame}\frametitle{Indirect Inference}\framesubtitle{}
\begin{itemize}
      \item In practice one often uses VAR-models.
       \item Great empirical forecast-performance
           (\enquote{\emph{work-horse}}).
   \item The solution of a DSGE-model in its state-space form corresponds
       closely to a VAR-model.
   \item Generally two possible methods of estimation:
       \begin{enumerate}
       \item Parameters of the VAR-model: Ruge-Marcia (2007).
       \item Impulse-Response-Matching: Christiano, Eichenbaum and
           Evans (2005).
       \end{enumerate}
   \item Very similar: Impulse-responses are functions of the parameters.
   \item The second method enables one to incorporate the dynamic
       properties of the VAR-model into the DSGE-model.
   \item However, identification issues: different combinations of
       parameters can generate the same impulse-responses.
\end{itemize}
\end{frame}

\subsection{Impulse-Response-Matching}
\begin{frame}\frametitle{Indirect Inference}\framesubtitle{Impulse-Response-Matching}
\begin{itemize}
   \item
\begin{block}{Indirect Inference Estimator}
\begin{align*}
  \boldsymbol{\widehat{\mu}_{I}} = \underset{\mu}{\text{min}} \left\{ (\boldsymbol{\Xi}-\boldsymbol{\Xi}(\boldsymbol{\mu}))'\times \boldsymbol{\Omega}\times (\boldsymbol{\Xi}-\boldsymbol{\Xi}(\boldsymbol{\mu}))\right\}.
\end{align*}
\end{block}
$\boldsymbol{\Xi}$: impulse-responses of the estimated VAR using the true
dataset, $\boldsymbol{\Xi}(\boldsymbol{\mu})$ the analogous with the
simulated datasets, $\boldsymbol{\Omega}$ a weight-matrix.
\item Smith (1993) shows, that using the inverse of the
    variance-covariance-matrix of $\boldsymbol{\Xi}$ for the weighting matrix,
    $\sqrt{T}(\boldsymbol{\widehat{\mu}}-\boldsymbol{\mu})$ is gaussian.
\item \emph{Method-of-Moments} interpretation, since the
    impulse-responses are functions of the covariances and
    autocovariances of the variables of the VAR-model.
\item \emph{Indirect Inference} interpretation, since the auxiliary model
    is a misspecified version of the true state-space representation.
\end{itemize}
\end{frame}

\subsection{Example: Estimating a simple DSGE-model using impulse-responses from a VAR(1)}
\begin{frame}\frametitle{Indirect Inference}\framesubtitle{Example: Estimating a simple DSGE-model using impulse-responses from a VAR(1)}
\end{frame}

\section{Discussion}
\begin{frame}\frametitle{Limited Information Estimation}
\tableofcontents[currentsection]
\end{frame}
\begin{frame}\frametitle{Limited Information Estimation}\framesubtitle{Discussion}
\begin{itemize}
   \item Methods are useful if the derivation of a specific criteria,
       like the likelihood, is analytically not possible or the
       evaluation too difficult.
   \item You only need a few assumptions about the first and second
       moments of the shocks (no distribution).
   \item Great advantage compared to calibration: Standard errors
       $\Rightarrow$ statistical inference is possible!
   \item Limiting to only relevant characteristics (distance function
       between theoretical and empirical moment) leads to robust
       estimators.
   \item J-Test of overidentification is a formal statistical test of the
       validity of your model.
   \item Rejection of the null, however, gives no hint on what is wrong
       with the model.
\end{itemize}
\end{frame}


\begin{frame}\frametitle{Limited Information Estimation}\framesubtitle{Discussion}
\begin{itemize}
\item GMM is robust towards misspecification, especially if you restrict
    to only a few conditions.
\item There is no need for an explicit solution or approximation of the
    DSGE-model.
\item GMM-estimators are reliable, however less efficient than the
    estimators you obtain using methods of full information.
\item Often one losses efficiency and there are identification issues, if
    one tries to exploit interdependencies between the blocks.
\item Choosing the right moment-conditions, instruments and algorithms
    for calculating the weight-matrix and numerical optimization are a
    very complex branch of research.
\end{itemize}
\end{frame}

\begin{frame}\frametitle{Limited Information Estimation}\framesubtitle{Discussion}
\begin{itemize}
\item \emph{Small-Sample-Bias}: Favorable properties of \emph{GMM} are
    only valid asymptotically.
\item Monte-Carlo-experiments show that for DSGE-models you need at least
    $T=300$ observations for the asymptotics to kick in.
\item For quarterly data that means about 75 years of data!
\item The relevant data for DSGE-models includes only the last 30-40
    years.
\item Further issue: How to find good and \textbf{time-homogenous} data
    for output-gap, technology, \dots?
\end{itemize}
\end{frame}

\begin{frame}\frametitle{Limited Information Estimation}\framesubtitle{Discussion}
\begin{itemize}
   \item Pros and cons of \emph{GMM} are also true for
       \emph{impulse-response-matching}.
   \item Main advantage of this form of \emph{Indirect Inference}:
       Limitation to only a few time-series.
   \item Further advantages: Auxiliary model needs not to be specified
       correctly.
   \item Opposed to GMM the DSGE-model needs to be solved explicitly,
       since otherwise one is not able to simulate data and
       impulse-responses from the model.
\end{itemize}
\end{frame}

\end{document}
