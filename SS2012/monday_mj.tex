\documentclass{beamer}
\usepackage{times,graphicx,pstricks,beamerprosper}
\usepackage[T1]{fontenc}
\title{DYNARE SUMMER SCHOOL}
\subtitle{Introduction to Dynare and local approximation.}
\date{June 18, 2012}
\author{Michel Juillard}
%\usetheme{Antibes}
%\useoutertheme[]{miniframes}%avoid horrors

%define custom colors
%\xdefinecolor{blob3}{rgb}{.3,.553,.753}
%\usecolortheme[named=blob3]{structure}

%\setbeamertemplate{navigation symbols}{}%remove navigation symbols


\begin{document}

\begin{frame}
  \titlepage
\end{frame}

\part<presentation>{Main talk}
\section[introduction]{Introduction}
  \begin{frame}
    \frametitle{Summer School website}
    \begin{center}
      http://www.dynare.org/summerschool/2012
    \end{center}
  \end{frame}



\begin{slide}{DYNARE}
  \begin{enumerate}
  \item computes the solution of deterministic models (arbitrary accuracy),
  \item computes first and second order approximation to solution of stochastic models,
  \item estimates (maximum likelihood or Bayesian approach) parameters of DSGE models,
  \item computes optimal policy,
  \item performs global sensitivity analysis of a model,
  \item solves problems under partial information,
  \item estimates BVAR and Markov-Switching Baysian VAR models.
  \end{enumerate}
\end{slide}

\section[Deterministic models]{Deterministic and stochastic models}
\begin{slide}{The general problem}
Deterministic, perfect foresight, case:
\[
f(y_{t+1},y_t,y_{t-1},u_t)=0
\]
Stochastic case:
\[
E_t\left\{f(y_{t+1},y_t,y_{t-1},u_t)\right\}=0
\]
\begin{description}
  \item[$y$]: vector of endogenous variables
  \item[$u$]: vector of exogenous shocks
\end{description}
\end{slide}


\begin{slide}{Solution methods}
\begin{itemize}
\item For a deterministic, perfect foresight, it is possible to compute
numerical trajectories for the endogenous variables
\item In a a stochastic framework, the unknowns are the decision functions:
\[
y_t = g(y_{t-1},u_t)
\]
\end{itemize}
For a large class of DSGE models, DYNARE computes approximated decision rules and transition equations by a perturbation method.
\end{slide}

\section[Introduction]{Introduction}
\begin{frame}
  \frametitle{Computation of first order approximation}

  \begin{itemize}
  \item Perturbation approach: recovering a Taylor expansion of the solution function from a Taylor expansion of the original model.
  \item A first order approximation is nothing else than a standard solution thru linearization.
  \item A first order approximation in terms of the logarithm of the variables provides standard log-linearization.
  \end{itemize}
\end{frame}

\section[A general model]{A general model}
\begin{frame}
\frametitle{General model}
\[
E_t\left\{f(y_{t+1},y_t,y_{t-1},u_t)\right\}=0
\]
\begin{eqnarray*}
E(u_t) &=& 0\\
E(u_tu_t') &=& \Sigma_u\\
E(u_tu_\tau') &=& 0\;\;\;\;t\ne\tau
\end{eqnarray*}

\begin{description}
  \item[$y$]: vector of endogenous variables
  \item[$u$]: vector of exogenous stochastic shocks
\end{description}
\end{frame}

\begin{frame} \frametitle{Timing assumptions}
  \[
  E_t\left\{f(y_{t+1},y_t,y_{t-1},u_t)\right\}=0
  \]
  \begin{itemize}
  \item shocks $u_t$ are observed at the beginning of period $t$,
  \item decisions affecting the current value of the variables $y_t$, are function of
    \begin{itemize}
    \item the previous state of the system, $y_{t-1}$,
    \item the shocks $u_t$.
    \end{itemize}
  \end{itemize}
\end{frame}

\begin{frame} \frametitle{The stochastic scale variable}
\[
E_t\left\{f(y_{t+1},y_t,y_{t-1},u_t)\right\}=0
\]
\begin{itemize}
\item At period $t$, the only unknown stochastic variable is $y_{t+1}$, and, implicitly, $u_{t+1}$.
\item We introduce the \emph{stochastic scale variable}, $\sigma$ and the auxiliary random variable, $\epsilon_t$, such that
\[
u_{t+1} = \sigma\epsilon_{t+1}
\]
\end{itemize}
\end{frame}

\begin{frame} \frametitle{The stochastic scale variable (continued)}
\begin{align}
E(\epsilon_t) &= 0\\
E(\epsilon_t\epsilon_t') &= \Sigma_\epsilon\\
E(\epsilon_t\epsilon_\tau') &= 0\;\;\;\;t\ne\tau  
\end{align}
and
\[
\Sigma_u = \sigma^2\Sigma_\epsilon
\]
\end{frame}


\begin{slide}{Remarks}
\[
E_t\left\{f(y_{t+1},y_t,y_{t-1},u_t)\right\}=0
\]
  \begin{itemize}
  \item The exogenous shocks may appear only at the current period
  \item There is no deterministic exogenous variables
  \item Not all variables are necessarily present with a lead and a lag
  \item Generalization to leads and lags on more than one period (2nd order approximation requires a more complicated algorithm)
  \end{itemize}
\end{slide}

\section[The solution function]{The solution function}
\begin{slide}{Solution function}
\[
y_t = g(y_{t-1},u_t,\sigma)
\]
where $\sigma$ is the stochastic scale of the model. If $\sigma = 0$, the model is deterministic. For $\sigma > 0$, the model is stochastic.

Under some conditions, the existence of $g()$ function is proven via an implicit function theorem. See H. Jin and K. Judd ``Solving Dynamic Stochastic Models'' {\tiny (\url{http://bucky.stanford.edu/papers/PerturbationMethodRatEx.pdf})}
\end{slide}

\begin{slide}{Solution function (continued)}
Then,
{
\begin{eqnarray*}
  y_{t+1} &=& g(y_t,u_{t+1},\sigma)\\
  &=& g(g(y_{t-1},u_t,\sigma),u_{t+1},\sigma)\\
\lefteqn{F(y_{t-1},u_t,\epsilon_{t+1},\sigma)}\\
 &=& f(g(g(y_{t-1},u_t,\sigma),\sigma\epsilon_{t+1},\sigma),g(y_{t-1},u_t,\sigma),y_{t-1},u_t)
\end{eqnarray*}
}
\[
  E_t\left\{F(y_{t-1},u_t,{\red \epsilon_{t+1}},\sigma)\right\} = 0
\]
  
\end{slide}

\begin{frame} \frametitle{The perturbation approach}
  \begin{itemize}
  \item Obtain a Taylor expansion of the unkown solution function in the neighborhood of a problem that we know how to solve.
  \item The problem that we know how to solve is the deterministic steady state.
  \item One obtains the Taylor expansion of the solution for the Taylor expansion of the original problem.
  \item One consider two different perturbations:
    \begin{enumerate}
    \item points in the neighborhood from the steady sate,
    \item from a deterministic model towards a stochastic one (by increasing $\sigma$ from a zero value).
    \end{enumerate}
  \end{itemize}
\end{frame}

\begin{frame} \frametitle{The perturbation approach (continued)}
\begin{itemize}
  \item The Taylor approximation is taken with respect to $y_{t-1}$, $u_t$ and $\sigma$, the arguments of the solution function
    \[
    y_t = g(y_{t-1},u_t,\sigma).
    \]
  \item At the deterministic steady state, all derivatives are deterministic as well.
  \end{itemize}
\end{frame}

\section[The steady state]{The steady state}
\begin{slide}{Steady state}
A deterministic steady state, $\bar y$, for the model satisfies
\[
f(\bar y, \bar y, \bar y, 0) = 0
\]
A model can have several steady states, but only one of them will be used for approximation.

Furthermore,
\[
\bar y = g(\bar y, 0, 0)
\]
\end{slide}


\section[First order approximation]{First order approximation}
\begin{slide}{First order approximation}
Around $\bar y$:
\begin{eqnarray*}
   \lefteqn{E_t\left\{F^{(1)}(y_{t-1},u_t,{\red \epsilon_{t+1}},\sigma)\right\} =}\\
&& E_t\Big\{f(\bar y, \bar y, \bar y, 0)+f_{y_+}\left(g_y\left(g_y\hat y+g_uu+g_\sigma\sigma\right)+g_u\sigma{\red \epsilon'}+g_\sigma\sigma\right)\\
&& + f_{y_0}\left(g_y\hat y+g_uu+g_\sigma\sigma\right)+f_{y_-}\hat y+f_u u\Big\}\\
&=& 0
\end{eqnarray*}
with $\hat y = y_{t-1} - \bar y$, $u=u_t$, $\epsilon'=\epsilon_{t+1}$, $f_{y_+}=\frac{\partial f}{\partial y_{t+1}}$, $f_{y_0}=\frac{\partial f}{\partial y_t}$, $f_{y_-}=\frac{\partial f}{\partial y_{t-1}}$, $f_{u}=\frac{\partial f}{\partial u_t}$, $g_y=\frac{\partial g}{\partial y_{t-1}}$, $g_u=\frac{\partial g}{\partial u_t}$, $g_\sigma=\frac{\partial g}{\partial \sigma}$.
\end{slide}

\begin{slide}{Taking the expectation}
\begin{eqnarray*}
   \lefteqn{E_t\left\{F^{(1)}(y_{t-1},u_t,{\red \epsilon_{t+1}},\sigma)\right\} =}\\
&& f(\bar y, \bar y, \bar y, 0)+f_{y_+}\left(g_y\left(g_y\hat y+g_uu+g_\sigma\sigma\right)+g_\sigma\sigma\right)\\
&& + f_{y_0}\left(g_y\hat y+g_uu+g_\sigma\sigma\right)+f_{y_-}\hat y+f_u u\Big\}\\
&=& \left(f_{y_+}g_yg_y+f_{y_0}g_y+f_{y_-}\right)\hat y+\left(f_{y_+}g_yg_u+f_{y_0}g_u+f_{u}\right)u\\
&& +\left(f_{y_+}g_yg_\sigma+f_{y_0}g_\sigma\right)\sigma\\
&=& 0\\
\end{eqnarray*}
  
\end{slide}

\subsection[Recovering $g_y$]{Recovering $g_y$}
\begin{slide}{Recovering $g_y$}
\[
\left(f_{y_+}{\red g_yg_y}+f_{y_0}{\red g_y}+f_{y_-}\right)\hat y= 0
\]
Structural state space representation:
\begin{eqnarray*}
\left[\begin{array}{cc}
    0 & f_{y_+} \\ I & 0 
  \end{array}\right]
\left[\begin{array}{c}
    I \\ {\red g_y}
  \end{array}\right]{\red g_y} \hat y
&=&
\left[\begin{array}{cc}
    -f_{y_-} & -f_{y_0} \\ 0 & I 
  \end{array}\right]
\left[\begin{array}{c}
    I \\ {\red g_y}
  \end{array}\right]\hat y
\end{eqnarray*}
or
\begin{eqnarray*}
\left[\begin{array}{cc}
    0 & f_{y_+} \\ I & 0 
  \end{array}\right]
\left[\begin{array}{c}
    y_t-\bar y \\ y_{t+1}-\bar y
  \end{array}\right]
&=&
\left[\begin{array}{cc}
    -f_{y_-} & -f_{y_0} \\ 0 & I 
  \end{array}\right]
\left[\begin{array}{c}
    y_{t-1}-\bar y \\ y_t-\bar y
  \end{array}\right]
\end{eqnarray*}
\end{slide}

\begin{slide}{Structural state space representation}
\[
Dx_{t+1} = Ex_t
\]
with
\[
x_{t+1} = \left[\begin{array}{c}
    y_t-\bar y \\ y_{t+1}-\bar y
  \end{array}\right]
\hspace{10mm}
x_t = \left[\begin{array}{c}
    y_{t-1}-\bar y \\ y_t-\bar y
  \end{array}\right]
\]
\begin{itemize}
\item There are multiple solutions but we want a unique stable one.\\
\item Problem when $D$ is singular.
\end{itemize}
\end{slide}

\subsubsection[Real generalized Schur decomposition]{Real generalized Schur decomposition}
\begin{slide}{Real generalized Schur decomposition}
Taking the real generalized Schur decomposition of the pencil $<E,D>$:
\begin{eqnarray*}
D&=&QTZ\\
E&=&QSZ
\end{eqnarray*}
with $T$, upper triangular, $S$ quasi-upper triangular, $Q'Q=I$ and $Z'Z=I$.
\end{slide}

\subsubsection[Generalized eigenvalues]{Generalized eigenvalues}
\begin{slide}{Generalized eigenvalues}

$\lambda_i$ solves
\[
\lambda_iDx_i = E x_i
\]

For diagonal blocks on $S$ of dimension 1 x 1:
\begin{itemize}
\item $T_{ii} \ne 0$: $\lambda_i = \frac{S_{ii}}{T_{ii}}$
\item $T_{ii} = 0$, $S_{ii} > 0$: $\lambda_i = +\infty$
\item $T_{ii} = 0$, $S_{ii} < 0$: $\lambda_i = -\infty$
\item $T_{ii} = 0$, $S_{ii} = 0$: $\lambda_i \in \mathcal{C}$
\end{itemize}

\end{slide}

\begin{frame}
\frametitle{A pair of complex eigenvalues}
When a diagonal block of matrix $S$ is a 2x2 matrix of the form 
$\left[
\begin{array}{cc}
S_{ii} & S_{i,i+1}\\
S_{i+1,i} & S_{i+1,i+1}
\end{array}\right]$,
\begin{itemize}
\item the corresponding block of matrix $T$ is a diagonal matrix,
\item $\left(S_{i,i}T_{i+1,i+1}+S_{i+1,i+1}T_{i,i}\right)^2<-4S_{i+1,i}S_{i+1,i}T_{i,i}T_{i+1,i+1}$,
\item there is a pair of conjugate eigenvalues
\end{itemize}
{\footnotesize
\begin{multline*}
\lambda_i, \lambda_{i+1} =\\
\frac{S_{ii}T_{i+1,i+1}+S_{i+1,i+1}T_{i,i}\pm\sqrt{\left(S_{i,i}T_{i+1,i+1}-S_{i+1,i+1}T_{i,i}\right)^2+4S_{i+1,i}S_{i+1,i}T_{i,i}T_{i+1,i+1}}}{2T_{i,i}T_{i+1,i+1}}  
\end{multline*}
}
\end{frame}

\begin{slide}{Applying the decomposition}
\begin{eqnarray*}
D\left[\begin{array}{c} I\\{\red g_y}\end{array}\right]{\red g_y}\hat y
&=&
E\left[\begin{array}{c} I\\{\red g_y}\end{array}\right]\hat y\\
\lefteqn{\left[\begin{array}{cc}T_{11} & T_{12}\\0 & T_{22}\end{array}\right]
\left[\begin{array}{cc}Z_{11} & Z_{12}\\Z_{21} & Z_{22}\end{array}\right]
\left[\begin{array}{c} I\\{\red g_y}\end{array}\right]{\red g_y}\hat y}\hspace{40mm}\\
&=&
\left[\begin{array}{cc}S_{11} & S_{12}\\0 & S_{22}\end{array}\right]
\left[\begin{array}{cc}Z_{11} & Z_{12}\\Z_{21} & Z_{22}\end{array}\right]
\left[\begin{array}{c} I\\{\red g_y}\end{array}\right]\hat y
\end{eqnarray*}
\end{slide}

\begin{slide}{Selecting the stable trajectory}
To exclude explosive trajectories, one imposes
\[
Z_{21}+Z_{22}{\red g_y}=0
\]
\[
{\red g_y}=-Z_{22}^{-1}Z_{21}
\]
A unique stable trajectory exists if $Z_{22}$ is non-singular: there are as many roots larger than one in modulus as there are forward--looking variables in the model (Blanchard and Kahn condition) and the rank condition is satisfied.
\end{slide}

\subsection[Recovering $g_u$]{Recovering $g_u$}
\begin{slide}{Recovering $g_u$}
\[
f_{y_+}g_y{\red g_u}+f_{y_0}{\red g_u}+f_{u}=0
\]
\[
{\red g_u} = -\left(f_{y_+}g_y+f_{y_0}\right)^{-1}f_u
\]

{\footnotesize Hong Lan \& Alexander Meyer-Gohde,
2012. "\href{http://sfb649.wiwi.hu-berlin.de/papers/pdf/SFB649DP2012-015.pdf}{\blue
  Existence and Uniqueness of
Perturbation Solutions to DSGE Models}," SFB 649 Discussion Papers,
Humboldt University, show that $f_{y_+}g_y+f_{y_0}$ is an invertible
matrix under standard regularity and saddle stability assumptions.}
\end{slide}

\subsection[Recovering $g_\sigma$]{Recovering $g_\sigma$}
\begin{slide}{Recovering $g_\sigma$}
\[
f_{y_+}g_y{\red g_\sigma}+f_{y_0}{\red g_\sigma} = 0
\]
\[
{\red g_\sigma} = 0
\]
Yet another manifestation of the certainty equivalence property of first order approximation.
\end{slide}

\subsection[First order approximated decision function]{First order approximated decision function}
\begin{slide}{First order approximated decision function}
\[
y_t = \bar y+g_y\hat y+g_uu
\]
\begin{eqnarray*}
  E\left\{y_t\right\} &=& \bar y\\
  \Sigma_y &=& g_y \Sigma_y g_y'+\sigma^2g_u\Sigma_\epsilon g_u'
\end{eqnarray*}
The variance is solved for with an algorithm for Lyapunov equations.
\end{slide}

\section[Example]{Example}
\begin{frame}
  \frametitle{A simple RBC model}
  Consider the following model of an economy.
\begin{itemize}

\item Representative agent preferences
\begin{equation*}
U=\sum_{t=1}^{\infty }\left( \frac{1}{1+\rho }\right) ^{t-1}E_{t}\left[ \log
\left( C_{t}\right) -\frac{L_{t}^{1+\gamma }}{1+\gamma }\right] .
\end{equation*}%
The household supplies labor and rents capital to the corporate sector. 
\begin{itemize}
\item $L_{t}$ is labor services
\item $\rho \in \left( 0,\infty \right) $
is the rate of time preference
\item $\gamma \in \left( 0,\infty \right) $ is a
labor supply parameter. 
\item $C_{t}$ is consumption,
\item $w_{t}$ is the real wage,
\item $r_{t}$ is the real rental rate
\end{itemize}
\end{itemize}
\end{frame}


\begin{frame}
\frametitle{RBC Model (continued)}
\begin{itemize}
\item The household faces the sequence of budget constraints

\begin{equation*}
K_t=K_{t-1}\left( 1-\delta \right) +w_{t}L_{t}+r_{t}K_{t-1}-C_{t},
\end{equation*}%
where 
\begin{itemize}
\item $K_{t}$ is capital at the end of period
\item $\delta \in \left(
0,1\right) $ is the rate of depreciation
\end{itemize}
\item The production function is given by the expression%
\begin{equation*}
Y_{t}=A_{t}K_{t-1}^{\alpha }\left( \left( 1+g\right) ^{t}L_{t}\right) ^{1-\alpha}
\end{equation*}%
where $g\in \left( 0,\infty \right) $ is the growth rate and $\alpha $ and $%
\beta $ are parameters. 
\item $A_{t}$ is a technology shock that follows the
process%
\begin{equation*}
A_{t}=A_{t-1}^{\lambda }\exp \left( e_{t}\right) ,
\end{equation*}%
where $e_{t}$ is an i.i.d. zero mean normally distributed error with
standard deviation $\sigma _{1}$ and $\lambda \in \left( 0,1\right) $ is a
parameter.
\end{itemize}
\end{frame}

\begin{frame}
  \frametitle{The household problem}
  Lagrangian
\begin{multline*}
L = \max_{C_t,L_t,K_t} \sum_{t=1}^\infty \left( \frac{1}{1+\rho }\right) ^{t-1}E_{t}\Big[ \log
\left( C_{t}\right) -\frac{L_{t}^{1+\gamma }}{1+\gamma }\\
-\mu_t\left(K_t-K_{t-1}\left( 1-\delta \right) -w_{t}L_{t}-r_{t}K_{t-1}+C_{t}\right)\Big]
\end{multline*}
First order conditions
\begin{align*}
  \frac{\partial L}{\partial C_t} &= \left( \frac{1}{1+\rho }\right) ^{t-1}\left(\frac{1}{C_t}-\mu_t\right) = 0\\
  \frac{\partial L}{\partial L_t} &= \left( \frac{1}{1+\rho }\right) ^{t-1}\left(L_t^\gamma-\mu_tw_t\right) = 0\\
  \frac{\partial L}{\partial K_t} &= -\left( \frac{1}{1+\rho }\right) ^{t-1}\mu_t+\left( \frac{1}{1+\rho }\right) ^tE_t\left(\mu_{t+1}(1-\delta+r_t)\right) = 0
\end{align*}
\end{frame}

\begin{frame}{First order conditions}
Eliminating the Lagrange multiplier, one obtains
\begin{align*}
L_t^\gamma&=\frac{w_t}{C_t}\\
\frac{1}{C_t}&=\frac{1}{1+\rho}E_t\left(\frac{1}{C_{t+1}}(r_{t+1}+1-\delta)\right)
\end{align*}

\end{frame}

\begin{frame}
  \frametitle{The firm problem}
  \[
  \max_{L_t,K_{t-1}} A_tK_{t-1}^\alpha\left( \left( 1+g\right) ^{t}L_t\right)^{1-\alpha}-r_tK_{t-1}-w_tL_t
  \]
First order conditions:
\begin{align*}
  r_t &= \alpha A_tK_{t-1}^{\alpha-1}\left( \left( 1+g\right) ^{t}L_t\right)^{1-\alpha}\\
  w_t &= (1-\alpha) A_tK_{t-1}^\alpha\left( \left( 1+g\right) ^{t}\right)^{1-\alpha}L_t^{-\alpha}
\end{align*}
\end{frame}

\begin{frame}
  \frametitle{Goods market equilibrium}
  \[
  K_t+C_t = K_{t-1}(1-\delta)+A_tK_{t-1}^\alpha\left( \left( 1+g\right) ^{t}L_t\right)^{1-\alpha}
  \]
\end{frame}

\begin{frame}
  \frametitle{Dynamic Equilibrium}
  \begin{align*}
    \frac{1}{C_t}&=\frac{1}{1+\rho}E_t\left(\frac{1}{C_{t+1}}(r_{t+1}+1-\delta)\right)\\
    L_t^\gamma&=\frac{w_t}{C_t}\\
    r_t &= \alpha A_tK_{t-1}^{\alpha-1}\left( \left( 1+g\right) ^{t}L_t\right)^{1-\alpha}\\
    w_t &= (1-\alpha) A_tK_{t-1}^\alpha\left( \left( 1+g\right) ^{t}\right)^{1-\alpha}L_t^{-\alpha}\\
    K_t+C_t &= K_{t-1}(1-\delta)+A_tK_{t-1}^\alpha\left( \left( 1+g\right) ^{t}L_t\right)^{1-\alpha}
  \end{align*}
\end{frame}

\begin{frame}
  \frametitle{Existence of a balanced growth path}
  There must exist a growth rates $g_c$ and $g_k$ so that
\begin{multline*}
(1+g_k)^tK_1 + (1+g_c)^tC_1 =\\
 \frac{(1+g_k)^t}{1+g_K}K_1(1-\delta) + A\left(\frac{(1+g_k)^t}{1+g_k}K_1\right)^\alpha\left( \left( 1+g\right) ^{t}L_t\right)^{1-\alpha}
\end{multline*}
So,
\[
g_c=g_c=g
\]
\end{frame}

\begin{frame}
  \frametitle{Stationarized model}
  Let's define
  \begin{align*}
    \widehat{C}_t &= C_t/(1+g)^t\\
    \widehat{K}_t &= K_t/(1+g)^t\\
    \widehat{w}_t &= w_t/(1+g)^t\\
  \end{align*}
\end{frame}

\begin{frame}
  \frametitle{Stationarized model (continued)}
  \begin{align*}
    \frac{1}{\widehat{C}_t{\red(1+g)^t}}&=\frac{1}{1+\rho}E_t\left(\frac{1}{\widehat{C}_{t+1}(1+g){\red(1+g)^t}}(r_{t+1}+1-\delta)\right)\\
    L_t^\gamma&=\frac{\widehat{w}_t{\red(1+g)^t}}{\widehat{C}_t{\red(1+g)^t}}\\
    r_t &= \alpha A_t\left(\widehat{K}_{t-1}\frac{\red(1+g)^t}{1+g}\right)^{\alpha-1}\left( {\red\left( 1+g\right) ^{t}}L_t\right)^{1-\alpha}\\
    \widehat{w}_t{\red(1+g)^t} &= (1-\alpha) A_t\left(\widehat{K}_{t-1}\frac{\red(1+g)^t}{1+g}\right)^\alpha\left( {\red\left( 1+g\right) ^{t}}\right)^{1-\alpha}L_t^{-\alpha}\\
    \left(\widehat{K}_t+\widehat{C}_t\right){\red(1+g)^t} &= \widehat{K}_{t-1}\frac{\red(1+g)^t}{1+g}(1-\delta)\\
&+A_t\left(\widehat{K}_{t-1}\frac{\red(1+g)^t}{1+g}\right)^\alpha\left( {\red\left( 1+g\right) ^{t}}L_t\right)^{1-\alpha}
  \end{align*}
    
\end{frame}

\begin{frame}
  \frametitle{Stationarized model (continued)}
  \begin{align*}
    \frac{1}{\widehat{C}_t}&=\frac{1}{1+\rho}E_t\left(\frac{1}{\widehat{C}_{t+1}(1+g)}(r_{t+1}+1-\delta)\right)\\
    L_t^\gamma&=\frac{\widehat{w}_t}{\widehat{C}_t}\\
    r_t &= \alpha A_t\left(\frac{\widehat{K}_{t-1}}{1+g}\right)^{\alpha-1}L_t^{1-\alpha}\\
    \widehat{w}_t &= (1-\alpha) A_t\left(\frac{\widehat{K}_{t-1}}{1+g}\right)^\alpha L_t^{-\alpha}\\
    \widehat{K}_t+\widehat{C}_t &= \frac{\widehat{K}_{t-1}}{1+g}(1-\delta)+A_t\left(\frac{\widehat{K}_{t-1}}{1+g}\right)^\alpha L_t^{1-\alpha}
  \end{align*}    
\end{frame}

\begin{frame}[fragile]
  \frametitle{Dynare implementation}
\begin{verbatim}
 var C K L w r A;
 varexo e;

 parameters rho delta gamma alpha lambda g;

 alpha = 0.33;
 delta = 0.1;
 rho = 0.03;
 lambda = 0.97;
 gamma = 0;
 g = 0.015;
\end{verbatim}
\end{frame}

\begin{frame}[fragile]
\frametitle{Dynare implementation (continued)}
\begin{verbatim}
 model;
 1/C=1/(1+rho)*(1/(C(+1)*(1+g)))*(r(+1)+1-delta);
 L^gamma = w/C;
 r = alpha*A*(K(-1)/(1+g))^(alpha-1)*L^(1-alpha);
 w = (1-alpha)*A*(K(-1)/(1+g))^alpha*L^(-alpha);
 K+C = (K(-1)/(1+g))*(1-delta)
      +A*(K(-1)/(1+g))^alpha*L^(1-alpha);
 log(A) = lambda*log(A(-1))+e;
 end;
\end{verbatim}
\end{frame}

\begin{frame}[fragile]
\frametitle{Dynare implementation (continued)}
\begin{verbatim}
 steady_state_model;
 A = 1;
 r = (1+g)*(1+rho)+delta-1;
 L = ((1-alpha)/(r/alpha-delta-g))*r/alpha;
 K = (1+g)*(r/alpha)^(1/(alpha-1))*L;
 C = (1-delta)*K/(1+g)
      +(K/(1+g))^alpha*L^(1-alpha)-K;
 w = C;
 end;

 steady;
\end{verbatim}
\end{frame}

\begin{frame}[fragile]
\frametitle{Dynare implementation (continued)}
\begin{verbatim}
 shocks;
 var e; stderr 0.01;
 end;

 check;

 stoch_simul(order=1);
\end{verbatim}
\end{frame}

\begin{frame}[fragile]
  \frametitle{Decision and transition functions}
Dynare output:
{\tiny
\begin{verbatim}
POLICY AND TRANSITION FUNCTIONS
               C           K           L          w           r           A
Constant   1.003043    3.125296    0.906526    1.003043    0.145450    1.000000
K(-1)      0.144433    0.779746   -0.105500    0.144433   -0.042523           0
A(-1)      0.757723    1.149948    0.589451    0.757723    0.204452    0.970000
e          0.781158    1.185514    0.607681    0.781158    0.210776    1.000000
\end{verbatim}
}
\[
C_t = 1.003 + 0.144\left(K_{t-1}-\bar K\right)+0.758
\left(A_{t-1}-\bar A\right) + 0.781 e_t
\]
\end{frame}

\section[Dating variables in Dynare]{Dating variables in Dynare}
\begin{frame}
  \frametitle{Dating variables in Dynare}
Dynare will automatically recognize predetermined and non--predetermined variables, but you must observe a few rules:
\begin{itemize}
\item period $t$ variables are set during period $t$ on the basis of the state of the system at period $t-1$ and shocks observed at the beginning of period $t$.
\item therefore, stock variables must be on an end--of--period basis: investment of period $t$ determines the capital stock at the end of period $t$.
\end{itemize}
\end{frame}

\begin{frame}
  \frametitle{Log--linearization}
\begin{itemize}
\item Taking a log--linear approximation of a model is equivalent to take a linear approximation of a model with respect to the logarithm of the variables.
\item In practice, it is sufficient to replace all occurences of variable $X$ with $exp(LX)$ where $LX=\log X$.
\item It is possible to make the substitution for some variables and not anothers. You wouldn't want to take a log approximation of a variable whose steady state value is negative \ldots
\item There is no evidence that log--linearization is more accurate than simple linearization. In a growth model, it is often more natural to do a log--linearization.
\end{itemize}
\end{frame}

\section[The Dynare preprocessor]{The Dynare preprocessor}
\begin{frame}
  \frametitle{The role of the Dynare preprocessor}

  \begin{itemize}
  \item the Dynare toolbox solves generic problems
  \item the parser reads your *.mod file and translates it in specific Matlab files
  \item \emph{filename}{\rm.m}: main Matlab script for your model
  \item \emph{filename}{\rm\_static.m}: static model
  \item \emph{filename}{\rm\_dynamic.m}: dynamic model
  \item \emph{filename}{\rm\_steadystate2.m}: steady state function
  \item \emph{filename}{\rm\_set\_auxiliary\_variables.m}: auxiliary
    variables function
  \end{itemize}
\end{frame}

\section[Approximated model]{Approximated model}
\begin{slide}{Second order approximation of the model}
{\tiny
\begin{eqnarray*}
   \lefteqn{E_t\left\{F^{(2)}(y_{t-1},u_t,{\red \epsilon_{t+1}},\sigma)\right\} =}\\
&& E_t\Big\{F^{(1)}(y_{t-1},u_t,{\red u_{t+1}},\sigma)\\
&&+0.5\Big(F_{y_-y_-}(\hat y\otimes \hat y)+F_{uu}(u\otimes u)+F_{u'u'}{\red \sigma^2}({\red \epsilon'}\otimes {\red \epsilon'})+F_{\sigma\sigma}\sigma^2\Big)\\
&&+F_{y_-u}(\hat y \otimes u)+F_{y_-u'}(\hat y\otimes {\red \sigma\epsilon'})+F_{y_-\sigma}\hat y\sigma+F_{uu'}(u\otimes {\red \sigma\epsilon'})+F_{u\sigma}u\sigma+F_{u'\sigma}{\red \sigma\epsilon'}\sigma\Big\}\\
&=& E_t\Big\{F^{(1)}(y_{t-1},u_t,{\red \epsilon_{t+1}},\sigma)\Big\}\\
&&+0.5\Big(F_{y_-y_-}(\hat y\otimes \hat y)+F_{uu}(u\otimes u)+F_{u'u'}({\red \sigma^2\vec\Sigma_\epsilon})+F_{\sigma\sigma}\sigma^2\Big)\\
&&+F_{y_-u}(\hat y \otimes u)+F_{y_-\sigma}\hat y\sigma+F_{u\sigma}u\sigma\\
&=& 0
\end{eqnarray*}
}
\end{slide}

\subsection[]{}
\begin{slide}{Representing the second order derivatives}
The second order derivatives of a vector of multivariate functions is a three dimensional object. We use the following notation
\[
\frac{\partial^2 F}{\partial x \partial x} = \left[\begin{array}{cccccc}
\frac{\partial^2 F_1}{\partial x_1\partial x_1} & \frac{\partial^2 F_1}{\partial x_1\partial x_2} & \ldots & \frac{\partial^2 F_1}{\partial x_2\partial x_1} & \ldots & \frac{\partial^2 F_1}{\partial x_n\partial x_n} \\
\rule{0pt}{15pt}\frac{\partial^2 F_2}{\partial x_1\partial x_1} & \frac{\partial^2 F_2}{\partial x_1\partial x_2} & \ldots & \frac{\partial^2 F_2}{\partial x_2\partial x_1} & \ldots & \frac{\partial^2 F_2}{\partial x_n\partial x_n} \\
\vdots & \vdots \ddots & \vdots & \ddots & \vdots\\
\frac{\partial^2 F_m}{\partial x_1\partial x_1} & \frac{\partial^2 F_m}{\partial x_1\partial x_2} & \ldots & \frac{\partial^2 F_m}{\partial x_2\partial x_1} & \ldots & \frac{\partial^2 F_m}{\partial x_n\partial x_n}
\end{array}\right]
\]
\end{slide}

\subsection[]{}
\begin{slide}{Composition of two functions}
Let
  \begin{eqnarray*}
    y &=& g(s)\\
    f(y)&=& f(g(s))
  \end{eqnarray*}
then,
\[
\frac{\partial^2 f}{\partial s\partial s}=\frac{\partial f}{\partial y}\frac{\partial^2 g}{\partial s\partial s}+\frac{\partial^2 f}{\partial y\partial y}\left(\frac{\partial g}{\partial s}\otimes\frac{\partial g}{\partial s}\right)
\]
\end{slide}


\begin{slide}{Recovering $g_{yy}$}
  \begin{eqnarray*}
    F_{y_-y_-} &=& f_{y_+}\left({\red g_{yy}}(g_y\otimes g_y)+g_y{\red g_{yy}}\right)+f_{y_0}{\red g_{yy}}+B\\
&=&0 
  \end{eqnarray*}
where $B$ is a term that doesn't contain second order derivatives of $g()$.

The equation can be rearranged:
\[
\left(f_{y_+}g_y+f_{y_0}\right){\red g_{yy}}+f_{y_+}{\red g_{yy}}(g_y\otimes g_y)=-B
\]
This is a Sylvester type of equation and must be solved with an appropriate algorithm.
\end{slide}

\subsection[]{}
\begin{slide}{Recovering $g_{yu}$}
  \begin{eqnarray*}
    F_{y_-u} &=& f_{y_+}\left(g_{yy}(g_y\otimes g_u)+g_y{\red g_{yu}}\right)+f_{y_0}{\red g_{yu}}+B\\
&=&0 
  \end{eqnarray*}
where $B$ is a term that doesn't contain second order derivatives of $g()$.

This is a standard linear problem:
\[
{\red g_{yu}}=-\left(f_{y_+}g_y+f_{y_0}\right)^{-1}\left(B+f_{y_+}g_{yy}(g_y\otimes g_u)\right)
\]
\end{slide}


\subsection[]{}
\begin{slide}{Recovering $g_{uu}$}
  \begin{eqnarray*}
    F_{uu} &=& f_{y_+}\left(g_{yy}(g_u\otimes g_u)+g_y{\red g_{uu}}\right)+f_{y_0}{\red g_{uu}}+B\\
&=&0 
  \end{eqnarray*}
where $B$ is a term that doesn't contain second order derivatives of $g()$.

This is a standard linear problem:
\[
{\red g_{uu}}=-\left(f_{y_+}g_y+f_{y_0}\right)^{-1}\left(B+f_{y_+}g_{yy}(g_u\otimes g_u)\right)
\]
\end{slide}

\subsection[]{}
\begin{slide}{Recovering $g_{y\sigma}$, $g_{u\sigma}$}
  \begin{eqnarray*}
    F_{y\sigma} &=& f_{y_+}g_y{\red g_{y\sigma}}+f_{y_0}{\red g_{y\sigma}}\\
&=&0 \\
    F_{u\sigma} &=& f_{y_+}g_y{\red g_{u\sigma}}+f_{y_0}{\red g_{u\sigma}}\\
&=&0 
  \end{eqnarray*}
as $g_\sigma=0$. Then
\[
g_{y\sigma}=g_{u\sigma}=0
\]
\end{slide}

\subsection[]{}
\begin{slide}{Recovering $g_{\sigma\sigma}$}
  \begin{eqnarray*}
    F_{\sigma\sigma}+F_{u'u'}\Sigma_\epsilon &=& f_{y_+}\left({\red g_{\sigma\sigma}}+g_y{\red g_{\sigma\sigma}}\right)+f_{y_0}{\red g_{\sigma\sigma}}\\
&&+\left(f_{y_+y_+}(g_u\otimes g_u)+f_{y_+}g_{uu}\right)\vec\Sigma_\epsilon\\
&=&0 
  \end{eqnarray*}
taking into account $g_\sigma=0$.

This is a standard linear problem:
\[
{\red g_{\sigma\sigma}}=-\left(f_{y_+}(I+g_y)+f_{y_0}\right)^{-1}\left(f_{y_+y_+}(g_u\otimes g_u)+f_{y_+}g_{uu}\right)\vec\Sigma_\epsilon
\]
\end{slide}

\section[Approximated decision function]{Approximated decision function}
\begin{slide}{Second order decision functions}
{\tiny
\[
y_t = \bar y+{\red 0.5g_{\sigma\sigma}\sigma^2}+g_y\hat y+g_uu+0.5\left(g_{yy}(\hat y\otimes \hat y)+g_{uu}(u \otimes u)\right)+g_{yu}(\hat y \otimes u) 
\]
}
We can fix $\sigma=1$.
\bigskip

Second order accurate moments:
\begin{eqnarray*}
  \Sigma_y &=& g_y \Sigma_y g_y'+\sigma^2g_u\Sigma_\epsilon g_u'\\
  E\left\{y_t\right\} &=& \bar y+\left(I-g_y\right)^{-1}\left(0.5\left(g_{\sigma\sigma}+g_{yy}\vec\Sigma_y+g_{uu}\vec\Sigma_\epsilon\right)\right)
\end{eqnarray*}
\end{slide}

\subsection[]{}
\begin{slide}{Three different concepts}
  \begin{enumerate}
  \item (deterministic) steady state
  \item risky steady state
  \item unconditional expectation
  \end{enumerate}
\end{slide}

\subsection[]{}
\begin{slide}{Deterministic steady state}
{\tiny
A linearized decision rule cuts the main diagonal at the deterministic steady state ($K_{ss}$).
  }
\begin{center}
\includegraphics[width=60mm]{sss1.eps}  
\end{center}
\end{slide}

\subsection[]{}
\begin{slide}{Quadratic decision rule}
{\tiny
In general, the decision is shifted at the deterministic steady state: agents don't decide to stay at the deterministic steady state.
}
\begin{center}
\includegraphics[width=60mm]{sss2.eps}  
\end{center}
\end{slide}
 
\subsection[]{}
\begin{slide}{Quadratic decision rule}
{\tiny
The distance between A and B is $g_{\sigma\sigma}/2$
}  
\begin{center}
\includegraphics[width=60mm]{sss3.eps}  
\end{center}
\end{slide}

\subsection[]{}
\begin{slide}{Risky steady state}
{\tiny
The risky steady state, $K_{sss}$, describes the point where agents decide to stay in absence of shocks this period, but taking into account the distibution of shocks in the future.
}
\begin{center}
\includegraphics[width=60mm]{sss4.eps}  
\end{center}
\end{slide}

\subsection[]{}
\begin{slide}{Unconditional expectation}
{\tiny
Because of Jensen inequality, the unconditional expectation, $E(K)$, is somewhere below the quadratic decision rule, but not on it. In absence of shocks, agents don't decide to go to the unconditional expectation.
}
\begin{center}
\includegraphics[width=60mm]{sss4.eps}  
\end{center}
\end{slide}

\begin{frame}
  \frametitle{Higher order approximation (I)}
  The Fa di Bruno formula for the $k$th derivative of the composition
  of two functions, $f(z(s))$:
\[
\left[F^i_{s^j}\right]_{\alpha_1\ldots\alpha_j} =
\sum_{l=1}^j\left[f^i_{z^l}\right]_{\beta_1\ldots\beta_l}\sum_{c \in{\mathcal
M}_{l,j}}\prod_{m=1}^l\left[z_{s^{|c_m|}}\right]^{\beta_m}_{\boldsymbol{\alpha}(c_m)}
\]
where ${\mathcal M}_{l,j}$ is the set of all partitions of the set of
$j$ indices with $l$ classes, $|.|$ is the cardinality of a set, $c_m$
is $m$-th class of partition $c$, and {\boldmath $\alpha$}$(c_m)$ is a
sequence of $\alpha$'s indexed by $c_m$.  Note that ${\mathcal
  M}_{1,j} = \{\{1,\ldots,j\}\}$ and ${\mathcal M}_{j,j} = \{\{1\},
\{2\}, \ldots, \{j\}\}$.  In order to keep the formulas compact, we
use {\boldmath $\alpha$}$_n$ for $\alpha_1\ldots\alpha_n$.

\end{frame}

\begin{frame}
  \frametitle{Higher order approximation (II)}
  In order to recover the $k$th
order derivatives of the decision function, $g_{y^k}$, it is necessary
to solve the following equation:
\[
\left(f_{y_+}g_y+f_{y_0}\right)g_{y^k}+f_{y_+}g_{y^k}g_y^{\otimes
  k}=-B
\]
where $g_y^{\otimes k}$ is the $k$th Kronecker power of matrix $g_y$
and $B$ is a term that doesn't contain the unknown $k$-order
derivatives of function $g()$, but only lower order derivatives of
$g()$ and first to $k$-order derivatives of $f()$.
\end{frame}

\section[Further issues]{Further issues}
\begin{slide}{Further issues}
  \begin{itemize}
  \item Impulse response functions depend of state at time of shocks and history of future shocks.
  \item For large shocks second order approximation simulation may explode
    \begin{itemize}
    \item pruning algorithm (Sims)
    \item truncate normal distribution (Judd)
    \end{itemize}
  \end{itemize}
\end{slide}

\section[Example]{Example}
\begin{slide}{An asset pricing model}
Urban Jermann (1998) ``Asset pricing in production economies'' \emph{Journal of Monetary Economics}, \emph{41}, 257--275.

\begin{itemize}
\item real business cycle model
\item consumption habits
\item investment adjustment costs
\item compares return on several securities
\item log--linearizes RBC model + log normal formulas for asset pricing
\end{itemize}
\end{slide}

\subsection[]{}

\begin{slide}{Firms}
{\small
The representative firm maximizes its value:
\[
{\mathcal E}_t\sum_{t+k}^\infty \beta^k\frac{\mu_{t+k}}{\mu_t}D_t
\]
with
\begin{eqnarray*}
  Y_t &=& A_tK_{t-1}^\alpha\left(X_tN_t\right)^{1-\alpha}\\
D_t &=& Y_t - W_tNt-I_t\\
K_t &=& (1-\delta)K_{t-1}+\left(\frac{a_1}{1-\xi}\left(\frac{I_t}{K_{t-1}}\right)^{1-\frac{1}{x}}+a_2\right)K_{t-1}\\
\log A_t &=& \rho \log A_{t-1}+e_t\\
X_t &=& (1+g)X_{t-1}
\end{eqnarray*}
}
\end{slide}

\subsection[]{}
\begin{slide}{Households}
The representative households maximizes current value of future utility:
\[
{\mathcal E}_t \sum_{k=0}^\infty\beta^k\frac{\left(C_t-\chi C_{t-1}\right)^{1-\tau}}{1-\tau}
\]
subject to the following budget constraint:
\[
W_tN_t+D_t = C_t
\]
and with $N_t=1$.
Good market equilibrium imposes
\[
Y_t = C_t+I_t
\]
\end{slide}

\subsection[]{}
\begin{slide}{Interest rate}
Risk free interest rate:
\[
r_f = \frac{1}{{\mathcal E}_t\left\{\beta g^{-\tau}\frac{\mu_{t+1}}{\mu_t}\right\}}
\]
where $\mu_t$ is the utility of a marginal unit of consumption in period $t$.
\[
\mu_t = \left(c_t-\chi c_{t-1}/g\right)^{-\tau}-\chi \beta\left(g c_{t+1}-\chi c_t\right)^{-\tau}
\]
\end{slide}

\subsection[]{}
\begin{slide}{Rate of return}
Rate of return of firms
\begin{eqnarray*}
r_t  &=& {\mathcal E}_t\Bigg\{a_1\left(\frac{g i_t}{k_{t-1}}\right)^{-\frac{1}{\xi}}\Bigg(\alpha z_{t+1}g^{1-\alpha}k_t^{\alpha-1}\\
&&+\frac{1-\delta+\frac{a_1}{1-\frac{1}{\xi}}\left(\frac{g i_{t+1}}{k_t}\right)^{1-\frac{1}{\xi}}+a_2}{a_1\left(\frac{g i_{t+1}}{k_t}\right)^{-\frac{1}{\xi}}}-\frac{g i_{t+1}}{k_t}\Bigg)\Bigg\}
\end{eqnarray*}
\end{slide}

\subsection[]{}
\begin{slide}{jermann98.mod}
{\tiny\begin{verbatim}
//---------------------------------------------------------------------
// 1. Variable declaration
//---------------------------------------------------------------------

var c, d, erp1, i, k, r1, rf1, w, y, z, mu; 
varexo ez;                          

\end{verbatim}
 } 
\end{slide}

\subsection[]{}
\begin{slide}{(continued)}
{\tiny\begin{verbatim}
  //---------------------------------------------------------------------
// 2. Parameter declaration and calibration
//---------------------------------------------------------------------

parameters alf, chihab, xi, delt, tau, g, rho, a1, a2, betstar, bet;

alf        = 0.36;    // capital share in production function
chihab     = 0.819;   // habit formation parameter
xi         = 1/4.3;   // capital adjustment cost parameter
delt       = 0.025;   // quarterly deprecition rate
g          = 1.005;   //quarterly growth rate (note zero growth =>g=1)
tau        = 5;       // curvature parameter with respect to c
rho        = 0.95;    // AR(1) parameter for technology shock

a1         = (g-1+delt)^(1/xi);             
a2         = (g-1+delt)-(((g-1+delt)^(1/xi))/(1-(1/xi)))*
             ((g-1+delt)^(1-(1/xi))); 
betstar    = g/1.011138;
bet        = betstar/(g^(1-tau));             
\end{verbatim}
}
\end{slide}

\subsection[]{}
\begin{slide}{(continued)}
{\tiny\begin{verbatim}
//---------------------------------------------------------------------
// 3. Model declaration
//---------------------------------------------------------------------

model;  
g*k  = (1-delt)*k(-1) + ((a1/(1-1/xi))*(g*i/k(-1))^(1-1/xi)+a2)*k(-1);
d    = y - w - i; 
w    = (1-alf)*y;
y    = z*g^(-alf)*k(-1)^alf;
c    = w + d; 
mu   = (c-chihab*c(-1)/g)^(-tau)-chihab*bet*(c(+1)*g-chihab*c)^(-tau);
mu   = (betstar/g)*mu(+1)*(a1*(g*i/k(-1))^(-1/xi))*(alf*z(+1)*g^(1-alf)*
       (k^(alf-1))+((1-delt+(a1/(1-1/xi))*(g*i(+1)/k)^(1-1/xi)+a2))/
       (a1*(g*i(+1)/k)^(-1/xi))-g*i(+1)/k);
log(z) = rho*log(z(-1)) + ez;
\end{verbatim}
}
\end{slide}

\subsection[]{}
\begin{slide}{(continued)}
{\tiny\begin{verbatim}
rf1  = 1/expecation(0)(betstar/g)*mu(+1)/mu);
r1   = (a1*(g*i/k(-1))^(-1/xi))*(alf*z(+1)*g^(1-alf)*(k^(alf-1))+
       (1-delt+(a1/(1-1/xi))*(g*i(+1)/k)^(1-1/xi)+a2)/
       (a1*(g*i(+1)/k)^(-1/xi))-g*i(+1)/k);
erp1 = r1 - rf1;

end;
\end{verbatim}
}
\end{slide}

\subsection[]{}
\begin{slide}{(continued)}
{\tiny\begin{verbatim}
steady_state_model;
rf1    = (g/betstar);
r1     = (g/betstar);
erp1   = r1-rf1;
z      = 1;
k      = (((g/betstar)-(1-delt))/(alf*g^(1-alf)))^(1/(alf-1));
y      = (g^(1-alf))*k^alf;
w      = (1-alf)*y;
i      = (1-(1/g)*(1-delt))*k;
d      = y - w - i;
c      = w + d;
mu     = ((c-(chihab*c/g))^(-tau))-chihab*bet*((c*g-chihab*c)^(-tau));
ez     = 0;
end;
\end{verbatim}
}
\end{slide}

\subsection[]{}
\begin{slide}{(continued)}
{\tiny
\begin{verbatim}
steady;                      

shocks;
var ez; stderr 0.01;  
end;

stoch_simul (order=2) rf1, r1, erp1, y, z, c, d, mu, k;
\end{verbatim}
}
\end{slide}

\begin{frame}
  \frametitle{3rd order approximation}
  \begin{itemize}
  \item same principle of derivation as 2nd order
  \item Don't forget options periods= in order to compute empirical moments
  \item No pruning at 3rd order
  \end{itemize}
\end{frame}
\end{document}
