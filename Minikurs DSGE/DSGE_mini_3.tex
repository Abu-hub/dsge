\documentclass[10pt]{beamer}  %%% F\"{U}R HANDOUT ALS PDF
\usetheme{metropolis}
\usepackage{iftex}
\ifPDFTeX
    \usepackage[T1]{fontenc}
    \usepackage[utf8]{inputenc}
\fi
\ifXeTeX
\fi
\ifLuaTeX
\fi
\usepackage[ngerman]{babel}
\usepackage{amsmath,amsfonts,amssymb}
\usepackage{cancel}
\usepackage[]{hyperref}
\usepackage{xcolor,pstricks}
\usepackage{appendixnumberbeamer}
\title{DSGE Mini Kurs - Teil 3 (L\"{o}sungsmethoden)}
\author{Willi Mutschler}


\begin{document}


\begin{frame}
    \titlepage
\end{frame}

\begin{frame}\frametitle{Inhaltsverzeichnis}
\setbeamertemplate{section in toc}[sections numbered]
    \tableofcontents
\end{frame}

\section{Allgemeine Darstellung und L\"{o}sung}
\begin{frame}\frametitle{}\framesubtitle{}\small
Die strukturelle Form eines DSGE-Modells besteht aus
\begin{itemize}
\item Menge von erwarteten, nichtlinearen Optimalit\"{a}tsbedingungen,
\item Bewegungsgleichungen f\"{u}r stochastische Prozesse,
\item Messgleichungen, zur Verkn\"{u}pfung von Modellvariablen mit Daten.
\end{itemize}
Kurzum: Strukturelle Form ist ein nichtlineares System erwarteter Differenzengleichungen
\begin{align}
 E_t f(x_{t+1},\varepsilon_{t+1},y_{t+1},x_t,\varepsilon_t,y_t|\theta) = 0 \label{eq:model}
\end{align}
\begin{itemize}
\item Vektor $\varepsilon_t$ enth\"{a}lt die stochastischen Innovationen (Schocks und Spezifikationsfehler)
\begin{itemize}
  \item mit $E(\varepsilon_t) = 0$, $E(\varepsilon_t \varepsilon_s') = \begin{cases} \Sigma_\varepsilon = \eta \eta' & \text{f\"{u}r } s=t,\\
                                               0    & \text{sonst}
                                 \end{cases}$
\end{itemize}
\item Vektor $y_t$ enth\"{a}lt die Kontrollvariablen
\item Vektor $x_t$ enth\"{a}lt die exogenen und endogenen Zustandsvariablen
\begin{itemize}\footnotesize
\item Exogene: von den Entscheidungen der Akteure unabh\"{a}ngig entwickelnde Variablen
\item Endogene: von den Entscheidungen der Akteure beeinflussbare Variablen
\end{itemize}
\item $\theta$ ist der Vektor der Modellparameter
\end{itemize}
\end{frame}


\begin{frame}\frametitle{Zustandsraumdarstellung}\framesubtitle{}
\begin{itemize}
\item Ein solches Modell rationaler Erwartungen zu l\"{o}sen bedeutet, sogenannte Politikfunktionen $h$ und $g$ zu finden, die das obige System (zumindest approximativ) l\"{o}sen:
\begin{align}
  x_{t+1} &= h(x_t,\varepsilon_{t+1})\\
  y_{t+1} &= g(x_t,\varepsilon_{t+1})
\end{align}
\item DSGE Modelle lassen sich als Zustandsraummodelle auffassen:
\begin{itemize}
\item wir suchen das Optimale Verhalten der Akteure als Funktion des aktuellen Zustands der \"{O}konomie
\item Aktueller Zustand ist eine Funktion vergangener Zust\"{a}nde und aktueller Innovationen
\end{itemize}
\item Problem: $g$ und $h$ sind nichtlinear und im Allgemeinen analytisch nicht herleitbar
\item Gesucht ist eine Sattelpfad-stabile L\"{o}sung, d. h. eine eindeutigen Zuordnung von Zustands- und Kontrollvariablen.
\end{itemize}

\end{frame}

\section{L\"{o}sungsmethoden}
\begin{frame}\frametitle{L\"{o}sung eines DSGE-Modells}\framesubtitle{}
Es gibt lineare und nichtlineare L\"{o}sungsverfahren:
\begin{itemize}
\item Lineare Verfahren:
\begin{itemize}
\item Log-linearisierung der Modellgleichungen \eqref{eq:model} um den steady-state liefert:
\begin{align}
 \Gamma_0 E_t\begin{bmatrix}\hat{x}_{t+1}\\\hat{y}_{t+1}\end{bmatrix} = \Gamma_1 \begin{bmatrix}\hat{x}_{t}\\\hat{y}_{t}\end{bmatrix} + \Gamma_\varepsilon \varepsilon_{t+1}\label{eq:logmodel}
\end{align}
mit $\hat{x}_t=log(x_t) - log(x)$ und $\hat{y}_t=log(y_t) - log(y)$
\item Anderson/Moore (1983), Binder und Pesaran (1997), Blanchard/Khan (1980), Christiano (2002), King/Watson (1998), \boxed{Klein (2000)}, \boxed{Sims (2001)}, Uhlig (1999)
\item Guter \"{U}berblick: Anderson (2008).
\end{itemize}
\item Nichtlineare Verfahren:
\begin{itemize}
\item Projektion, Iteration oder \boxed{Perturbation}
\item Guter \"{U}berblick DeJong/Dave (2011) und Herr/Maussner (2009).
\end{itemize}
\end{itemize}
\end{frame}


\subsection{Lineare Verfahren: Klein (2000)}

\begin{frame}
\frametitle{\secname}\framesubtitle{\subsecname, computing $g_x,h_x$}\footnotesize
\begin{itemize}
  \item Take derivative of $F$ w.r.t. $x_t$ and evaluate at the non-stochastic steady-state
  \begin{eqnarray*}
&    F_x (\overline{x},0) = f_1 h_x + f_2 g_x h_x + f_3 + f_4 g_x = 0\\
& \underbrace{- \begin{bmatrix} \underset{(n\times n_x)}{f_1} & \underset{(n\times n_y)}{f_2} \end{bmatrix}}_{:=A} \begin{bmatrix} \underset{(n_x\times n_x)}{h_x} \\ \underset{(n_y\times n_x)}{g_x} \underset{(n_x\times n_x)}{h_x} \end{bmatrix}  = \underbrace{\begin{bmatrix} \underset{(n\times n_x)}{f_3} & \underset{(n\times n_y)}{f_4}\end{bmatrix}}_{:=B} \begin{bmatrix} \underset{(n_x\times n_x)}{I} \\\underset{(n_y\times n_x)}{g_x} \end{bmatrix}
  \end{eqnarray*}
  \item $n\times n_x$ equations for $n\times n_x$ unknown elements of $h_x$ and $g_x$
  \item Postmultiply by $\widehat{x}_t := (x_t-\overline{x})$
  \begin{align*}
    A \begin{bmatrix} h_x \widehat{x}_t \\ g_x h_x \widehat{x}_t \end{bmatrix} = B \begin{bmatrix} \widehat{x}_t \\ g_x \widehat{x}_t \end{bmatrix}
  \end{align*}
  \item Notice that the coefficient matrices are equivalent to the first order approximation
  \begin{eqnarray*}
    AE_t\begin{bmatrix} \widehat{x}_{t+1} \\ \widehat{y}_{t+1} \end{bmatrix} = B\begin{bmatrix} \widehat{x}_t\\ \widehat{y}_t \end{bmatrix} + \begin{bmatrix} \sigma \eta_x \varepsilon_{t+1} \\ 0\end{bmatrix}
  \end{eqnarray*}
\end{itemize}
\end{frame}



\begin{frame}
\frametitle{Lineare Verfahren: Klein (2000) (I)}\framesubtitle{}
\begin{itemize}
  \item Ausgangspunkt ist das log-linearisierte Modell in Gleichung \eqref{eq:logmodel}
  \begin{align*}
 \Gamma_1 E_t\begin{bmatrix}\hat{x}_{t+1}\\\hat{y}_{t+1}\end{bmatrix} = \Gamma_0 \begin{bmatrix}\hat{x}_{t}\\\hat{y}_{t}\end{bmatrix} + \Gamma_\varepsilon \varepsilon_{t+1}\label{eq:logmodel}
  \end{align*}
  \item Die lineare L\"{o}sung besitzt die Form
  \begin{align*}
  \hat{x}_{t} = h_x \hat{x}_{t-1} + h_\varepsilon \varepsilon_{t}\\
  \hat{y}_{t} = g_x \hat{x}_{t-1} + g_\varepsilon \varepsilon_{t}
  \end{align*}
  \item $\Gamma_1$, $\Gamma_0$ und $\Gamma_\varepsilon$ sind bekannte Matrizen, wir m\"{u}ssten nur $\Gamma_1$ invertieren...
\item \textbf{ABER:} $\Gamma_1$ ist im Allgemeinen singul\"{a}r und nicht invertierbar
  \item Klein (2000)'s Ansatz:
  \begin{itemize}
      \item Entkopplung des Systems in ein Block-dreieckiges Gleichungssystem mithilfe der verallgemeinerten Schur Dekomposition
      \item System l\"{a}sst sich dann rekursiv l\"{o}sen
  \end{itemize}
\end{itemize}
\end{frame}


\section{Aside: Matrix theory}


\begin{frame}
\frametitle{\secname}\framesubtitle{}\footnotesize
\begin{block}{Matrix pencil}
  Let $A$ and $B$ be two $n\times n$ matrices. The set of all matrices of the form $A-\lambda B$ with $\lambda \in \mathbb{C}$ is said to be a \emph{pencil}. The eigenvalues of the pencil are elements of the set $\lambda(A,B)$ defined by
  \begin{align*}
    \lambda(A,B) =  \left\{z \in \mathbb{C}:det(A-zB)=0\right\}
  \end{align*}
\end{block}
\begin{block}{Generalized Eigenvalue problem}
  Let $A$ and $B$ be two $n\times n$ matrices. Then $\lambda \in \lambda(A,B)$ is called a generalized Eigenvalue if there exist a nonzero vector $q\in \mathbb{C}^n$ such that $$A q = \lambda B q$$
\end{block}
\begin{itemize}
\item If $B=I$, then this simplifies to the ordinary Eigenvalue problem: $A q = \lambda q$
\item A always has $n$ eigenvalues, which can be ordered (in more than one way) to form an $n\times n$ diagonal matrix $\Lambda$ and a corresponding matrix of nonzero columns $Q$ that satisfies the eigenvalue equation: $$AQ = Q\Lambda$$
\end{itemize}
\end{frame}



\begin{frame}[shrink]
\frametitle{\secname}\framesubtitle{}\footnotesize
\begin{block}{Schur decomposition (Complex version)}
  Let $A$ be an $n \times n$ matrix. Then there exist a \textbf{unitary} $n\times n$ matrix $S$ (that is, $S^*S=SS^*=S^{-1}S=I_n$) and an upper triangular matrix $M$ whose diagonal elements are the eigenvalues of $B$, such that
  \begin{align*}
    S^* A S = M \Leftrightarrow A = S M S^*
  \end{align*}
\end{block}
\begin{block}{Schur decomposition (Real version)}
  Let $A$ be a real symmetric $n \times n$ matrix. Then there exist an \textbf{orthogonal} real $n\times n$ matrix $S$ (that is, $S'S=SS'=S^{-1}S=I_n$), whose columns are eigenvectors of $A$ and a diagonal matrix $\Lambda$ whose diagonal elements are the eigenvalues of $A$, such that
  \begin{align*}
    S' A S = \Lambda \Leftrightarrow A = S \Lambda S'
  \end{align*}
\end{block}
\begin{itemize}
  \item $~^*$ denotes conjugate or Hermitian transpose, $~'$ denotes the ordinary transpose.
  \item A complex matrix always has a complex Schur decomposition.
  \item A real matrix has a real Schur decomposition if and only if all eigenvalues are real.
  \item $S$ is structured, i.e. unitary or orthogonal.
  \item Useful for proofs (e.g. Eigenvalues of Kronecker products, differentials,\dots) and numerically attractive.
\end{itemize}
\end{frame}

\begin{frame}[shrink]\scriptsize
\frametitle{\secname}\framesubtitle{}
\begin{block}{Generalized (complex) Schur decomposition or QZ decomposition}
Let $A$ and $B$ be $n\times n$ matrices. Then there exist matrices $Q,Z,T$ and $S$ such that
\begin{align*}
  Q^* A Z &= S \Leftrightarrow A = Q S Z^*\\
  Q^* B Z &= T \Leftrightarrow B = Q T Z^*
\end{align*}
\begin{enumerate}
  \item $Q$ and $Z$ are unitary, i.e. $Q^*Q=QQ^*=I_n$ and $Z^*Z=ZZ^*=I_n$.
  \item $S$ and $T$ are upper triangular.
 \item pairs $(s_{ii},t_{ii})$ can be arranged in any desired order.
  \item If for some $i$, $t_{ii}$ and $s_{ii}$ are both zero, then $\lambda(A,B)=\mathbb{C}$. Otherwise:
$\lambda(A,B) = \left\{\frac{t_{ii}}{s_{ii}}:s_{ii} \neq 0 \right\}$
\end{enumerate}
\end{block}
\begin{itemize}\scriptsize
  \item There is also a real version.
  \item We will limit ourselves to the case $\lambda(A,B)\neq \mathbb{C}$ and rule-out unit roots, that is $t_{ii}$ and $s_{ii}$ are not both zero, and $|t_{ii}|\neq |s_{ii}|$.
  \item Eigenvalues
  \begin{itemize}\scriptsize
  \item If A is singular, then there are some generalized eigenvalues missing, i.e. $s_{ii}=0$ for some $i$ $\Rightarrow$ call these \emph{infinite},
  \item If $|\lambda_i|>1 \Leftrightarrow |s_{ii}|<|t_{ii}|$ $\Rightarrow$ call these finite and unstable,
  \item If $|\lambda_i|<1 \Leftrightarrow |s_{ii}|>|t_{ii}|$ $\Rightarrow$ call these finite and stable.
  \end{itemize}
\end{itemize}
\end{frame}



\begin{frame}
\frametitle{Lineare Verfahren: Klein (2000) (II)}\framesubtitle{}
\begin{itemize}
\item Die verallgemeinerte Schur Dekomposition von $\Gamma_1$ und $\Gamma_0$ ist gegeben durch
\begin{align*}
  Q^* \Gamma_1 = S Z^*,\qquad   Q^* \Gamma_0 = T Z^*
\end{align*}
\begin{itemize}
\item wobei folgende Anordnung gew\"{a}hlt wird: stabile verallgemeinerte Eigenwerte $(|s_{ii}|>|t_{ii}|)$ kommen zuerst
\end{itemize}
\item Multiplikation mit $Q^*$ und
$\begin{bmatrix} \underset{n_x \times 1}{s_t} \\ \underset{n_y\times1}{u_t} \end{bmatrix} : = Z^* \begin{bmatrix} \underset{n_x \times 1}{\widehat{x}_t} \\ \underset{n_y \times 1}{\widehat{y}_t} \end{bmatrix}$, ergibt
$
  S  \begin{bmatrix} E_t s_{t+1}\\ E_t u_{t+1} \end{bmatrix} = T \begin{bmatrix} s_t \\ u_t \end{bmatrix}
$
\end{itemize}
\end{frame}

\begin{frame}
\frametitle{Lineare Verfahren: Klein (2000) (III)}\framesubtitle{}
\begin{itemize}
\item $S$ und $T$ sind obere Dreiecksmatrizen:
\begin{align*}
\begin{bmatrix} \underset{n_x \times n_x}{S_{11}} & \underset{n_x \times n_y}{S_{12}} \\ \underset{n_y \times n_x}{0} & \underset{n_y \times n_y}{S_{22}}\end{bmatrix} \begin{bmatrix} E_t s_{t+1}\\ E_t u_{t+1} \end{bmatrix} = \begin{bmatrix} \underset{n_x \times n_x}{T_{11}} & \underset{n_x \times n_y}{T_{12}} \\ \underset{n_y \times n_x}{0} & \underset{n_y \times n_y}{T_{22}}\end{bmatrix} \begin{bmatrix} s_t \\ u_t \end{bmatrix}
\end{align*}
\item L\"{o}sung des unteren Blocks ergibt
\begin{align*}
  S_{22} E_t[u_{t+1}] = T_{22} u_t
\end{align*}
\item Aufgrund der gew\"{a}hlten Anordnung gilt f\"{u}r den unteren Block $|s_{ii}|<|t_{ii}|$ (oder $|s_{ii}|\leq|t_{ii}|$ ).
\item Vorw\"{a}rts einsetzen ergibt, dass f\"{u}r eine L\"{o}sung mit beschr\"{a}nktem Erwartungswert und beschr\"{a}nkter Varianz ungleich 0 gelten muss, dass $u_t=0$ f\"{u}r alle $t$ gilt. Sonst h\"{a}tten wir ein explosives System.
\end{itemize}
\end{frame}

\begin{frame}
\frametitle{Lineare Verfahren: Klein (2000) (IV)}\framesubtitle{}
\begin{itemize}
\item \textbf{Achtung:} Die Anzahl der Zustandsvariablen muss der Anzahl verallgemeinerter Eigenwerte mit $|s_{ii}|>|t_{ii}|$ entsprechen!
\item Diese Eigenschaft hat einen Namen
\begin{block}{Blanchard/Khan Bedingungen}
Die Anzahl der verallgemeinerten Eigenwerte, die im Modulus kleiner als 1 sind, muss gleich der Anzahl der Zustandsvariablen sein, um eine stabile L\"{o}sung zu erhalten.
\end{block}
\end{itemize}
\end{frame}


\begin{frame}
\frametitle{Lineare Verfahren: Klein (2000) (V)}\framesubtitle{}
Gegeben $u_t=0$, k\"{o}nnen wir $x_t$ und $y_t$ aus der Definition von $s_t$ und $u_t$ zur\"{u}ck herleiten:
\begin{align*}
  \begin{bmatrix} x_t \\ y_t \end{bmatrix} = \begin{bmatrix} Z_{11} & Z_{12}\\ Z_{21} & Z_{22} \end{bmatrix} \begin{bmatrix} s_t \\ u_t \end{bmatrix}
  =
  \begin{bmatrix}
    Z_{11} s_t\\
    Z_{21} s_t
  \end{bmatrix}
\end{align*}
Wenn $Z_{11}$ invertierbar ist, dann folgt
\begin{align*}
  y_t = \underbrace{Z_{21} Z_{11}^{-1}}_{=g_x} x_t
\end{align*}
Um also $g_x$ zu berechnen, brauchen wir die Nichtsingularit\"{a}t von $Z_{11}$!
\end{frame}

\begin{frame}
\frametitle{Lineare Verfahren: Klein (2000) (VI)}\framesubtitle{}
Mithilfe von $u_t=0$ und nichtsingul\"{a}rem $Z_{11}$ l\"{a}sst sich der erste Block l\"{o}sen
\begin{align*}
   E_t[s_{t+1}] &= S_{11}^{-1} T_{11} s_t
\end{align*}
$S_{11}$ ist so konstruiert, dass es immer invertierter ist. Einsetzen von $s_t = Z_{11}^{-1} x_t$ ergibt
\begin{align*}
   E_t[ x_{t+1}] = \underbrace{Z_{11} S_{11}^{-1} T_{11} Z_{11}^{-1}}_{=h_x} x_t
\end{align*}
F\"{u}r die Berechnung von $h_x$ ben\"{o}tigt man nichtsingul\"{a}re $S_{11}$ und $Z_{11}$.
\end{frame}

\begin{frame}
\frametitle{Lineare Verfahren: Klein (2000) (VII)}\framesubtitle{}
\begin{block}{Algorithmus Klein (2000)}
\begin{itemize}
  \item Berechne verallgemeinerte Schur Dekomposition von $\Gamma_1=-\begin{bmatrix} f_1 & f_2\end{bmatrix}$ und $\Gamma_0=\begin{bmatrix} f_3 & f_4\end{bmatrix}$.
  \item Ordne die verallgemeinerten Eigenwerte nach Gr\"{o}{\ss}e an, so dass $|s_{ii}|>|t_{ii}|$ oben links ist.
  \item \"{U}berpr\"{u}fe die Anzahl stabiler Eigenwerte (Blanchard-Khan Bedingungen).
  \item \"{U}berpr\"{u}fe, ob $Z_{11}$ invertierbar ist.
  \item Berechne $h_x=Z_{11} S_{11}^{-1} T_{11} Z_{11}^{-1}$ und $g_x=Z_{21} Z_{11}^{-1}$.
\end{itemize}
\end{block}

\end{frame}

\begin{frame}\frametitle{Lineare Verfahren: Klein (2000) (VI)}\framesubtitle{Beispiel: An und Schorfheide (2007)}
\begin{block}{Excercise}
Consider the log-linearized version of the CGG-model:
\begin{align*}
  \pi_{t} &=\beta E_{t}\pi _{t+1}+\kappa x_{t} &\text{(Phillips curve)} \\
x_{t} &= E_{t}x_{t+1}-\left( r_{t}-E_{t}\pi _{t+1}-r_{t}^{**}\right) &\text{(IS equation)} \\
r_{t} &=\alpha r_{t-1}+(1-\alpha )\left[ \phi _{\pi }\pi _{t}+\phi _{x}x_{t}\right] &\text{(baseline Taylor-rule)} \\
\Delta a_t &= \rho_a \Delta a_{t-1} + \varepsilon_{t}^a &\text{(technological shock)}\\
\tau_t &= \rho_\tau \tau_{t-1} + \varepsilon_{t}^\tau &\text{(preference shock)}\\
r_{t}^{**} &= \rho_a \Delta a_{t}+\frac{1-\rho_\tau}{1+\varphi } \tau_t &\text{(natural interest rate)} \\
\Delta y_t &= x_{t} - x_{t-1} + \Delta a_t - \frac{\tau_t - \tau_{t-1}}{1+\phi} &\text{(output growth)}
\end{align*}
\begin{enumerate}
  \item What are state variables, what are control variables?
  \item Solve this model using Klein(2000)'s algorithm with Matlab.
  \item Compare the solution to Dynare's policy functions.
\end{enumerate}
\end{block}
\end{frame}


\subsection{Lineare Verfahren: Sims (2001)}

\begin{frame}\frametitle{Lineare L\"{o}sungsverfahren}\framesubtitle{}
\begin{itemize}
   \item Zun\"{a}chst wird die allgemeine Form \eqref{AllgForm} um den \emph{steady-state} linearisiert bzw. log-linearisiert.
   \item Zusammen mit den Bewegungsgleichungen f\"{u}r die stochastischen Prozesse, ergibt sich das reduzierte Modell:
\begin{align*}
    \mathbf{A} \mathbf{x_{t+1}} =
\mathbf{B} \mathbf{x_t} + \mathbf{C} \boldsymbol{\upsilon_{t+1}} + \mathbf{D}
\boldsymbol{\eta_{t+1}} + \mathbf{E}.
\end{align*}
%\item Die Matrizen $\mathbf{A}, \mathbf{B}, \mathbf{C}$ und $\mathbf{D}$
%    sind dabei Funktionen des strukturellen Parametervektors
%    $\boldsymbol{\mu}$; $\mathbf{E}$ ist ein Vektor von Konstanten (oft
%    Null).
%\item Die L\"{o}sung dieser linearen Repr\"{a}sentation besitzt eine
%    VAR(1)-Form:
%\begin{align}\label{Linlsg}
%\mathbf{x_{t+1}} = \mathbf{F(\boldsymbol{\mu})} \mathbf{x_t} + \mathbf{G(\boldsymbol{\mu})} \boldsymbol{\upsilon_{t+1}},
%\end{align}
%wobei $\mathbf{F}$ und $\mathbf{G}$ Funktionen des Parametervektors
%$\boldsymbol{\mu}$ bezeichnen.
%\item Die treibende Kraft des Modells sind hierbei die exogenen Schocks
%    $\boldsymbol{\upsilon_t}$.
%\item \eqref{Linlsg} beschreibt somit die Fluktuationen um den
%    \emph{steady-state}, sowie das Modellverhalten als Antwort auf die
%    stochastischen Innovationen.
\end{itemize}
\end{frame}
\end{document}

\begin{frame}\frametitle{Der Sims (2001)-Algorithmus}\framesubtitle{Konzepte}
\begin{itemize}
\begin{block}{Notation}
Im Voraus bestimmte Variablen: $\qquad ~E_t X_{1,t+1}=X_{1,t+1}$,\\ Im
    Voraus nicht bestimmte Variablen: $E_t X_{2,t+1}=E_t X_{2,t+1}$.
\end{block}
\begin{block}{Unit\"{a}re Matrizen}
    $\mathbf{M}'\mathbf{M}=\mathbf{M}\mathbf{M}'=\mathbf{I}$ sind das
    komplexe Analogon zu orthogonalen Matrizen. Sie sind zudem
    diagonalisierbar.
\end{block}
   \item Die Methode von Sims (2001) beginnt mit einer
       \emph{QZ-Faktorisierung} (Verallgemeinerte Schur Dekomposition), bei der die Matrizen $\mathbf{A}$ und
       $\mathbf{B}$ in unit\"{a}re obere Dreiecksmatrizen transformiert
       werden:
\begin{align*}
 \mathbf{A} = \mathbf{Q}' \boldsymbol{\Lambda} \mathbf{Z}' ,\qquad \mathbf{B} = \mathbf{Q}' \boldsymbol{\Omega} \mathbf{Z}'.
\end{align*}
\item $\boldsymbol{\Lambda}$ und $\boldsymbol{\Omega}$ sind hier obere
    Dreiecksmatrizen mit den verallgemeinerten Eigenwerten von
    $\mathbf{A}$ und $\mathbf{B}$, und werden in aufsteigender
    Reihenfolge von links nach rechts sortiert.
\end{itemize}
\end{frame}

\begin{frame}\frametitle{Der Sims (2001)-Algorithmus}\framesubtitle{}
\begin{itemize}
   \item Die Eigenwerte sind entscheidend daf\"{u}r, ob ein Teilsystem
       konvergiert oder explodiert.
\begin{block}{Blanchard/Khan-Bedingungen}
Die Anzahl der Eigenwerte, die betragsm\"{a}{\ss}ig gr\"{o}{\ss}er gleich 1 sind, muss
    gleich der Anzahl der im Voraus unbestimmten Variablen sein, damit
    ein stabiler Sattelpfad existiert.
\end{block}
\item Bezeichne $\mathbf{z_{t+1}}=\mathbf{Z}'\mathbf{x_{t+1}}$, dann wird
    das System in einen nicht-explosiven Teil (oben) und einen explosiven
    Teil (unten) aufgeteilt:
\begin{align}\label{sims}
  \begin{bmatrix} \boldsymbol{\Lambda_{11}} & \boldsymbol{\Lambda_{12}}\\ \mathbf{0} & \boldsymbol{\Lambda_{22}} \end{bmatrix}
  \begin{pmatrix} \mathbf{z_{1,t+1}} \\ \mathbf{z_{2,t+1}} \end{pmatrix}
= \begin{bmatrix} \boldsymbol{\Omega_{11}}&\boldsymbol{\Omega_{12}}\\\mathbf{0}&\boldsymbol{\Omega_{22}}\end{bmatrix}
  \begin{pmatrix} \mathbf{z_{1,t}} \\ \mathbf{z_{2,t}} \end{pmatrix}\\
+ \begin{pmatrix} \mathbf{Q_1 }\\ \mathbf{Q_2} \end{pmatrix}
  \begin{bmatrix} \mathbf{E_1}+\mathbf{C_1}\boldsymbol{\upsilon_{1,t+1}} + \mathbf{D_1} \boldsymbol{\eta_{1,t+1}} \\ \mathbf{E_2}+\mathbf{C_2}\boldsymbol{\upsilon_{2,t+1}} +\mathbf{ D_2} \boldsymbol{\eta_{2,t+1}} \end{bmatrix}\nonumber.
\end{align}

\end{itemize}
\end{frame}

\begin{frame}\frametitle{Der Sims (2001)-Algorithmus}\framesubtitle{}
\begin{itemize}
   \item Die Differenzengleichungen zu den Eigenwerten gr\"{o}{\ss}er als eins
       werden vorw\"{a}rts gel\"{o}st.
   \item Beachte: $\underset{t\rightarrow
       \infty}{\lim}\left(\boldsymbol{\Omega_{22}}^{-1}
       \boldsymbol{\Lambda_{22}}\right)^t \mathbf{z_{2,t}}=\mathbf{0}$
       und f\"{u}r alle $s>0: E_t \boldsymbol{\upsilon_{2,t+s}} = E_t
\boldsymbol{\eta_{2,t+s}}=0$ (Erwartungsfehler spielen keine Rolle)
\begin{align*}
  \mathbf{z_{2,t}} & = \boldsymbol{\Omega_{22}}^{-1} \boldsymbol{\Lambda_{22}} \mathbf{z_{2,t+1}} - \boldsymbol{\Omega_{22}}^{-1}\mathbf{ Q_2} \left[\mathbf{E_2} + \mathbf{C_2} \boldsymbol{\upsilon_{2,t+1}} + \mathbf{D_2} \boldsymbol{\eta_{2,t+1}}\right]\\
  & = -\sum_{i=0}^\infty \left(\boldsymbol{\Omega_{22}}^{-1} \boldsymbol{\Lambda_{22}}\right)^i \boldsymbol{\Omega_{22}}^{-1} \mathbf{Q_2 }\left[\mathbf{E_2} + \mathbf{C_2} \boldsymbol{\upsilon_{2,t+1+i}} + \mathbf{D_2} \boldsymbol{\eta_{2,t+1+i}}\right]\\
  & = -\sum_{i=0}^\infty \left(\boldsymbol{\Omega_{22}}^{-1} \boldsymbol{\Lambda_{22}}\right)^i \boldsymbol{\Omega_{22}}^{-1} \mathbf{Q_2} \mathbf{E_2}\\
  & = -\left[\mathbf{I}-\boldsymbol{\Omega_{22}}^{-1}\boldsymbol{\Lambda_{22}}\right]^{-1} \boldsymbol{\Omega_{22}}^{-1} \mathbf{Q_2} \mathbf{E_2}
   = \left[\boldsymbol{\Lambda_{22}} - \boldsymbol{\Omega_{22}}\right]^{-1} \mathbf{Q_2} \mathbf{E_2}.
\end{align*}
\end{itemize}
\end{frame}

\begin{frame}\frametitle{Der Sims (2001)-Algorithmus}\framesubtitle{}
\begin{itemize}
   \item Die Differenzengleichungen zu den Eigenwerten kleiner als eins
       werden r\"{u}ckw\"{a}rts gel\"{o}st.
   \item Beachte: Systematische Verkn\"{u}pfung der Erwartungsfehler
       $\boldsymbol{\eta_{1,t}} \& \boldsymbol{\eta_{2,t}}$.
\begin{block}{Hinreichende Bedingung f\"{u}r stabilen Sattelpfad}\centering $\mathbf{Q_1}\mathbf{D} =
  \boldsymbol{\Phi} \mathbf{Q_2} \mathbf{D}$.
\end{block}
\item $\boldsymbol{\Phi}$ hat Dimension $n_s \times n_c$ (mit
    $\mathbf{z_{1,t}}: n_s\times1$ und $\mathbf{z_{2,t}}: n_c \times 1$).
\item Damit l\"{a}sst sich \eqref{sims} umformen zu:
\begin{multline*}
\underbrace{\begin{bmatrix} \underset{n_s \times n_s}{\mathbf{I}} &
\underset{n_s \times n_c}{-\boldsymbol{\Phi}}\end{bmatrix}}_{n_s \times
(n_s+n_c)}
  \begin{bmatrix}\boldsymbol{\Lambda_{11}} & \boldsymbol{\Lambda_{12}}\\\mathbf{0}&\boldsymbol{\Lambda_{22}}\end{bmatrix}
  \begin{pmatrix} \mathbf{z_{1,t}} \\ \mathbf{z_{2,t}}\end{pmatrix} =\\
  \begin{bmatrix} \mathbf{I} & -\boldsymbol{\Phi}\end{bmatrix} \begin{bmatrix} \boldsymbol{\Omega_{11}}&\boldsymbol{\Omega_{12}}\\\mathbf{0}&\boldsymbol{\Omega_{22}}\end{bmatrix}
  \begin{pmatrix} \mathbf{z_{1,t-1}} \\ \mathbf{z_{2,t-1}} \end{pmatrix}
+ \begin{bmatrix} \mathbf{I} & -\boldsymbol{\Phi}\end{bmatrix}
\begin{pmatrix} \mathbf{Q_1} \\ \mathbf{Q_2} \end{pmatrix}
  \begin{bmatrix} \mathbf{E}+\mathbf{C}\boldsymbol{\upsilon_{t}} + \mathbf{D} \boldsymbol{\eta_{t}} \end{bmatrix}.
\end{multline*}
\end{itemize}
\end{frame}

\begin{frame}\frametitle{Der Sims (2001)-Algorithmus}\framesubtitle{}
\begin{multline*} \Leftrightarrow \begin{bmatrix}
       \boldsymbol{\Lambda_{11}} &
       \boldsymbol{\Lambda_{12}}-\boldsymbol{\Phi}
       \boldsymbol{\Lambda_{22}} \end{bmatrix}
  \begin{pmatrix} \mathbf{z_{1,t}} \\ \mathbf{z_{2,t}}\end{pmatrix} =
  \begin{bmatrix} \boldsymbol{\Omega_{11}} & \boldsymbol{\Omega_{12}}-\boldsymbol{\Phi} \boldsymbol{\Omega_{22}} \end{bmatrix}
  \begin{pmatrix} \mathbf{z_{1,t-1}} \\ \mathbf{z_{2,t-1}}\end{pmatrix}\\
+ \begin{bmatrix}\mathbf{ Q_1 }- \boldsymbol{\Phi} \mathbf{Q_2}
\end{bmatrix}
  \begin{bmatrix} \mathbf{E}+\mathbf{C}\boldsymbol{\upsilon_{t}} \end{bmatrix}
+ \underbrace{\begin{bmatrix} \left(\mathbf{Q_1} - \boldsymbol{\Phi}
\mathbf{Q_2}\right) \mathbf{D}
\boldsymbol{\eta_{t}}\end{bmatrix}}_{=\mathbf{0}}.
\end{multline*}
\begin{block}{Der Algorithmus}\centering
QZ-Faktorisierung: $\mathbf{A} = \mathbf{Q}' \boldsymbol{\Lambda} \mathbf{Z}'
, \mathbf{B} = \mathbf{Q}' \boldsymbol{\Omega} \mathbf{Z}'$,
$\mathbf{x_{t}}=\mathbf{Z}
\begin{pmatrix}\mathbf{z_{1,t}}\\\mathbf{z_{2,t}}\end{pmatrix}$,
\begin{align*}
\begin{split}
  \mathbf{z_{1,t} } =& - \boldsymbol{\Lambda_{11}}^{-1}\left(\boldsymbol{\Lambda_{12}}-\boldsymbol{\Phi} \boldsymbol{\Lambda_{22}}\right) \mathbf{z_{2,t}}
  + \boldsymbol{\Lambda_{11}}^{-1}
  \boldsymbol{\Omega_{11}}\mathbf{z_{1,t-1}} \\
  &+\boldsymbol{\Lambda_{11}}^{-1}\left(\boldsymbol{\Omega_{12}}-\boldsymbol{\Phi}
  \boldsymbol{\Omega_{22}}\right)\mathbf{z_{2,t-1}} +
  \boldsymbol{\Lambda_{11}}^{-1} \left(\mathbf{Q_1} - \boldsymbol{\Phi}
  \mathbf{Q_2}\right)\left(\mathbf{E}+\mathbf{C}
  \boldsymbol{\upsilon_t}\right),
\end{split}\\
   \mathbf{ z_{2,t}} =& \left(\boldsymbol{\Lambda_{22}}-\boldsymbol{\Omega_{22}}\right)^{-1}\mathbf{Q_2} \mathbf{E}.
\end{align*}
\end{block}
\end{frame}

\subsection{Aufgabe 1: Fortsetzung}
\begin{frame}\frametitle{Aufgabe 1: Fortsetzung}\framesubtitle{}
Der Sims (2001)-Algorithmus ist f\"{u}r verschiedene Programmpakete
    vorprogrammiert, sowie in Dynare implementiert.
\begin{enumerate}[(a)]
   \item Bringen Sie das linearisierte Modell in die Form $ \mathbf{A} \mathbf{x_{t+1}} =
       \mathbf{B} \mathbf{x_t} + \mathbf{C}
\boldsymbol{\upsilon_{t+1}} + \mathbf{D} \boldsymbol{\eta_{t+1}} +
\mathbf{E} $
\item Berechnen Sie die \emph{policy-function} mithilfe des Sims (2001)-
    Algorithmus, nehmen Sie daf\"{u}r folgende Parameterwerte an:
    $\beta=0.99,\quad \alpha=0.36,\quad \sigma=2,\quad \delta
    =0.025,\quad \rho=0.9$.
\end{enumerate}
\end{frame}


\end{document}



\subsection{Nichtlineare Verfahren: Perturbation (Gomme und Klein 2011)}
\begin{frame}\frametitle{Lineare L\"{o}sungsverfahren}\small
\textbf{Vorteile}:
\begin{itemize}
   \item Einfache lineare Zustandsraumdarstellung des Modells, die f\"{u}r viele Fragestellungen ausreichend exakt ist.
   \item Kalman-Filter erm\"{o}glicht es, das System empirisch zu analysieren.
\end{itemize}
\textbf{Nachteile}:
\begin{itemize}
   \item Wichtige Informationen gehen bei der Linearisierung verloren.
   \item H\"{o}here Momente spielen bei Markt- und Risikostrukturen, sowie Wohlfahrtsimplikationen eine wichtige Rolle.
   \item Schon eine Linearisierung zweiter Ordnung kann hier zu unterschiedlichen Resultaten f\"{u}hren, da die Varianz der zuk\"{u}nftigen Schocks im Erwartungswert nicht Null ist.
\end{itemize}
\begin{block}{\emph{Certainty-equivalence-property}}
In Modellen rationaler Erwartungen ber\"{u}cksichtigen die Akteure bei ihren Entscheidungen die Auswirkungen zuk\"{u}nftiger Schocks. Bei einer Linearisierung erster Ordnung sind diese im Erwartungswert gleich Null, so dass sie keine Rolle bei den Politikfunktionen spielen.
\end{block}
\end{frame}

\begin{frame}\frametitle{\secname}
Unser Fokus liegt auf Perturbations Methoden im Stil von Schmitt-Groh\'{e}/Uribe (2004) und Gomme and Klein (2011)
\begin{itemize}
\item Perturbationsansatz findet eine lokale Approximation der Politikfunktionen in der Umgebung eines bestimmten Punkts
\item Meist wird der steady-state betrachtet, da dieser analytisch oder numerisch herleitbar ist
\end{itemize}
\begin{block}{Non-stochastic steady-state}
\begin{eqnarray*}
f \left(\bar{x},\bar{y},\bar{x},\bar{y}|\theta \right)=0, ~\bar{x}=h(\bar{x},0|\theta),~ \bar{y}=g(\bar{x},0|\theta),~\bar{d} = D \bar{y}
\end{eqnarray*}
\end{block}
\begin{block}{Perturbation approach}
\begin{itemize}
  \item Problem: Analytical derivation of $g$ and $h$ most of the times impossible
  \item Solution: Approximate policy functions using a perturbation approach
  \item Idea: introduce perturbation parameter $\sigma$ that captures stochastic nature of the model
  \begin{itemize}
    \item We know $f$ and the steady-state analytically
    \item Thus, we know $g$ and $h$ at the non-stochastic steady-state $(\sigma=0)$
    \item How do the functions behave, if we switch on stochastics $(\sigma>0)$
    \item[$\rightarrow$] Taylor-approximation of $g$ and $h$ around the steady-state!
  \end{itemize}
\end{itemize}
\end{block}
\end{frame}

\begin{frame}\frametitle{\secname}
\begin{block}{General perturbation framework (Schmitt-Groh\'{e}/Uribe (2004))}
\begin{align*}
   0 &=   E_t f \left( x_{t+1},y_{t+1},x_t,y_t|\theta \right),\\
 x_{t+1}  &= h(x_{t},\sigma|\theta) + \sigma \eta_x \varepsilon_{t+1},\\
   y_t &= g(x_t,\sigma|\theta) ,\\
   d_t &= D y_t+  \eta_d \varepsilon_{t}.
\end{align*}
with $E(\varepsilon_t)=0$, $E(\varepsilon_t\varepsilon_t')=\Sigma_\varepsilon$
\end{block}
\end{frame}
\end{document}

\begin{frame}
\frametitle{\secname}
\begin{block}{Perturbation approach}
\begin{itemize}
  \item introduce perturbation parameter $\sigma$ that captures stochastic nature of the model
  \item solution is characterized by \emph{policy-functions}, $g$ and $h$, that solve (at least approximately) the system of equations $f$.
\end{itemize}
\end{block}

\begin{block}{General DSGE model, perturbation approach}
\begin{align*}
   0 &=   E_t f \left( x_{t+1},y_{t+1},x_t,y_t|\theta \right),\\
 x_{t+1}  &= h(x_{t},\sigma|\theta) + \sigma \varepsilon_{t+1},\\
   y_t &= g(x_t,\sigma|\theta) ,\\
   d_t &= D y_t+  \mu_{t}.
\end{align*}
with $E(\varepsilon_t)=0$, $E(\varepsilon_t\varepsilon_t')=\Sigma_\varepsilon$ and $E(\mu_t)=0$, $E(\mu_t\mu_t')=\Sigma_\mu$.
\end{block}

\begin{block}{Non-stochastic steady-state}
\begin{eqnarray*}
&\overline{z}:=(\bar{x}',\bar{y}')',~ \bar{x}=h(\bar{x},0|\theta),~ \bar{y}=g(\bar{x},0|\theta),~\bar{d} = D \bar{y}
\end{eqnarray*}
\end{block}
\end{frame}

\begin{frame}
\frametitle{Exakte Zustandsraumdarstellung}\framesubtitle{}
\begin{align*}
  x_{t+1} &= h(x_t,\sigma) + \sigma \eta \varepsilon_{t+1}\\
  y_t &= g(x_t,\sigma)
\end{align*}
mit $E(u_t) = 0$, $E(u_t u_s') = \begin{cases} \Sigma_u = \eta \eta' & \text{f\"{u}r } s=t,\\
                                               0    & \text{sonst}
                                 \end{cases}$
\end{frame}



Non-stochastic steady-state is given by $\overline{z}:=(\bar{x}',\bar{y}')'$
\begin{align*}
    f(\bar{x},\bar{y},\bar{x},\bar{y}|\theta)&:=f(\bar{z}|\theta)=0,&
        \bar{y}&=g(\bar{x},0|\theta),&
    \bar{d} &= D \bar{y},&
\bar{x}&=h(\bar{x},0|\theta).
\end{align*}
\end{frame}



\subsection{First-order approximation}
\begin{frame}
\frametitle{\secname}\framesubtitle{\subsecname}
\begin{itemize}
\item We are looking for approximations to $g$ and $h$ around the point $(x_t,\sigma)=(\bar{x},0)$ of the form
\begin{align*}
  g(x_t,\sigma) &= g(\overline{x},0) + g_x(\overline{x},0)(x_t-\overline{x}) + g_\sigma(\overline{x},0)(\sigma-0)\\
  h(x_t,\sigma) &= h(\overline{x},0) + h_x(\overline{x},0)(x_t-\overline{x}) + h_\sigma(\overline{x},0)(\sigma-0)
\end{align*}
\item We would like to know $g_x,h_x,g_\sigma,h_\sigma$.
\item Substitute the solution into the model:
\begin{align*}
F(x_t,\sigma):=E_t f \left( \underbrace{h(x_{t},\sigma)+\sigma \varepsilon_{t+1}}_{x_{t+1}},\underbrace{g[\overbrace{h(x_{t},\sigma)+\sigma\varepsilon_{t+1}}^{x_{t+1}},\sigma]}_{y_{t+1}},\underbrace{x_t}_{x_t},\underbrace{g(x_t,\sigma)}_{y_t} \right)
\end{align*}
\item Insight: $f$ as well as all derivatives of $f$ are $0$ when evaluated at the non-stochastic steady state $(\overline{x},0)$.
\end{itemize}
\end{frame}

\begin{frame}
\frametitle{\secname}\framesubtitle{\subsecname, computing $g_\sigma,h_\sigma$}
\begin{itemize}
  \item Take derivative of $F$ w.r.t. $\sigma$ and evaluate at the non-stochastic steady-state
  \footnotesize\begin{eqnarray*}
&F_\sigma(\overline{x},0) = E_t f_1[h_\sigma + \varepsilon_{t+1}] + E_t f_2 [g_x(h_\sigma+ \varepsilon_{t+1})+g_\sigma] + f_3\cdot 0 + f_4 g_\sigma = 0\\
 & \Leftrightarrow \begin{bmatrix} \underset{(n \times n_x)}{f_1 + f_2 g_x} & \underset{(n\times n_y)}{f_2 +f_4}\end{bmatrix} \begin{bmatrix} \underset{(n_x \times 1)}{h_\sigma} \\ \underset{(n_y \times 1)}{g_\sigma} \end{bmatrix} = \underset{n \times 1}{0}
  \end{eqnarray*}
with $f_1=\partial f(\overline{z})/\partial x_{t+1}, f_2=\partial f(\overline{z})/\partial y_{t+1}, f_3=\partial f(\overline{z})/\partial x_{t}, f_4=\partial f(\overline{z})/\partial y_{t}$\normalsize
  \item We have n equations in n unknowns
  \item Notice that this is a linear and homogenous system, that is, if there is a unique solution, it must be $$h_\sigma=\underset{n_x\times1}{0} \text{ and } g_\sigma=\underset{n_y\times1}{0}$$
\end{itemize}
\end{frame}

\begin{frame}
\frametitle{\secname}\framesubtitle{\subsecname, Certainty-equivalence}
Important theoretical result:
\begin{itemize}
  \item Even though agents take the effect of future shocks into account when optimizing, in a linearization to the first-order they don't matter for the decision rule.
\begin{block}{\emph{Certainty-equivalence-property}}
       \begin{itemize}
         \item In a first-order approximation the constant term needs not to be corrected for uncertainty (i.e. variance of shocks)
         \item Expectation of $x_t$ and $y_t$ is equal to its non-stochastic steady-state
       \end{itemize}
\end{block}
\item This is problematic when uncertainty does matter: risk premia, welfare evaluation, \dots
\end{itemize}
\end{frame}


\begin{frame}
\frametitle{\secname}\framesubtitle{\subsecname, computing $g_x,h_x$}\footnotesize
\begin{itemize}
  \item Take derivative of $F$ w.r.t. $x_t$ and evaluate at the non-stochastic steady-state
  \begin{eqnarray*}
&    F_x (\overline{x},0) = f_1 h_x + f_2 g_x h_x + f_3 + f_4 g_x = 0\\
& \underbrace{- \begin{bmatrix} \underset{(n\times n_x)}{f_1} & \underset{(n\times n_y)}{f_2} \end{bmatrix}}_{:=A} \begin{bmatrix} \underset{(n_x\times n_x)}{h_x} \\ \underset{(n_y\times n_x)}{g_x} \underset{(n_x\times n_x)}{h_x} \end{bmatrix}  = \underbrace{\begin{bmatrix} \underset{(n\times n_x)}{f_3} & \underset{(n\times n_y)}{f_4}\end{bmatrix}}_{:=B} \begin{bmatrix} \underset{(n_x\times n_x)}{I} \\\underset{(n_y\times n_x)}{g_x} \end{bmatrix}
  \end{eqnarray*}
  \item $n\times n_x$ equations for $n\times n_x$ unknown elements of $h_x$ and $g_x$
  \item Postmultiply by $\widehat{x}_t := (x_t-\overline{x})$
  \begin{align*}
    A \begin{bmatrix} h_x \widehat{x}_t \\ g_x h_x \widehat{x}_t \end{bmatrix} = B \begin{bmatrix} \widehat{x}_t \\ g_x \widehat{x}_t \end{bmatrix}
  \end{align*}
  \item Notice that the coefficient matrices are equivalent to the first order approximation
  \begin{eqnarray*}
    AE_t\begin{bmatrix} \widehat{x}_{t+1} \\ \widehat{y}_{t+1} \end{bmatrix} = B\begin{bmatrix} \widehat{x}_t\\ \widehat{y}_t \end{bmatrix} + \begin{bmatrix} \sigma \eta_x \varepsilon_{t+1} \\ 0\end{bmatrix}
  \end{eqnarray*}
\end{itemize}
\end{frame}










\subsection{Second-order approximation, Gomme and Klein (2011)}\footnotesize
\begin{frame}
\frametitle{\secname}\framesubtitle{Second-order approximation}
For the second-order approximations to $g$ and $h$ around the point $(\overline{x},0)$ we first need some useful tool in Matrix Calculus:
\begin{block}{Magnus-Neudercker-definition of the Hessian}
\begin{align*}
  \mathcal{D }f(\bar{z}) & := \begin{pmatrix} \frac{\partial f(\overline{z})}{\partial x_{t+1}'} & \frac{\partial f(\overline{z})}{\partial y_{t+1}'} & \frac{\partial f(\overline{z})}{\partial x_{t}'} & \frac{\partial f(\overline{z})}{\partial y_{t}'}\end{pmatrix}:= \begin{pmatrix} f_1 & f_2 & f_3 & f_4  \end{pmatrix},\\
  \mathcal{H} f(\bar{z}) & := \mathcal{D} vec((\mathcal{D} f(\bar{z}))')
\end{align*}
with Jacobian $\mathcal{D}f(\overline{z})$ and Hessian $\mathcal{H}f(\overline{z})$ of $f$ evaluated at steady-state.
\end{block}
This definition simplifies the algorithm, since no tensor notation is needed and basic matrix algebra can be used. Main tool:
\begin{block}{Chain-Rule}
Let $f:R^n \mapsto R^m$ and $g:R^m \mapsto R^p$ be twice differentiable. Define $h(x)=g(f(x))$. Then letting $y=f(x)$,
\begin{align*}
\mathcal{H} h(x) = (I_p \otimes \mathcal{D}f(x))'(\mathcal{H}g(y))*\mathcal{D}f(x) + (\mathcal{D}g(y)\otimes I_n) \mathcal{H} f(x)
\end{align*}
\end{block}
\end{frame}

\begin{frame}
\frametitle{\secname}\framesubtitle{Second-order approximation}
The second-order Taylor approximation at the non-stochastic steady-state is given by
\begin{block}{Second order Taylor approximation}
\begin{align*}
    x_{t+1} &= \bar{x}+h_x(\bar{x},0)\cdot(x_t-\bar{x}) + \sigma \eta_x \varepsilon_{t+1}\\
  &+ \frac{1}{2} \left[I_{n_x}\otimes (x_t-\bar{x})'\right] h_{xx}(\bar{x},0) (x_t-\bar{x})
  + \frac{1}{2} \sigma^2 h_{\sigma\sigma}(\bar{x},0)\\
  y_{t} &= \bar{y}+g_x(\bar{x},0)\cdot(x_t-\bar{x})\\
  &+ \frac{1}{2} \left[I_{n_y}\otimes (x_t-\bar{x})'\right] g_{xx}(\bar{x},0) (x_t-\bar{x})
  + \frac{1}{2} \sigma^2 g_{\sigma\sigma}(\bar{x},0)
\end{align*}
\end{block}
\begin{itemize}
\item $g_x$ and $h_x$ are the gradients of $g$ and $h$ with respect to $x$ (using e.g. Klein's algorithm), all evaluated at $(\bar{x},0)$
\item $g_{xx},h_{xx}$ and $g_{\sigma\sigma},h_{\sigma\sigma}$ the corresponding Magnus-Neudecker-Hessians, all evaluated at $(\bar{x},0)$
\item Note: No cross terms involving $x$ and $\sigma$, because approximation is around $\sigma=0$
\end{itemize}

\end{frame}

\begin{frame}
\frametitle{\secname}\framesubtitle{Second-order approximation}
\begin{itemize}
  \item Use an algorithm to obtain $h_x(\bar{x},0)$ and $g_x(\bar{x},0)$, the coefficients of a first order approximation of the model
  \item The other matrices can be calculated by inserting the policy functions into $f$ and differentiating it twice using the chain-rule of Magnus and Neudecker (1999).
  \item Evaluating the Jacobian $\mathcal{D}f=(f_1~f_2~f_3~f_4)$ and Hessian $H$ of $f$ at the non-stochastic steady-state, and setting it to $0$ yields:
\end{itemize}
\end{frame}

\begin{frame}
\frametitle{\secname}\framesubtitle{Second-order approximation}
\begin{block}{Second order solution matrices}\footnotesize
\begin{align*}
  \begin{bmatrix}
    vec(g_{xx})\\vec(h_{xx})
  \end{bmatrix}&=-Q^{-1}vec(A), &
  \begin{bmatrix}
    h_{ss}\\g_{ss}
  \end{bmatrix} &=   -B^{-1} C
\end{align*}
\begin{eqnarray*}
 &Q = \begin{bmatrix} h_x' \otimes f_2 \otimes h_x' + I_{n_x} \otimes f_4 \otimes I_{n_x}&  I_{n_x} \otimes (f_1 \otimes I_{n_x}+f_2 g_x \otimes I_{n_x})
  \end{bmatrix}\\
 & A = (I_{n_x+n_y}\otimes M')HM,\\
 & B = \begin{bmatrix} f_1 + f_2 g_x & f_2 + f_4 \end{bmatrix}\\
 & C = f_2 trm[(I_{n_y}\otimes (\eta_x\eta_x'))g_{xx}]+ trm[(I_{n_x+n_y}\otimes N') H N (\eta_x \eta_x')]\\
  &  M=\begin{bmatrix}
    h_x\\ g_x h_x \\ I_{n_x} \\g_x
  \end{bmatrix},\quad
  N = \begin{bmatrix}
    I_{n_x}\\g_x\\ 0_{(n_x+n_y)\times n_x}
  \end{bmatrix}
\end{eqnarray*}
and $trm$ defines the matrix trace of an $nm\times n$ matrix $[Y_1'~Y_2'~\dots~Y_m']'$ as the $m\times1$ vector $[tr(Y_1)~tr(Y_2)~\dots~tr(Y_m)]'$.
\end{block}
\end{frame}

\begin{frame}\frametitle{\secname}
\begin{block}{Excercise}
Consider the An and Schorfheide (2007) model.
\begin{enumerate}
  \item Write a mod-file for the nonlinear model. What are state variables, what are control variables?
  \item Compare the solution of a first-order and a second-order approximation.
\end{enumerate}
\end{block}
\end{frame}

\section{Unconditional Moments}
\begin{frame}
\frametitle{\secname}\framesubtitle{First-order approximation}
Given stationarity and the approximated solution, we can calculate the unconditional moments
\begin{block}{Unconditional Moments for first-order approximation}\footnotesize
\begin{align*}
E(x_t) &= \overline{x}; E(y_t)= \overline{y}; E(d_t)= \overline{d} = D \overline{y}\\
  \Sigma_x &:=  E (x_{t}-\bar{x})(x_{t}-\bar{x})' = h_x \Sigma_x h_x' + \sigma^2 \eta_x E(\varepsilon_t \varepsilon_t') \eta_x'\\
  \Sigma_y &:= E (y_{t}-\bar{y})(y_{t}-\bar{y})'  = g_x \Sigma_x g_x'\\
  \Sigma_d &:= E (d_{t}-\bar{d})(d_{t}-\bar{d})'  = D \Sigma_y D' + \eta_d E(\varepsilon_t \varepsilon_t') \eta_d'\\
  \Sigma_x(t) &:= E[(x_t- \bar{x})(x_0- \bar{x})'] = (h_x)^t \Sigma_x\\
  \Sigma_y(t) &:= E[(y_t- \bar{y})(y_0- \bar{y})'] = g_x (h_x)^t \Sigma_xg_x'\\
  \Sigma_d(t) &:= E[(d_t- \bar{d})(d_0- \bar{d})'] = D g_x (h_x)^t \Sigma_xg_x' D'
\end{align*}
\end{block}
\end{frame}

\begin{frame}
\frametitle{\secname}\framesubtitle{Second-order approximation}
\begin{block}{Unconditional mean for second-order approximation}
\begin{align*}
  E(x_t) &:= \bar{x} + \left(I_{n_x}-h_x\right)^{-1} \frac{ \Lambda_x}{2},\\
  E(y_t) &:= \bar{y} +g_x (\mu_x - \bar{x})+ \frac{ \Lambda_y}{2},\\
  E(d_t) &:= D\mu_y,\\
 with \\ \Lambda_x &=  \left[ I_{n_x} \otimes vec(\Sigma_x)'\right] vec(h_{xx}') + \sigma^2 h_{\sigma\sigma},&\\
  \Lambda_y &=  \left[ I_{n_y} \otimes vec(\Sigma_x)'\right] vec(g_{xx}') + \sigma^2 g_{\sigma\sigma}.
\end{align*}
Variances and autocovariances have additional higher order terms.
\end{block}

\end{frame}









\subsection{First-order approximation}
\begin{frame}
\frametitle{\secname}\framesubtitle{\subsecname}
\begin{itemize}
\item We are looking for approximations to $g$ and $h$ around the point $(x,\sigma)=(\bar{x},0)$ of the form
\begin{align*}
  g(x,\sigma) &= g(\overline{x},0) + g_x(\overline{x},0)(x-\overline{x}) + g_\sigma(\overline{x},0)(\sigma-0)\\
  h(x,\sigma) &= h(\overline{x},0) + h_x(\overline{x},0)(x-\overline{x}) + h_\sigma(\overline{x},0)(\sigma-0)
\end{align*}
\item We would like to know $g_x,h_x,g_\sigma,h_\sigma$.
\item Idea:
\begin{itemize}
\item Substitute the solution into the model,
\begin{align*}
F(x,\sigma):=E_t f \left( \underbrace{h(x_{t},\sigma)+\sigma \varepsilon_{t+1}}_{x_{t+1}},\underbrace{g[h(x_{t},\sigma)+\sigma \varepsilon_{t+1},\sigma]}_{y_{t+1}},\underbrace{x_t}_{x_t},\underbrace{g(x_t,\sigma)}_{y_t} \right)
\end{align*}
\item Insight: $f$ as well as all derivatives of $f$ are $0$ when evaluated at the non-stochastic steady state $(\overline{x},0)$.
\end{itemize}
\end{itemize}
\end{frame}

\begin{frame}
\frametitle{\secname}\framesubtitle{\subsecname, computing $g_\sigma,h_\sigma$}
\begin{itemize}
  \item Take derivative of $F$ w.r.t. $\sigma$ and evaluate at the non-stochastic steady-state
  \footnotesize\begin{eqnarray*}
&F_\sigma(\overline{x},0) = E_t f_1[h_\sigma + \varepsilon_{t+1}] + E_t f_2 [g_x(h_\sigma+ \varepsilon_{t+1})+g_\sigma] + f_3\cdot 0 + f_4 g_\sigma = 0\\
 & \Leftrightarrow \begin{bmatrix} \underset{(n \times n_x)}{f_1 + f_2 g_x} & \underset{(n\times n_y)}{f_2 +f_4}\end{bmatrix} \begin{bmatrix} \underset{(n_x \times 1)}{h_\sigma} \\ \underset{(n_y \times 1)}{g_\sigma} \end{bmatrix} = \underset{n \times 1}{0}
  \end{eqnarray*}
with $f_1=\partial f(\overline{z})/\partial x_{t+1}, f_2=\partial f(\overline{z})/\partial y_{t+1}, f_3=\partial f(\overline{z})/\partial x_{t}, f_4=\partial f(\overline{z})/\partial y_{t}$\normalsize
  \item We have n equations in n unknowns
  \item Notice that this is a linear and homogenous system, that is, if there is a unique solution, it must be $$h_\sigma=g_\sigma=0$$
\end{itemize}
\end{frame}

\begin{frame}
\frametitle{\secname}\framesubtitle{\subsecname, Certainty-equivalence}
Important theoretical result:
\begin{itemize}
  \item Even though agents take the effect of future shocks into account when optimizing, in a linearization to the first-order they don't matter for the decision rule.
\begin{block}{\emph{Certainty-equivalence-property}}
       \begin{itemize}
         \item In a first-order approximation the constant term needs not to be corrected for uncertainty (i.e. variance of shocks)
         \item Expectation of $x_t$ and $y_t$ is equal to its non-stochastic steady-state
       \end{itemize}
\end{block}
\item This is problematic when uncertainty does matter: risk premia, welfare evaluation, \dots
\end{itemize}
\end{frame}


\begin{frame}
\frametitle{\secname}\framesubtitle{\subsecname, computing $g_x,h_x$}\footnotesize
\begin{itemize}
  \item Take derivative of $F$ w.r.t. $x_t$ and evaluate at the non-stochastic steady-state
  \begin{eqnarray*}
&    F_x (\overline{x},0) = f_1 h_x + f_2 g_x h_x + f_3 + f_4 g_x = 0\\
& \underbrace{- \begin{bmatrix} \underset{(n\times n_x)}{f_1} & \underset{(n\times n_y)}{f_2} \end{bmatrix}}_{:=A} \begin{bmatrix} \underset{(n_x\times n_x)}{h_x} \\ \underset{(n_y\times n_x)}{g_x} \underset{(n_x\times n_x)}{h_x} \end{bmatrix}  = \underbrace{\begin{bmatrix} \underset{(n\times n_x)}{f_3} & \underset{(n\times n_y)}{f_4}\end{bmatrix}}_{:=B} \begin{bmatrix} \underset{(n_x\times n_x)}{I} \\\underset{(n_y\times n_x)}{g_x} \end{bmatrix}
  \end{eqnarray*}
  \item $n\times n_x$ equations for $n\times n_x$ unknown elements of $h_x$ and $g_x$
  \item Postmultiply by $\widehat{x}_t := (x_t-\overline{x})$
  \begin{align*}
    A \begin{bmatrix} h_x \widehat{x}_t \\ g_x h_x \widehat{x}_t \end{bmatrix} = B \begin{bmatrix} \widehat{x}_t \\ g_x \widehat{x}_t \end{bmatrix}
  \end{align*}
  \item Notice that this is equivalent to the first order approximation of the model
  \begin{eqnarray*}
    A\begin{bmatrix} \widehat{x}_{t+1} \\ E_t \widehat{y}_{t+1} \end{bmatrix} = B\begin{bmatrix} \widehat{x}_t\\ \widehat{y}_t \end{bmatrix} + \begin{bmatrix} \sigma \varepsilon_{t+1} \\ 0\end{bmatrix}
  \end{eqnarray*}
\end{itemize}
\end{frame}




\subsection{Second-order approximation, Schmitt-Groh\'{e}/Uribe (2004)}
\begin{frame}
\frametitle{\secname}\framesubtitle{\subsecname}\footnotesize
The second-order approximations to $g$ and $h$ around the point $(\overline{x},0)$ are of the form
\begin{align*}
  [g(x,\sigma)]^i &= [g(\overline{x},0)]^i + [g_x(\overline{x},0)]^i_a [(x-\overline{x})]_a + [g_\sigma(\overline{x},0)]^i [\sigma-0]\\
  & + \frac{1}{2} [g_{xx}(\overline{x},0)]^i_{ab} [(x-\overline{x})]_a [(x-\overline{x})]_b\\
  & + \frac{1}{2} [g_{x\sigma}(\overline{x},0)]^i_{a} [(x-\overline{x})]_a [\sigma-0]\\
  & + \frac{1}{2} [g_{\sigma x}(\overline{x},0)]^i_{a} [(x-\overline{x})]_a [\sigma-0]\\
  & + \frac{1}{2} [g_{\sigma\sigma}(\overline{x},0)]^i [\sigma-0][\sigma-0]\\
  [h(x,\sigma)]^j &= [h(\overline{x},0)]^j + [h_x(\overline{x},0)]^j_a [(x-\overline{x})]_a + [h_\sigma(\overline{x},0)]^j [\sigma-0]\\
  & + \frac{1}{2} [h_{xx}(\overline{x},0)]^j_{ab} [(x-\overline{x})]_a [(x-\overline{x})]_b\\
  & + \frac{1}{2} [h_{x\sigma}(\overline{x},0)]^j_{a} [(x-\overline{x})]_a [\sigma-0]\\
  & + \frac{1}{2} [h_{\sigma x}(\overline{x},0)]^j_{a} [(x-\overline{x})]_a [\sigma-0]\\
  & + \frac{1}{2} [h_{\sigma\sigma}(\overline{x},0)]^j [\sigma-0][\sigma-0]
\end{align*}
where $i=1,\dots,n_y$; $a,b, = 1,\dots,n_x$; and $j=1,\dots n_x$.
\end{frame}

\begin{frame}
\frametitle{\secname}\framesubtitle{\subsecname}
This is so-called tensor notation:
\begin{itemize}
  \item $[f_2]_\alpha^i$ is the $(i,\alpha)$ element of the derivative of $f$ with respect to $y_{t+1}$. It is of dimension $n\times n_y$.
  \item $[f_2]_\alpha^i [g_x]_\beta^\alpha [h_x]_j^\beta=\sum_{\alpha=1}^{n_y}\sum_{\beta=1}^{n_x} \frac{\partial f^i}{\partial y_{t+1}^\alpha} \frac{\partial g^\alpha}{\partial x_{t}^\beta} \frac{\partial h^\beta}{\partial x_{t}^j}$
  \item $f_{11}$ is a three-dimensional array with $n$-rows, $n_y$ columns and $n_y$ pages. $[f_{11}]^i_{\alpha\gamma}$ denotes the element of $f_{11}$ located at the intersection of row $i$, column $\alpha$ and page $\gamma$.
  \item Aside note: There is also a different (prettier) way to do this using the Magnus/Neudecker definition of the Hessian, see Gomme/Klein (2011).
\end{itemize}
\end{frame}

\begin{frame}
  \frametitle{\secname}\framesubtitle{\subsecname, computing $g_{xx}$ and $h_{xx}$}\scriptsize
We can use the derivative of $F_x = f_1 h_x + f_2 g_x h_x + f_3 + f_4 g_x$ w.r.t. $x$ to identify $g_{xx}$ and $h_{xx}$:
\begin{align*}
  [F_{xx}(\overline{x},0)]^i_{jk} = &\left([f_{11}]^i_{\beta \delta} [h_x]_k^\delta + [f_{12}]^i_{\beta \gamma} [g_x]_\delta^\gamma [h_x]_k^\delta + [f_{13}]^i_{\beta k} + [f_{14}]^i_{\beta \gamma} [g_x]_k^\gamma\right)[h_x]_j^\beta\\
  &+[f_1]_\beta^i[h_{xx}]_{jk}^\beta\\
  &+\left([f_{21}]_{\alpha\delta}^i[h_x]_k^\delta + [f_{22}]_{\alpha\gamma}^i[g_x]_\delta^\gamma[h_x]_k^\delta + [f_{23}]_{\alpha k}^i + [f_{24}]_{\alpha \gamma}^i[g_x]_k^\gamma \right)[g_x]_\beta^\alpha[h_x]_j^\beta\\
  &+[f_2]_\alpha^i[g_{xx}]_{\beta\delta}^\alpha [h_x]_k^\delta [h_x]_j^\beta\\
  &+[f_2]_\alpha^i[g_x]_\beta^\alpha[h_{xx}]_{jk}^\beta\\
  &+[f_{31}]_{j\delta}^i[h_x]_k^\delta + [f_{32}]_{j\gamma}^i[g_x]_\delta^\gamma[h_x]_k^\delta + [f_{33}]_{jk}^i + [f_{34}]_{j\gamma}^i[g_x]_k^\gamma\\
  &+\left([f_{41}]_{\alpha\delta}^i[h_x]_k^\delta +[f_{42}]_{\alpha\gamma}^i[g_x]_\delta^\gamma[h_x]_k^\delta + [f_{43}]_{\alpha k}^i + [f_{44}]_{\alpha \gamma}^i [g_x]_k^\gamma \right)[g_x]_j^\alpha\\
  &+[f_4]_\alpha^i[g_{xx}]_{jk}^\alpha\\
  &= 0, \qquad i=1,\dots n,\qquad j,k,\beta,\delta = 1,\dots n_x, \qquad \alpha,\gamma=1,\dots n_y
\end{align*}
This is a linear system of $n\times n_x\times n_x$ equations in $n\times n_x\times n_x$ unknowns given by the elements of $g_{xx}$ and $h_{xx}$ $\Rightarrow$ easy to solve using matrix algebra (and a computer).
\end{frame}

\begin{frame}
  \frametitle{\secname}\framesubtitle{\subsecname, computing $g_{xx}$ and $h_{xx}$}
Similarly, we can use the derivative of $$F_\sigma = E_t f_1[h_\sigma + \varepsilon_{t+1}] + E_t f_2 [g_x(h_\sigma+\sigma \varepsilon_{t+1})+g_\sigma] + f_4 g_\sigma = 0$$ w.r.t. $\sigma$ to identify $g_{\sigma\sigma}$ and $h_{\sigma\sigma}$, we omit the expression here.\\ $\Rightarrow$ This yields a system of $n$ linear equations in the $n$ unknowns given by the elements of $g_{\sigma\sigma}$ and $h_{\sigma\sigma}$.
\end{frame}

\begin{frame}
\frametitle{\secname}\framesubtitle{\subsecname, computing $g_{\sigma x}$ and $h_{\sigma x}$}
Let's have a look at the cross-derivatives, i.e. $F_{\sigma,x}(\overline{x},0)=0$ using the fact that $g_\sigma=h_\sigma=0$:
\begin{align*}
  [F_{\sigma_x}]^i_j = [f_1]^i_\beta[h_{\sigma x}]_j^\beta + [f_{2}]_\alpha^i[g_x]_\beta^\alpha[h_{\sigma x}]_j^\beta + [f_2]_\alpha^i [g_{\sigma x}]_\gamma^\alpha [h_x]_j^\gamma + [f_4]_\alpha^i[g_{\sigma x}]_j^\alpha = 0
\end{align*}
$i=1,\dots, n; \alpha=1,\dots, n_y; \beta,\gamma,j=1,\dots,n_x$\\~\\
This is a system of $n\times n_x$ equations in $n\times n_x$ unknowns given by the elements of $g_{\sigma x}$ and $h_{\sigma x}$. Obviously, this is a linear \textbf{homogenous} system of equations, thus a unique solution implies $g_{\sigma x}=h_{\sigma x}=0$
\end{frame}
\subsection{Unconditional moments}
\begin{frame}
\frametitle{\secname}\framesubtitle{\subsecname}
\begin{itemize}
  \item Given stationary and the approximated solution, we can calculate the unconditional moments.
  \item Note: we focus on second order accurate moments, that is to keep only first and second order terms in the formula to compute moments.
  \item Compute variance from first order approximation alone, since all other cross-product terms result in something of order higher than two.
\end{itemize}
\begin{block}{Unconditional autocovariogram}\footnotesize
\begin{align*}
  \Sigma_x :=  E (x_{t}-\bar{x})(x_{t}-\bar{x})' &= h_x \Sigma_x h_x' + \sigma^2 \Sigma_\varepsilon\\
  \Sigma_y := E (y_{t}-\bar{y})(y_{t}-\bar{y})' & = g_x \Sigma_x g_x'\\
  \Sigma_d := E (d_{t}-\bar{d})(d_{t}-\bar{d})' & = D \Sigma_y D' + \Sigma_\mu\\
  \Sigma_x(t) := E[(x_t- \bar{x})(x_0- \bar{x})'] &= (h_x)^t \Sigma_x\\
  \Sigma_y(t) := E[(y_t- \bar{y})(y_0- \bar{y})'] &= g_x (h_x)^t \Sigma_xg_x'\\
  \Sigma_d(t) := E[(d_t- \bar{d})(d_0- \bar{d})'] &= D g_x (h_x)^t \Sigma_xg_x' D'
\end{align*}
\end{block}
\end{frame}

\begin{frame}
\frametitle{\secname}\framesubtitle{\subsecname}\footnotesize
Regarding the mean, we have
\begin{align*}
  x_{t+1}^i \approx \overline{x}^i + h_{x}^i (x_t - \overline{x}) + \frac{1}{2} \underbrace{(x_t - \overline{x})' h_{xx}^i (x_t - \overline{x})}_{(x_t-\bar{x})'\otimes(x_t-\bar{x})' vec(h_{xx}^i)}  + \frac{1}{2} h_{\sigma\sigma}^i + \sigma \varepsilon_{t+1}
\end{align*}

Notice that $E\left[(x_t-\bar{x})'\otimes(x_t-\bar{x})'\right]=vec(\Sigma_x)'$
\begin{block}{Unconditional mean}\footnotesize
\begin{align*}
  \mu_x &:= E (x_t) = \bar{x} + \left(I_{n_x}-h_x\right)^{-1} \frac{ \Lambda_x}{2},\quad
  \mu_y :=   E (y_t) = \bar{y} +g_x (\mu_x - \bar{x})+ \frac{ \Lambda_y}{2},\\
 with \\ \Lambda_x &=  \begin{bmatrix} vec(h_{xx}^1)'\\vec(h_{xx}^2)'\\ \dots \\ vec(h_{xx}^{n_x})'\end{bmatrix}  vec(\Sigma_x)  + \sigma^2 h_{\sigma\sigma},\quad
  \Lambda_y =  \begin{bmatrix} vec(g_{xx}^1)'\\vec(g_{xx}^2)'\\ \dots \\ vec(g_{xx}^{n_y})'\end{bmatrix} vec(\Sigma_x) + \sigma^2 g_{\sigma\sigma}.
\end{align*}
\end{block}
First order approximation: $\Lambda_x=\Lambda_y=0 \Rightarrow \mu_x = \bar{x}, \mu_y = \bar{y}$.
\end{frame}

\section{Summary}
\begin{frame}
\frametitle{\secname}
\begin{itemize}
  \item In a first-order approximation the unconditional mean is equal to the steady-state.
  \item Up to second order, the presence of uncertainty affects only the constant term of the decision rules
  \item[$\Rightarrow$] Unconditional mean can be significantly different from non-stochastic steady-state!
\end{itemize}
\end{frame}


\section{Further topic: Indeterminacy}
\begin{frame}
\frametitle{\secname}\framesubtitle{Lubik \& Schorfheide (2004)}
\begin{itemize}
    \item Rationale Erwartungen: $\boldsymbol{\eta_{t+1}} =
        E_t\mathbf{x_{t+1}} - \mathbf{x_{t+1}}$.
    \item Nicht-prognostizierbarer Erwartungsfehler
        $\boldsymbol{\eta_{t+1}}$ tritt aufgrund der Realisation der
        strukturellen Schocks auf:
        $\boldsymbol{\eta_t}=f(\boldsymbol{\upsilon_t})$.

  \item Consider a different log-linear (first-order) canonical form of the DSGE model (e.g. Sims (2002))
  \begin{align*}
    \Gamma_0(\theta) z_t = \Gamma_1(\theta) z_{t-1} + \Psi(\theta) \varepsilon_t + \Pi(\theta)\eta_t
  \end{align*}
  \item $z_t$ are all model variables, $\varepsilon_t$ are model shocks and $\eta_t=\widetilde{x}_t-E_{t-1}[\widetilde{x}_{t-1}]$ are expectational errors for the non-predetermined variables $\widetilde{x}$.
  \item $\Gamma_0,\Gamma_1, \Psi, \Pi$ are parameter matrices dependent on deep parameters $\theta$.
\end{itemize}
\end{frame}

\begin{frame}
\frametitle{\secname}\framesubtitle{Lubik \& Schorfheide (2004)}
\begin{itemize}
  \item Assume that there is a sunspot shock $\zeta_t$, i.e. the expectational errors are driven by
  \begin{align*}
    \eta_t = A_1 \varepsilon_t + A_2 \zeta_t
  \end{align*}
  \item Solution algorithms construct a mapping from the shocks to the expectation errors:
  \begin{enumerate}
    \item nonexistence of a stable solution
    \item existence of a unique stable solution (determinacy), i.e. $A_1=A_1(\theta)$ and $A_2=0$
    \item existence of multiple stable solutions (indeterminacy), i.e. $A_1$ is not uniquely determined by $\theta$ and $A_2\neq0$.
  \end{enumerate}
 \item Loosely speaking: there exists a unique stable solution, if the Taylor principle, that is the central bank raises the real interest rate in response to inflation more than one-to-one, and has multiple stable solutions otherwise (interesting in its own right)
\end{itemize}
\end{frame}







\begin{frame}
\frametitle{\secname}\framesubtitle{Solution}
We follow {Gomme.2011}'s approach to approximate the policy functions using:
\begin{block}{{Magnus.1999} definition of the Hessian}
\begin{align*}
  \mathcal{D }f(\bar{z}) &:= \begin{pmatrix} f_1 & f_2 & f_3 & f_4  \end{pmatrix} := \begin{pmatrix} \frac{\partial f(\overline{z})}{\partial x_{t+1}'} & \frac{\partial f(\overline{z})}{\partial y_{t+1}'} & \frac{\partial f(\overline{z})}{\partial x_{t}'} & \frac{\partial f(\overline{z})}{\partial y_{t}'}\end{pmatrix},\\
  \mathcal{H} f(\bar{z}) & := \mathcal{D} vec((\mathcal{D} f(\bar{z}))')
\end{align*}
with Jacobian $\mathcal{D}f(\overline{z})$ and Hessian $\mathcal{H}f(\overline{z})$ of $f$ evaluated at steady-state.
\end{block}
This definition simplifies the analytical derivatives, since no tensor notation is needed and basic matrix algebra can be used.
\end{frame}

\begin{frame}
\frametitle{\secname}\framesubtitle{Solution}
The second-order Taylor approximation at the non-stochastic steady-state is then given by
\begin{block}{Taylor approximation}
\begin{align*}
    x_{t+1} &= \bar{x}+h_x(\bar{x},0)\cdot(x_t-\bar{x}) + \sigma \eta_x \varepsilon_{t+1}\\
  &+ \frac{1}{2} \left[I_{n_x}\otimes (x_t-\bar{x})' \otimes (x_t-\bar{x})' \right] vec(h_{xx}(\bar{x},0)')
  + \frac{1}{2} \sigma^2 h_{\sigma\sigma}(\bar{x},0)\\
  y_{t} &= \bar{y}+g_x(\bar{x},0)\cdot(x_t-\bar{x})\\
  &+ \frac{1}{2} \left[I_{n_y}\otimes (x_t-\bar{x})' \otimes (x_t-\bar{x})' \right] vec(g_{xx}(\bar{x},0)')
  + \frac{1}{2} \sigma^2 g_{\sigma\sigma}(\bar{x},0)
\end{align*}
\end{block}
$g_x$ and $h_x$ are the gradients of $g$ and $h$ with respect to $x_t$, $g_{xx},h_{xx}$ and $g_{\sigma\sigma},h_{\sigma\sigma}$ the corresponding Hessians, all evaluated at $(\bar{x},0)$.
\end{frame}

\begin{frame}
\frametitle{\secname}\framesubtitle{Solution}
\begin{itemize}
  \item We follow {Klein.2000} to obtain $h_x(\bar{x},0)$ and $g_x(\bar{x},0)$, the coefficients of a first order approximation of the model.
  \item The other matrices can be calculated by inserting the policy functions into $f$:
  \begin{align*}
E_t f \left( h(x_{t},\sigma)+\sigma \varepsilon_{t+1},g(h(x_{t},\sigma)+\sigma \varepsilon_{t+1},\sigma),x_t,g(x_t,\sigma) \right)= 0.
\end{align*}
\item Idea: $f$ as well as all derivatives of $f$ are $0$ when evaluated at the steady state $(\overline{x},0)$.
\item Differentiating $f$ twice using the chain-rule of {Magnus.1999}, evaluating the Jacobian $\mathcal{D}f=(f_1~f_2~f_3~f_4)$ and Hessian $H$ of $f$ at the non-stochastic steady-state, and setting it to $0$ yields:
\end{itemize}
\end{frame}

\begin{frame}
\frametitle{\secname}\framesubtitle{Solution}
\begin{block}{Second order solution matrices}\footnotesize
\begin{align*}
  \begin{bmatrix}
    vec(g_{xx})\\vec(h_{xx})
  \end{bmatrix}&=-Q^{-1}vec(A), &
  \begin{bmatrix}
    h_{ss}\\g_{ss}
  \end{bmatrix} &=   -B^{-1} C
\end{align*}
\begin{eqnarray*}
 &Q = \begin{bmatrix} h_x' \otimes f_2 \otimes h_x' + I_{n_x} \otimes f_4 \otimes I_{n_x}&  I_{n_x} \otimes (f_1 \otimes I_{n_x}+f_2 g_x \otimes I_{n_x})
  \end{bmatrix}\\
 & A = (I_{n_x+n_y}\otimes M')HM,\\
 & B = \begin{bmatrix} f_1 + f_2 g_x & f_2 + f_4 \end{bmatrix}\\
 & C = f_2 trm[(I_{n_y}\otimes (\eta_x\eta_x'))g_{xx}]+ trm[(I_{n_x+n_y}\otimes N') H N (\eta_x \eta_x')]\\
  &  M=\begin{bmatrix}
    h_x\\ g_x h_x \\ I_{n_x} \\g_x
  \end{bmatrix},\quad
  N = \begin{bmatrix}
    I_{n_x}\\g_x\\ 0_{(n_x+n_y)\times n_x}
  \end{bmatrix}
\end{eqnarray*}
and $trm$ defines the matrix trace of an $nm\times n$ matrix $[Y_1'~Y_2'~\dots~Y_m']'$ as the $m\times1$ vector $[tr(Y_1)~tr(Y_2)~\dots~tr(Y_m)]'$.
\end{block}
\end{frame}

\begin{frame}
\frametitle{\secname}\framesubtitle{Unconditional Moments}
\begin{itemize}
  \item Given stationary and the approximated solution, we can calculate the unconditional moments.
  \item Note: we focus on second order accurate moments, that is to keep only first and second order terms in the formula to compute moments.
  \item Compute variance from first order approximation alone, since all other cross-product terms result in something of order higher than two.
\end{itemize}
\begin{block}{Unconditional autocovariogram}\footnotesize
\begin{align*}
  \Sigma_x :=  E (x_{t}-\bar{x})(x_{t}-\bar{x})' &= h_x \Sigma_x h_x' + \sigma^2 \eta_x \eta_x'\\
  \Sigma_y := E (y_{t}-\bar{y})(y_{t}-\bar{y})' & = g_x \Sigma_x g_x'\\
  \Sigma_d := E (d_{t}-\bar{d})(d_{t}-\bar{d})' & = D \Sigma_y D' + \eta_y \eta_y'\\
  \Sigma_x(t) := E[(x_t- \bar{x})(x_0- \bar{x})'] &= (h_x)^t \Sigma_x\\
  \Sigma_y(t) := E[(y_t- \bar{y})(y_0- \bar{y})'] &= g_x (h_x)^t \Sigma_xg_x'\\
  \Sigma_d(t) := E[(d_t- \bar{d})(d_0- \bar{d})'] &= D g_x (h_x)^t \Sigma_xg_x' D'
\end{align*}
\end{block}
\end{frame}

\begin{frame}
\frametitle{\secname}\framesubtitle{Unconditional Moments}
Regarding the mean, notice that $E\left[(x_t-\bar{x})'\otimes(x_t-\bar{x})'\right]=vec(\Sigma_x)'$
\begin{block}{Unconditional mean}\footnotesize
\begin{align*}
  \mu_x &:= E (x_t) = \bar{x} + \left(I_{n_x}-h_x\right)^{-1} \frac{ \Lambda_x}{2},\\
  \mu_y &:=   E (y_t) = \bar{y} +g_x (\mu_x - \bar{x})+ \frac{ \Lambda_y}{2},\\
  \mu_d &:= E(d_t) = D\mu_y,\\
 with \\ \Lambda_x &=  \left[ I_{n_x} \otimes vec(\Sigma_x)'\right] vec(h_{xx}') + \sigma^2 h_{\sigma\sigma},&\\
  \Lambda_y &=  \left[ I_{n_y} \otimes vec(\Sigma_x)'\right] vec(g_{xx}') + \sigma^2 g_{\sigma\sigma}.
\end{align*}
\end{block}
First order approximation: $\Lambda_x=\Lambda_y=0 \Rightarrow \mu_x = \bar{x}, \mu_y = \bar{y}$ and $\mu_d = D\bar{y}$.
\end{frame}



\subsection{An and Schorfheide (2007) model}
\begin{frame}\frametitle{Identification of DSGE-models}
  \tableofcontents[currentsection]
\end{frame}

\begin{frame}[shrink]
\frametitle{\secname: \subsecname}
\scriptsize\begin{gather*}
0 =1- e^{-\tau E_tc_{t+1} +\tau c_t+R_t - \rho_z z_{t} - E_t\pi_{t+1}},\\
0  = \frac{1-\nu}{\nu \phi \pi^{*2}} \left(e^{\tau c_{t}}-1\right) -
\left(e^{\pi_t}-1\right) \left[\left(1-\frac{1}{2\nu}\right)e^{\pi_t} + \frac{1}{2\nu}\right]
+ \beta \left(e^{E_t\pi_{t+1}}-1\right)e^{-\tau E_t c_{t+1} + \tau c_t + E_t dy_{t+1} + E_t\pi_{t+1}},\\
0 = e^{c_t-y_t} - e^{-g_t} +\frac{\phi \pi^{*2} g^*}{2} \left(e^{\pi_t}-1\right)^2,\\
0 = R_t -\rho_R R_{t-1} - (1-\rho_R)\psi_1 \pi_t - (1-\rho_R)\psi_2 \left(y_t-g_t\right) - e_{R,t},\\
0 = dy_t - y_t+y_{t-1},\\
0 = g_t - \rho_g g_{t-1} + e_{g,t},\\
0 = z_t - \rho_z z_{t-1} + e_{z,t},\\
0 = YGR_t - \gamma^{(Q)} - 100(dy_t+\widehat{z}_t),\\
0 = INFL_t - \pi^{(A)} - 400 \pi_t,\\
0 = INT_t - \pi^{(A)} - r^{(A)} - 4 \gamma^{(Q)} - 400 R_t,\\
0 = E_t e_{r,t+1},\\
0 = E_t e_{g,t+1},\\
0 = E_t e_{z,t+1}.
\end{gather*}
given the auxiliary variables:
\begin{align*}
  \kappa = \frac{\tau (1-\nu)}{\nu \pi^{*2}\phi}, \qquad
  \beta = \left(1+\frac{r^{(A)}}{400}\right)^{-1}, \qquad \pi^* = 1+\frac{\pi^{(A)}}{400}, \qquad \phi = \frac{\tau(1-\nu)}{\kappa \nu \pi^{*2}}, \qquad g^* = (c/y)^*.
\end{align*}
\end{frame}

\begin{frame}
\frametitle{\secname: \subsecname}
\footnotesize\begin{itemize}
\item $x_t = (y_{t-1},R_{t-1},g_{t-1},z_{t-1},e_{R,t},e_{g,t},e_{z,t})'$, $y_t = (c_t, dy_t, \pi_t, YGR_t, INFL_t, INT_t)'$ $d_t = (YGR_t, INFL_t, INT_t)'$, $\eta_v=(0~0~0)'$, $\sigma=\sigma_z$, $$\eta_u = \begin{pmatrix}0&0&0\\0&0&0\\0&0&0\\0&0&0\\\sigma_R/\sigma_z&0&0\\ 0&\sigma_g/\sigma_z&0\\0&0&1\end{pmatrix}, \qquad D = \begin{pmatrix} 0&0&0&1&0&0\\0&0&0&0&1&0\\0&0&0&0&0&1\end{pmatrix}.$$
\item The steady-state of this model is given by
\begin{gather*}
y=R=g=z=e_R=e_g=e_z=c=dy=\pi=0\\
YGR = \gamma^{(Q)}, \quad INFL = \pi^{(A)}, \quad INT = \pi^{(A)} + r^{(A)} +4 \gamma^{(Q)}
\end{gather*}
\item Identification of the parameter vector $$(\tau, ~ \kappa, ~ \psi_1, ~ \psi_2, ~ \rho_R,~ \rho_g,~\rho_z,~r^{(A)},~\pi^{(A)},~\gamma^{(Q)},~\sigma_R,~\sigma_g~,\sigma_z,~\nu,(c/y)^*)'$$ at the local point $$(2,~0.33,~1.5,~0.125,~0.75,~0.95,~0.9,~1,~3.2,~0.55,~0.002,~0.006,~0.003,~0.1,~0.85)'.$$
\end{itemize}
\end{frame}
