\documentclass[10pt]{beamer}  %%% F\"{U}R HANDOUT ALS PDF
\usetheme{metropolis}
\usepackage{iftex}
\ifPDFTeX
    \usepackage[T1]{fontenc}
    \usepackage[utf8]{inputenc}
\fi
\ifXeTeX
\fi
\ifLuaTeX
\fi
\usepackage[ngerman]{babel}
\usepackage{amsmath,amsfonts,amssymb}
\usepackage{cancel}
\usepackage[]{hyperref}
\usepackage{xcolor,pstricks,beamerprosper}
\title{DSGE Mini Kurs - Teil 1 (Einf\"{u}hrung)}
\author{Willi Mutschler}
\date{Dezember 2015}


\begin{document}

\begin{frame}
\titlepage
\end{frame}

\begin{frame}\frametitle{Inhaltsverzeichnis}
\setbeamertemplate{section in toc}[sections numbered]
    \tableofcontents
\end{frame}

\section{Was sind Dynamic Stochastic General Equilibrium Modelle?}\small
\begin{frame}
\frametitle{Was sind Dynamic Stochastic General Equilibrium Modelle?}
\begin{itemize}
\item Ansammlung von Gleichungen, die eine \"{O}konomie beschreiben
\item DSGE Modelle sind mikrofundiert:
\begin{itemize}\footnotesize
\item Akteure bilden Erwartungen und optimieren ihr Verhalten \textbf{vorausschauend}
\item antizipieren und reagieren auf Politik\"{a}nderungen
\end{itemize}
\item \textbf{General Equilibrium}: Gleichungen m\"{u}ssen IMMER erf\"{u}llt sein
\begin{itemize}\footnotesize
\item Kurzfristiges Gleichgewicht: Entscheidungen, Mengen und Preise passen sich so an, dass Gleichungen immer erf\"{u}llt sind
\item Langfristiges Gleichgewicht: Zustand mit Beharrungsverm\"{o}gen (steady-state)
    \begin{itemize}\scriptsize
    \item z.B. Wachstumsgleichgewicht: Gleichgewichtsgr\"{o}{\ss}en ver\"{a}ndern sich im Zeitablauf, aber Rate der Ver\"{a}nderung ist im Zeitablauf stabil
    \end{itemize}
\end{itemize}
\item \textbf{Stochastic}: Schocks k\"{o}nnen daf\"{u}r sorgen, dass das Gleichgewicht vom steady-state tempor\"{a}r abweicht und somit Konjunkturzyklen, oder allgemeiner einen Daten-generierenden Prozess beschreiben
\item \textbf{Dynamic}: Akteure, die intertemporal optimieren. System erlaubt es uns zudem die dynamischen Effekte
    \begin{itemize}\footnotesize
    \item nach einem tempor\"{a}ren Schock zum alten steady-state oder
    \item nach einer permanenten \"{A}nderungen zum neuen steady-state
    \end{itemize}
    zu analysieren und quantifizieren
\end{itemize}
\end{frame}

\section{Wof\"{u}r sind DSGE Modelle n\"{u}tzlich?}
\begin{frame}
\frametitle{Wof\"{u}r sind DSGE Modelle n\"{u}tzlich?}
\begin{itemize}
\item Analyse von
\begin{itemize}
\item dynamischen Verhaltens auf tempor\"{a}re oder permanente \"{A}nderungen
\item Konjunkturzyklen
\item verschiedenen kurzfristiger sowie langfristiger Politikma{\ss}nahmen
\item Wohlfahrtswirkungen (z.B. verschiedener Politikma{\ss}nahmen)
\end{itemize}
\item Kontrafaktische Analyse: Was w\"{a}re wenn, \"{o}konomische Struktur oder Parameter oder.... anders gewesen w\"{a}re?
\item Prognosen (wenn man DSGE Modelle sch\"{a}tzt)
\item Berechnung von Multiplikatoren und anderen Kennzahlen
\end{itemize}
\end{frame}

\section{Grunds\"{a}tzliche Struktur von DSGE Modellen}
\begin{frame}
\frametitle{Grunds\"{a}tzliche Struktur von DSGE Modellen (I)}
\begin{itemize}
\item \"{O}konomie wird in Sektoren eingeteilt, typischerweise
\begin{itemize}
\item Haushaltssektor
\item Unternehmenssektor
\item Regierungssektor (Geld- und Fiskalpolitik)monetary and fiscal policy), if applicable
\item weitere Sektoren: Banken, Immobilien, internationale Verflechtungen,...
\end{itemize}
\item Grob zwei Klassen von DSGE Modellen
\begin{itemize}
\item ohne Preisfriktionen $\hookrightarrow$ Real-Business-Cycle (RBC) Modelle
\item mit Marktunvollkommenheiten und Preisfriktionen $\hookrightarrow$ New-Keynesian Modelle
\end{itemize}
\end{itemize}
\end{frame}


\begin{frame}
\frametitle{Grunds\"{a}tzliche Struktur von DSGE Modellen (II)}\footnotesize
Die Sektoren folgen \"{u}blicherweise folgendem Kalk\"{u}l
\begin{itemize}
  \item Haushalte erhalten Nutzen aus privatem Konsum und staatlichen Leistungen; sie unterliegen dabei einer Budget Restriktion, in der sie ihre Ausgaben \"{u}ber (nutzenschm\"{a}hlernde) Arbeit, Vermietung von Kapital,  Bonds und Staatsanleihen finanzieren m\"{u}ssen $\hookrightarrow$ Nutzenmaximierung
  \item Unternehmen produzieren eine Vielfalt von Produkten mithilfe von gemieteten Kapital und Arbeit. Sie verf\"{u}gen \"{u}ber Marktmacht \"{u}ber ihr Produkt und legen den Preis hierf\"{u}r fest. Alle Zwischeng\"{u}ter werden zur Herstellung des Konsum- und Investitionsguts ben\"{o}tigt. $\hookrightarrow$ Kostenminimierung und Gewinnmaximierung
  \item Geldpolitik folgt einer Feedbackregel: Nominalzins reagiert auf Abweichungen der Inflationsrate vom Inflationsziel und des Outputs vom potential Output
  \item Fiskalpolitik erhebt Steuern (Haushalte und Unternehmen) um seine Staatsausgaben (potentiell Nutzen-stiftend) und Staatsinvestitionen (potentiell Produktivit\"{a}tssteigernd)zu finanzieren. Zus\"{a}tzlich kann sich der Staat verschulden und emittiert Schuldtitel.
  \item Beziehung zum Ausland zum Handel von G\"{u}tern und Kapital
  \item und beliebige Erweiterungen...
\end{itemize}
\end{frame}

\section{Deterministische vs. Stochastische Modelle}
\begin{frame}\frametitle{Deterministische vs. Stochastische Modelle}
	\begin{itemize}
		\item Grundlegende Unterscheidung: Sind zuk\"{u}nftige Schocks bekannt?
		\begin{itemize}
			\item Deterministisches Modell: Auftreten aller zuk\"{u}nftiger Schocks ist genau bekannt (Zeitpunkt und Auspr\"{a}gung); n\"{u}tzlich:
            \begin{itemize}
			\item f\"{u}r Modelle mit vollen Informationen, perfekter Voraussicht und keiner Unsicherheit gegen\"{u}ber Schocks.
			\item f\"{u}r Modelle mit Fokus auf Wechsel in Regimes, z.B. Einf\"{u}hrung einer neuen Steuer.
			\item bei Schocks, die irgendwann in der Zukunft, in einer oder mehrerer Perioden auftreten.
			\item da L\"{o}sung exakt und nichtlinear mithilfe numerischer Techniken berechnet werden kann.
			\item im praktischen Gebrauch, um einen ersten Eindruck des Modells zu bekommen.		
            \end{itemize}
			\item Stochastisches Modell: Nur die Verteilung von zuk\"{u}nftigen Schocks ist bekannt; n\"{u}tzlich:
            \begin{itemize}
            \item da sch\"{a}tzbar!
            \item bei Schocks, die \"{u}berraschend auftreten sollen.
            \item da beliebter in der Literatur: RBC Modelle oder Neu-Keynesianische Modelle
            \end{itemize}
		\end{itemize}
	\end{itemize}
\end{frame}
\note{Let's consider a shock to a model's innovation only in period 1. In a deterministic context, agents will take their decisions knowing that future values of the innovations will be zero in all periods to come. In a stochastic context, agents will take their decisions knowing that the future value of innovations are random but will have zero mean. This isn't the same thing because of Jensen's inequality.\\
A second order approximation will instead lead to very different results, as the variance of shocks will matter.}


\section{Arbeitspferde}
\begin{frame}
\frametitle{Arbeitspferde}\footnotesize
\begin{itemize}
  \item Smets/Wouters (2007, AER)
  \begin{itemize}\tiny
  \item Grundmodell mit vielen Friktionen, gesch\"{a}tzt
  \end{itemize}
  \item Global Integrated Monetary and Fiscal Model (GIMF) vom IWF
  \begin{itemize}\tiny
    \item Gro{\ss}es Model mit nicht-Ricardianischen,  Generationen\"{u}bergreifenden Haushalten mit endlichem Planungshorizont, detaillierte Fiskalpolitik, viele L\"{a}nder und Blocks; kalibriert
  \end{itemize}
\item SIGMA von der FED
\begin{itemize}\tiny
\item Sieben L\"{a}nder Modell zur Analyse US spezifischier Fragestellungen (insbesondere \"{O}lpreis); kalibriert
\end{itemize}
\item New Area Wide Model (NAWM) von der EZB:
\begin{itemize}\tiny
\item Zwei L\"{a}nder Modell, beides Mitglieder der EWU, sehr detailierte Fiskalpolitik; gesch\"{a}tzt
\end{itemize}
\item Euro Area Global Economy Model (EAGLE) von der EZB:
    \begin{itemize}\tiny
    \item Vier L\"{a}nder Model: Zwei L\"{a}nder Mitglied der EWU, USA und Rest der Welt, verschiedene Versionen (detaillierte Arbeitsmarktstruktur, Bankensektor, Fiskalpolitik); kalibriert
    \end{itemize}
\item Quest von der Europ\"{a}ischen Kommission
\begin{itemize}\tiny
    \item Mehrl\"{a}nder-Modell mit Wachstum und Generationen\"{u}bergreifenden Haushalten; gesch\"{a}tzt
\end{itemize}
\item GEAR von der Bundesbank:
\begin{itemize}\tiny
\item Drei Regionen Modell (Deutschland, Europa, Rest der Welt) mit detailiertem Fiskalpolitischen Block, Arbeitslosigkeit; gesch\"{a}tzt
\end{itemize}
\item Weitere
\begin{itemize}\tiny
\item PESSOA (Bank of Spain); ToTEM (Bank of Canada), BEQM (Bank of England), Aino (Bank of Finland), RAMSES (Riksbank), NEMO (Norges Bank), MAS (Bank of Chile),...
\end{itemize}
\end{itemize}
\end{frame}

\end{document} 